% aastex6.1 will be released as part of TeXLive 2017 around June 1. 
\documentclass[manuscript]{aastex6}

% Make the underscore behave correctly.
\usepackage[T1]{fontenc}
% Import images
\usepackage{graphicx}
% Used for consistent handling of Figures/Sections/Tables
\usepackage[capitalize]{cleveref}
% Format URLs correctly
\usepackage{url}
\usepackage{amsmath}
\usepackage[T1]{fontenc}
\usepackage{aecompl}
\usepackage{color}

% Want to have a separate storage folder for my figures.
\graphicspath{{./fig/}}

% Settings for cleveref to make Figures and Tables more appropriately
% defined.
% I want figure to be abbreviated.
\crefname{figure}{Fig.}{Figs.}

\newcommand{\vsini}{\ensuremath{v \sin i}}
\newcommand{\Kepler}{\mbox{\textit{Kepler}}}
\newcommand{\Gaia}{\mbox{\textit{Gaia}}}
\newcommand{\McQuillan}{\citep{McQuillan14}}
\newcommand{\Teff}{\ensuremath{T_{eff}}}
\newcommand{\logg}{\ensuremath{\log(g)}}
\newcommand{\kms}{\textrm{~km~s}\ensuremath{^{-1}}}

% Targeting flags
\newcommand{\STARBAD}{\texttt{STAR\_BAD}}
\newcommand{\STARWARN}{\texttt{STAR\_WARN}}
\newcommand{\VSINIWARN}{\texttt{VSINI\_WARN}}
\newcommand{\APOGEESEISMO}{\texttt{APOGEE2\_SEISMO\_DWARF}}

\newcommand{\gvs}{\authorcomment1}

\shorttitle{Comparison of Rotation Measurements in \Kepler{}}
\shortauthors{Simonian et al.}
\begin{document}

\title{A Comparison between Spectroscopic and Photometric Rotation in the
\Kepler{} Field}
\author{Gregory V. Simonian; Marc H. Pinsonneault; Donald M. Terndrup}
\affil{Department of Astronomy, The Ohio State University}
\affil{140 West 18th Avenue, Columbus, OH 43210}
\email{simonian@astronomy.ohio-state.edu}

\begin{abstract}
    The \Kepler{} fields have yielded a large data set of photometric rotation
    periods, which when combined with \Gaia{} parallaxes and APOGEE
    high-resolution spectroscopic data for over 4,000 dwarfs and subgiants, 
    can reveal significant insights into the evolution of rotation in stars, 
    We find that even with radii derived from \Gaia{} DR2 parallaxes and
    spectroscopic parameters, uncertainties caused by confusion between the
    main sequence, subgiants, and photometric binaries lead to difficulties
    interpreting stellar rotation for hot (\(\Teff > 5500\) K) stars. We 
    report on spectroscopic measurements of stellar rotation from APOGEE in an
    easily-characterized sample of over 600 cool dwarfs and subgiants in the
    \Kepler{} field. Our data are in good agreement with literature values for
    targets in common, with a detection threshold between 7 and 10 \kms. We
    find a rapid rotation fraction of \(\sim 10\%\), a larger than expected
    population of young stars, an enhancement we attribute to rapid rotation
    being associated with photometric binaries. When comparing spectroscopic 
    to photometric rotation rates, we find that the photometric rapid rotator 
    fraction is lower than the spectroscopic fraction by a factor of 4--5; a 
    point-by-point analysis reveals that almost all targets have rotation 
    periods too long for their measured \vsini{}. We find these results 
    suggest that either significant contamination of field \vsini{}
    measurements by photometric binaries with blended spectral features, or 
    current large-scale photometric surveys may mis-characterize or exclude 
    cool rapid rotators in the field. 
\end{abstract}
\keywords{stars: rotation}

\section{Introduction}

One of the significant results from the \Kepler{} satellite has been the
determination of rotational periods for a large sample of old, field stars
\citep{Basri11,Affer12,Nielsen13,Reinhold13,McQuillan14,Garcia14}.
These rotation rates are expected to be extremely useful to deriving a field
gyrochronology, which can better characterize the ages of
\Kepler{} stars, and hence better understand discovered planets.

The \Kepler{} field was chosen to point outside of the plane of the galactic
disk, biasing the field toward older stars. 

Current attempts to synthesize rotation information with known ages of
asteroseismic targets in the \Kepler{} field have consistently found a dearth
of long-period (\(P_{rot} \sim 60\) day) rotators \citet{Angus15,vanSaders16}. 
Forward modeling of the \Kepler{} field using existing rotation-activity
relations and estimates of the \Kepler{} selection function further confirm 
that models significantly under-predict the number of (\(P_{rot} > 30\) day) 
rotators in the \Kepler{} field \citep{vanSaders18}. Proposed solutions for 
relieving this tension have been modifying the activity relation in models to 
drop precipitously at a particular rotation threshold, or modifyling angular 
momentum loss to shut off at \(P_{rot} \sim 30\) days \citep{vanSaders18}.

Despite the old age of the
field, there is still a population of rapid rotators at the level of 1\% 
which have rotation rates significantly higher than expected from
existing gyrochronological relations \citep{McQuillan14}. Possible explanation
for these rapid rotators are a truly young stellar population, a background of
tidally-synchronized binaries, contamination of background pulsators in the
\Kepler{} PSF, or confusion with rapidly-rotating subgiants \citep{vanSaders13}. 

All attempts to evaluate the determination of rotation periods in
\Kepler{} field stars have been comparisons between different pereiodograms and
light-curve processing algorithms. The combination of large-scale
high-resolution spectroscopy from APOGEE and uniform radii derived from precise
\Gaia{} parallaxes \citep{Stevens17} will allow for more powerful tests of our
ability to measure stellar rotation, and our interpretation of stellar rotation
data, especially in old, inactive stars where data is sparse.

Previous studies have succesfully combined rotational periods and \vsini{} 
to study systematics in the radii of pre-main sequence stars
\citep{Rhode01,Jeffries07}, the inclination distribution in clusters 
\citep{Jackson10}, and the inference of radius inflation
\citep{Jackson09,Jackson16,Jackson18}. All of these studies were done for
clusters, where large numbers of spectra could be observed and analyzed under 
the assumption of uniform distances and composition. The availability of
rotation periods and \vsini{} for field stars will be instrumental in 
illuminating other phenomena such as gyrochronology 
\citep{Barnes07,Mamajek08,Angus15}, radius inflation \citep{Jackson18}, and 
rotation on the subgiant branch \citep{vanSaders13}.

In this paper, we study the rotation distribution of \Kepler{} targets as
seen from APOGEE \vsini{}s and compare to published periods. In 
Section~\ref{sec:sample} we describe the
selection of cool dwarfs to ensure a reliable comparison between
\vsini{} and period. In Section~\ref{sec:analysis}, we describe the reliability
of the quantities we need in order to compare rotation via \vsini{} and
photometric modulation. We also demonstrate our methodology works with the 
APOKASC asteroseismic sample. In Section~\ref{sec:fraction}, we compare the 
rapid rotator fraction from photometric studies to that found using APOGEE 
spectra. In Section~\ref{sec:results}, we compare the two measures of 
rotation on a point-by-point basis and postulate reasons for disagreement. 
Lastly, we summarize our findings and discuss future efforts in 
Section~\ref{sec:conclusions}.

\section{Sample Selection}
\label{sec:sample}

\begin{figure*}
    \plottwo{astero}{cool_mk_sample}
    \caption{\emph{Left:} Stellar properties of the full APOKASC 
    asteroseismic dwarf/subgiant control sample. The black points mark
    K-band absolute magnitudes for the full control sample. Green stars mark
    the subsample showing oscillations. Rapid rotators with 
    \(\vsini > 10 \kms\) are denoted as dark blue circles, while stars with 
    marginally detected rotation \(7 \kms < \vsini < 10 \kms\) are denoted 
    with light blue circles. Objects classified as showing double-lines in 
    their APOGEE spectra are labeled as salmon stars.
    \emph{Right:} The full cool dwarf sample as measured with APOGEE stellar 
    parameters. The stellar classifications follow that in \citet{Berger18},
    with dwarfs, subgiants, and giants marked as teal, purple, and mustard
    dots. Rapid rotators are classified similarly to the previous figure. 
    Objects showing rapid (\(P_{rot} < 3 \) day) photometric rotation are 
    indicated with purple diamonds. Objects showing double-lined APOGEE 
    spectra are indicated as salmon stars. The dotted vertical 
    line indicates the \(\Teff = 5450\) K boundary.\label{fig:sample}}
\end{figure*}

The central dataset for this analysis is from the Apache Point Observatory 
Galactic Evolution Experiment (APOGEE) \citep{Blanton17,Majewski17}. The
APOGEE instrument is a high-resolution (\(R \sim 22,000\)) multi-fiber 
near-infrared spectrograph \citep{Wilson10}, mounted on the SDSS 2.5-meter 
telescope at Apache Point Observatory \citep{Gunn06}. Observations for each
target are automatically reduced and combined as part of the APOGEE data 
pipeline \citep{Nidever15}. Stellar parameters are then determined from the 
combined spectra through the APOGEE Stellar Parameters and Chemical Abundance 
Pipeline (ASPCAP), which outputs \Teff{}, \logg{}, \([M/H]\), \vsini{}, as 
well as 15 additional chemical abundances \citep{GarciaPerez16}.

Under several major observing programs, APOGEE systematically observed over 
5,000 dwarfs and subgiants in the \Kepler{} field. Around half were
observed as part of a program to determine false positives for
asteroseismic
detections\footnote{\url{http://www.sdss.org/dr14/irspec/targets/\#AsteroseismicTargets}}. 
They were part of a magnitude-limited sample
selected according to 6500 K \(> \Teff > 5000\) K and \(\logg > 3.5\)
according to parameters derived by \citet{Huber14} \citep{Zasowski17}.
This sample is shown in the left panel of \cref{fig:sample}. An additional 
1000 cool dwarfs with (\(\Teff < 5500\) K) and \(\logg > 4.0\) as determined
by \citet{Pinsonneault12} and \citet{Brown11} were also 
observed\footnote{\url{http://www.sdss.org/dr14/algorithms/ancillary/apogee/kepler_dwarfs/}}, 
and the full sample is shown in the left panel of \cref{fig:sample}. 
Eclipsing
binaries\footnote{\url{http://www.sdss.org/dr14/irspec/targets/\#EclipsingBinaries}} 
and transiting planet host 
candidates\footnote{\url{http://www.sdss.org/dr14/irspec/targets/\#KeplerObjectsofInterest}} 
were also observed 
by APOGEE in the \Kepler{} field, but with more complex selection functions, 
which we didn't include in this analysis.

\subsection{Hot Dwarfs}

We selected our sample to minimize ambiguity in the stellar radius when
comparing rotation periods to \vsini. The hot (\(\Teff > 5500 K\)) star regime
in particular contains significant confusion between main-sequence stars,
photometric binaries, and subgiants crossing the Hertzsprung Gap.
In order to sidestep detailed modeling of the complex radius uncertainties 
for all \Kepler{} targets, we chose instead to study measures of
stellar rotation in the cool dwarf regime, where
slow stellar evolution allows the radii to be easily characterizable through
stellar models, and the cool subgiants where photometric binaries are not a
contamination source. 

\subsection{Cool Dwarfs}

For our base sample, we chose objects observed as part of the APOGEE cool
dwarf ancillary targeting
program
This program targeted stars observed by \Kepler{} with SDSS-corrected
\(\Teff < 5500\) K \citep{Pinsonneault12}, original KIC \(\logg > 4.0\)
\citep{Brown11}, and 2MASS \(7 < H < 11\). While many of these objects
were observed in the first phase of APOGEE with the
\textit{APOGEE\_KEPLER\_COOLDWARF} targeting flag enabled, much of the rest of
the sample was observed in APOGEE-2 under the \textit{APOGEE2\_SEISMO\_DWARF}
flag. We applied the original selection criteria to the new sample in order get
the updated target list. As of the DR14, the program has been \(\sim 85\)\% 
completed with 1028 objects having been observed. 

We also selected targets based on APOGEE stellar parameter fits.
Depending on the reduced chi-square of the atmosphere model fit, or
disagreement with photometric temperature relations, the ASPCAP pipeline
may trigger a flag indicating that the ouputted parameters may not be
reliable. We categorized the sample into those with \STARBAD, \STARWARN,
\VSINIWARN{}, and no flag enabled to check the quality of the fits. We
visually inspected the spectra for a representative sample of these
quality classes. 
Visual inspection of targets with the \STARBAD{} flag showed that they are 
largely the result of poor subtraction or normalization of the spectrum in the
pipeline, or are due to poor model fits on the cool (\(\Teff < 4250\) K) end. 
123 targets in the full sample have the \STARBAD{} quality flag enabled. 
Targets labeled as \STARWARN{} are a significantly larger fraction of the 
sample. We found that the model fits reasonably resembled the target spectra. 
For this reason, we retained targets with the \STARWARN{} flag enabled. The 
same occured for targets with the \VSINIWARN{} flags.

We supplement the stellar parameters provided by APOGEE with distances and radii from the 
\Gaia{} satellite \citep{Gaia16} Data Release 2 \citep{Gaia18}, reanalyzed 
using the methodology of \citet{Huber14} \citep{Berger18}. In summary, this
procedure derived new stellar parameters based on priors informed from stellar
parameters in the Kepler Stellar Parameter Pipeline DR25 \citep{Mathur17}.
These parameters were ultimately derived from KIC photometry \citep{Brown11},
with more precise inputs where applicable.

We classify our sample into dwarfs, subgiants, and giants according to
K-band absolute magnitude from 2MASS \citep{Skrutskie06}. The full sample is
shown in the left panel of \cref{fig:sample}. The dwarf and subgiant branches
show a clean separation in K-band absolute magnitude at \(M_K = 2.4\) mag,
which we use as our cut criterion. To separate subgiants from giants, we use a
absolute magnitude cut of \(M_K = 0.7\) mag, which roughly corresponds to the
luminosity of objects with \(\logg = 3.5\), which we define as our subgiant
criterion. We these K-band absolute magnitude lead to a sample of 654 dwarfs 
and 91 subgiants. \gvs{Check this}

To cross-check the validity of this classification, and to ensure that the Gaia
data do not suffer from large systematics, we perform a similar separation
using raw APOGEE \logg{} values to check how well the two methods agree. When
APOGEE \logg{} is plotted against \Teff{}, the subgiants and dwarf
qualitatively behave like two separate branches, despite the branches in detail
behaving unphysically.

To control for differences between the photometric and spectroscopic sample,
we performed the bulk of our analysis on the subset of 408 objects
observed in APOGEE which also have period detections in \citet{McQuillan14}.
An additional 205 did not have significant periods in \citet{McQuillan14}, and
will be analyzed separately. The remaining 44 targets were excluded by
the original sample selection in \citet{McQuillan14}. 

We also inspected the spectra for every target with \(\vsini > 6 \kms\)
for the presence of double lines. Most targets showed obvious cases of ASPCAP 
fitting a single broad line over two separate narrow features, other cases 
only revealed the second component in individual visit spectra. The targets 
showing double-lined spectra were excluded from the analysis because of
the unreliability of the catalog vsini. More sophisticated future
analyses \citet{ElBadry18} may be able to disentangle \vsini{}
measurements from each component. There are
also cases where the observed spectra show high RV variability
without a visible second component. We include these cases in our sample
because the measured \vsini{} shouldn't be impacted.


We also found that stars hotter than
\(\Teff = 5450\) K were significantly more difficult to characterize due to
rapid age evolution and overlap with subgiants. Because only a small 
fraction of our sample has \(\Teff > 5450\) K, we removed targets hotter
than cut. The effect of the cut on the sample is shown in \cref{fig:sample}.

\subsection{Asteroseismic Targets}

One of the most well-characterized subsets of the \Kepler{} field is the
APOKASC asteroseismic sample. The APOKASC asteroseismic dwarf and subgiant 
sample consists of 426 targets which presented solar-type oscillations in 
\Kepler{} short-cadence data \citep{Chaplin11}, and were observed in APOGEE DR13
\citep{Majewski17}. Using asteroseismic scaling relations, the radii
for these objects can be determined to an accuracy of around 3\%
\citep{Serenelli17}. Because of their excellent parametrization, we treat the
asteroseismic catalog as a control sample, where stellar properties are
well-known. It should be noted that subgiants are overrepresented in
the asteroseismic population because the amplitude of solar-type oscillations 
increases as stars evolve. The locations of the asteroseismic targets in an HR
diagram using both the asteroseismic and spectroscopic \logg{} is shown in 
the left panel of \cref{fig:sample}.

The selection of the asteroseismic targets mirrors that of the cool dwarf
sample. From the 426 asteroseismic dwarfs from the APOKASC v4.2.4 catalog, we 
removed three objects with the STAR\_BAD flag, while including 51 objects with 
the \STARWARN{} and 46 objects with the \VSINIWARN{} flags after visual 
inspection. All targets with \(\vsini \ge 6\) \kms{} were visually
inspected for double-lined spectra leading to four being excluded. 91 of 
the remaining stars have period detections in \citet{McQuillan14}, 240 were 
period nondetections, and 88 did not fall under the selection criteria in 
\citet{McQuillan14}. 

\section{Analysis}
\label{sec:analysis}

In order to consistently compare different measures of rotation, we require 
reliable \vsini{}, rotational periods, and stellar radii. We perform validation
exercises to make sure that the quantities adopted for these three measures are 
reasonable.


\subsection{\vsini{} validation}
\label{sec:vsini_check}

\begin{figure*}
  \gridline{\fig{Bruntt_comp}{0.3\textwidth}{(a)}
  \fig{Pleiades_comp}{0.3\textwidth}{(b)} 
  \fig{astero_rot}{0.3\textwidth}{(c)}}
    \caption{\emph{Left:} \vsini{} comparison between the \citet{Bruntt12}
        overlap sample with APOGEE\@. A discontinuity in the scatter occurs
        around \(\vsini = 7 \kms\), indicated by the dotted line. The dashed
    line shows the best-fit relation between the two. Not shown are targets 
    run through the APOGEE giant grid. \emph{Middle:} \vsini{} comparison for 
    the Pleiades cool dwarfs \citep{Stauffer87} overlap sample with APOGEE\@. 
    A discontinuity in the scatter occurs around \(\vsini = 12 \kms\), 
    indicated by the dotted line. 2MASS J03475973+2443528 is not shown
    because \citet{Stauffer87} flagged it as a possible SB2. Red points are 
    upper limits in \citet{Stauffer87}.\emph{Right:} Comparison between
    \vsini{} and equatorial \(v_{eq} = \frac{2\pi R}{P}\) for the 
    asteroseismic sample. Dark blue points correspond to confirmed
    \vsini{} detections while light blue points correspond to marginal
    \vsini{} detections. The lines corresponding to \(\sin i = 1\) and
    \(\sin i = 0.5\) are denoted as solid and dashed lines. The hatch
    marks denote the forbidden region where \(\sin i > 1\).\label{fig:comps}}
\end{figure*}


The ability to measure \vsini{} is relatively new to the ASPCAP pipeline, only
being included since DR13. As a result, catalog \vsini{}s have not been 
extensively vetted and verified. APOGEE does not provide a reasonable
detection limit, or an estimate on the uncertainty of \vsini{}. We
attempt to characterize the APOGEE \vsini{} measurements ourselfs
by comparing to literature \vsini{} for targets studied with high-resolution 
optical spectroscopy. Our method of determining the detection limit and
uncertainty for \vsini{} is similar to that performed in
\citet{Tayar15}.

One large collection of \vsini{}s comes from an effort to study the
\Kepler{} asteroseismic targets \citep{Bruntt12}. \citet{Bruntt12} analyzed 
spectra of 93 solar-type asteroseismic targets using two R=80,000 spectrographs 
on the 3.6-meter Canada--France--Hawaii Telescope and 2-m Bernard Lyot
Telescope. While uncertainties are not quoted in \citet{Bruntt12},
previous analyses have determined \vsini{} uncertainties of about 0.6
\kms \citep{Bruntt10a,Bruntt10b}.

A comparison between the APOGEE and \citet{Bruntt12} targets is shown in
\cref{fig:comps}. Not shown are six
objects which the ASPCAP pipeline classified as giants, and omitted
\vsini{}. The \citet{Bruntt12} \vsini{} is less than 5
\kms{} for all of these objects, again making them consistent with 
nondetections.  In order to characterize the APOGEE \vsini{}s, we model the 
two datasets as being linearly related above a specific detection threshold 
with a constant fractional uncertainty. The sample shows a 
discontinuity in the scatter at APOGEE \(\vsini=6\kms\), which we 
interpret as the detection limit of APOGEE, above which \vsini{} yields a 
valid stellar rotation measurement. Since the \citet{Bruntt12} errors are
significantly smaller than the expected APOGEE errors, we performed a simple 
linear regression for all targets with APOGEE \vsini{} above the detection 
limit to obtain the relationship. We got that the two datasets have slopes
consistent with one, but with a significant zero-point offset of 9\%,
with the APOGEE \vsini{} being lower than the \citet{Bruntt12}
\vsini{}. We estimated the measurement error by calculating the mean squared 
error of the residuals, yielding a 13\% uncertainty which we adopted as the 
uncertainty in APOGEE \vsini{}.

There is little overlap between our cool dwarfs and the 
\citet{Bruntt12} sample because the asteroseismic targets tend to be massive and 
highly populated by subgiants. In order to perform a check with a more
representative sample, we compared APOGEE \vsini{}s with  
measurements of cool dwarfs in the Pleiades \citep{Stauffer87}. Because
Pleiades dwarfs are significantly younger than those in the field,
they rotate much more rapidly. This is advantageous because their 
wide distribution of \vsini{}s allows for a wide range of \vsini{}s to be 
validated. The \citet{Stauffer87} survey reported a detection limit of
10 \kms, which should lie below our reported limit. We plot the overlap 
sample with \citet{Stauffer87} in \cref{fig:comps}.  Unlike the asteroseismic 
targets, the Pleiades targets experience a discontinuity in scatter at 
\(\vsini = 12 \kms\).  Nonetheless, the errors and fit seem consistent above 
the detection threshold.

The difference in detection limit between the Pleiades and asteroseismic sample
indicates that the ability to measure \vsini{} changes is a function of stellar
parameters. In order to ensure that our results are robust to the choice of
detection threshold, we
perform our analyses with different detection thresholds to show that any
results are persistent.

\subsection{Period Validation}

The validation of rotational periods in the \Kepler{} field has been a
well-studied problem. There have been multiple different approaches to
measuring rotation periods \citep{Reinhold13,Nielsen13,McQuillan14,Garcia14}.
\citet{McQuillan14} compared their periods to those of previous work
\citep{Reinhold13,Nielsen13}. The overlap samples with
\citet{Reinhold13} and \citet{Nielsen13} are 20,009 and 10,381,
respectively. Within these overlaps, the different methods get
consistent periods 65\% and 94\% of the time, respectively, and detect
aliased periods 7\% and 2\% of the time. The large discrepancy with
\citet{Reinhold13} is attributed to the short baseline of one \Kepler
quarter, in contrast to the 12 used by \citet{McQuillan14}.

The completeness of \citet{McQuillan14}, on the other hand, is fairly
good. For targets which had periods reported by \citet{Reinhold13},
\citet{McQuillan14} reported missing 1503 of them, leading to a detection
discrepancy of 7\%. The discrepancy with \citet{Nielsen13} is 2\%.

A comprehensive, hounds-and-hares exercise analyzed the performance of
several different periodograms on injected stellar rotation signals 
\citep{Aigrain15}. The Autocorrelation Function method used in 
\citet{McQuillan14} was found to have a 100\% detection rate on noisy
data while still maintaining around 70 percent reliability in correctly
identifying an injected signal. In contrast, \citet{Nielsen13} has a low
detection fraction of 16\%, but 100\% reliability \citep{Aigrain15}.
\citet{Aigrain15} identifies the most prominent failure modes: detection
of the spot lifetime instead of the rotational period, and detection of
the stellar cycle period for stars with short cycles.

Despite the comprehensiveness of the hounds-and-hares exercise, it has
several limitations. The light curves are idealized with a single
stellar signal, and do not take into account blended signals with
multiple potential periodicities. The injected signals also did not
include highly active stars and stars with persistent spots, which are
expected to be overrepresented among rapid rotators. It was also
difficult to compare the performances of each algorithm specifically in
the rapidly-rotating regime without the underlying data.

\subsection{Comparison for Asteroseismic Targets}
\label{sec:astero}

We first validated our comparison methodology on asteroseismic targets. We 
compared the measured \vsini{} to the equatorial velocity calculated
using the rotational periods and the asteroseismic radii. The resulting 
comparison is shown in the right panel of \cref{fig:comps}. We distinguished 
between confident \vsini{} detections with \(\vsini > 12 \kms\), and marginal 
detections with \(6 \kms < \vsini < 12 \kms\). Of the confident and
marginal detections, two and three, respectively, have non-physical solutions 
with \(\sin i > 1\). 

The marginal detections with unphysical \vsini{} (KIC 8677016, 8952800, 10518551) 
have \vsini{} values close to the detection threshold. Because the errors of 
marginal \vsini{} detections are complicated, it may be possible that these 
are outliers. 

Of the confident detections with unphysical \vsini{}, KIC 3223000 has a
\vsini{} which is slightly above the predicted equatorial velocity. It
is a 1.7-\(\sigma\) difference based on the \vsini{} uncertainty, which
may be consistent as a statistical fluctuation. It can also
be explained by the 4.9 day measured rotation period actually being a 
half-period detection alias, implying a twice as large equatorial velocity. 

The final outlier, KIC 9908400, is so significantly off of the relation
that it can not be explained by either uncertainties or aliasing. However,
it may be adequately explained by contamination of the \Kepler{} PSF\@. 
High-resolution J-band imaging of the \Kepler{} field, provided by the United
Kingdom InfraRed Telescope (UKIRT) reveals a bright, nearby blended companion.
Additionally, KIC 9908400 has a high \Teff{} placing it above the Kraft
break, where convective envelopes are thin;
making spots unlikely. Photometric variation originating from the fainter 
star would explain the mismatch in \vsini{} and rotation period.

For the rest of the objects, a more comprehensive upcoming analysis 
(Simonian et al.\ in prep) indicates that not only are the \vsini{} values 
physically plausible, but their distributions are consistent with the
rotational periods within the given uncertainties.

\subsection{Photometric Binaries}

\Gaia{} parallaxes provide a valuable way of separating photometric binaries
from the rest of the sample. As can be seen in \cref{fig:sample}, photometric
binaries clearly stand out approximately 0.75 mag above the main sequence, a
level significantly higher than the typical uncertainty in absolute magnitude. 
We attempt to separate the photometric binaries from the main sequence by
comparing the observed K-band absolute magnitude to that predicted by DSEP
isochrones.

The primary uncertainties in the predicted isochrone are the metallicity and
age. The availability of APOGEE metallicities throughout the sample provides
significantly improved constraints on whether targets are photometric binaries
or not. For derive K-band absolute magnitudes by inputting APOGEE
\Teff{} and [Fe/H] values, and assuming solar alpha abundance on a grid of
models with metallicity differences of \([Fe/H] = 0.05\). The resulting offset
between observed and calculated K-band absolute magnitude is shown in 
\cref{fig:binarity}. 

The effect of age can be seen as the temperature-dependent scatter on the main
sequence. Based on DSEP-modelled K-band luminosity differences between 1--10
Gyr, the scatter due to age ignorance can be up to 0.25 mag at \(\Teff \sim
5500\) K, down to 0.05 mag at \(\Teff \sim 4000 \) K. For this reason, we adopt
a more conservative cut at the hot end to avoid misclassifiying evolved single
stars, with a more generous cut at the cool end, where age effects are less
significant.


\subsection{Stellar Radii Validation}
\label{sec:radii}

The main property of cool stars we exploited is that their slow age evolution
constrains stellar radii. We quantified this uncertainty using isochrones from 
the Dartmouth Stellar Evolution Program (DSEP) to estimate the impact of age 
ignorance on the radius uncertainties. We defined our uncertainty as the 
difference in radius between stars of similar \Teff{}, with ages between 1 
and 10 Gyr. For solar metallicity models, the uncertainty in radius due to 
age is about 8.5\% at the hot end of our sample, and decreases sharply with 
\Teff. 

Error bars for the individual stellar radii follow the same convention as
above, except are based on the individual star's temperature and metallicity 
at 1 Gyr and 10 Gyr. Because the ``global'' APOGEE uncertainty for [Fe/H] is 
around 0.05 dex, we measured radii on an [Fe/H] grid spaced by 0.05 dex.

A significantly larger systematic uncertainty than the impact of age is the 
misclassification of
subgiants as dwarfs. This uncertainty would certainly lead to underestimating
the stellar radii of dwarfs by several factors. We estimated this
contamination rate by using the asteroseismic sample, rich in
well-characterized subgiants, shown in the top right
panel of \cref{fig:sample}.

The asteroseismic sample allows for a test of contamination because each
object has an asteroseismic \logg{}, derived from the light curve and is
expected to be extremely accurate, and a spectroscopic \logg{} which is
less accurate, but independently determined. In order to obtain a sample of 
contaminating subgiants representative of our cool dwarf sample, we
selected only those with \(\Teff < 5500\) K, resulting in 317
asteroseismic targets. Of these, all but eight have asteroseismic \logg{} on 
the subgiant branch, the rest being dwarfs. None of the asteroseismic
subgiants have corresponding spectroscopic \logg{}s on the 
dwarf branch. There are asteroseismic dwarfs which APOGEE places on the
subgiant branch, but those are not concerning because they will not act as 
contaminants in our dwarf sample. By approximating the evolutionary-state 
classification as a random process following a binomial distribution, we
expect zero subgiant contaminants in our sample, but can infer a 1-\(\sigma\) 
upper limit of 2.4 subgiant contaminants in the full cool dwarf sample
\citep{Gehrels86}, which is significantly less than the number of rapid
rotators observed. 

\section{Rapid Rotator Fractions}
\label{sec:fraction}

\begin{figure}[htb]
    \plotone{detection_fraction}
    \caption{Cumulative distribution function of \vsini{} for the single and
        binary samples for spectroscopic and photometric rotation data. The 
        rapid rotator fraction at a given \vsini{} threshold
    would be one minus the value of the cumulative distribution function
    at that \vsini{} value. Solid lines denote the cumulative
    distribution function of APOGEE \vsini{} for the subsample with rotation
    periods. Dashed lines denote  
    the cumulative distribution function of \(v_{eq}\) derived from the
    rotation periods and stellar radii. Dotted lines denote the
    cumulative distribution function of \vsini{} for the subsample without
    rotation periods. The teal and green colors denote rapid rotator fractions
in the single-star and photometric binary sequences. Error bars represent 
    1-\(\sigma\) binomial confidence levels.\label{fig:detection_fraction}}
\end{figure}

The broadest check to determine agreement between the spectroscopic
and photometric rotation data is comparing the rapid rotator fraction
in both datasets. We defined spectroscopic rapid rotators in this exercise as 
stars having \vsini{} above the detection threshold of the APOGEE 
spectrograph. As mentioned in 
Section~\ref{sec:vsini_check}, APOGEE \vsini{} measurements have not been fully
characterized, and the detection limit could reasonably be placed anywhere 
between 6-12 \kms{}. We report rapid rotator fractions over multiple detection 
thresholds to ensure the robustness of our results.

The cumulative \vsini{} distribution of our sample is shown in
\cref{fig:detection_fraction}, with each bin corresponding to the
fraction of targets with APOGEE \vsini{} less than the given
\vsini{}. Error bars were chosen to be binomial 1-\(\sigma\) confidence 
intervals based on selecting the number of rapid rotators from the full 
sample.  As such, the rapid rotator fraction is the complement of that
value. Depending on the choice of detection threshold, the rapid rotator 
fraction ranges from 26\% for a detection threshold of 6 \kms, to 7\%
for a detection threshold of 12 \kms.

In order to obtain a photometric rapid rotator fraction corresponding to
a \vsini{} detection limit, we transformed the \vsini{} threshold into period 
space. The relationship between rotational period and equatorial velocity is 
given by the relation 
\begin{displaymath}
    v_{eq} = \frac{2 \pi R}{P} 
\end{displaymath}
where \(R\) is the equatorial radius and \(P\) is the rotation period. As we
noted in Section~\ref{sec:radii}, there is a relatively clean and narrow
relationship between radius and \Teff{} on the main sequence. We used the 
DSEP-derived radius for each individual object to transform  between
velocity and period coordinates. The fraction of objects with period
short enough to produce detectable \vsini{} is shown as the solid red
line in \cref{fig:detection_fraction}. The photometric rapid rotator fraction
ranges from 5\% at a detection limit of 6 \kms down to 2\% at a 
detection limit of 12 \kms.


Due to the \(\sin i\) ambiguity, there isn't a direct correlation
between a \vsini{} cut and the corresponding period cut given a stellar radius.
Many rapid rotators in period-space scatter into being non-detections in
velocity space after convolution with the inclination distribution. However,
this distribution can be statistically characterized for randomly
aligned systems. Since our underlying sample was chosen based on
\Teff{} and \logg{}, we don't expect a strong bias in inclination.
Therefore, we perform a statistical correction of \(\pi/4=0.785\)\footnote{We 
  note that the presence of starspot variability biases the
    inclination distribution against face-on alignments. Measurements
    in clusters indicate that spots are not seen for inclinations where 
    \(\sin i < 1/2\) \citep{Jackson10}, whose exclusion only changes the 
correction factor to 0.740} 
to account for photometric rapid rotators scattering to low
\vsini{}. The  corrected distribution is shown as the red dotted line in
\cref{fig:detection_fraction}, ranging from 3\% at the low dettection
limit to 1\% at the high detection limit.

Comparing these three rapid rotator fractions shows that the fraction of
spectroscopic rapid rotators is significantly larger than the fraction
of spectroscopic rapid rotators by a factor of 4-6. Accounting for
inclination effects makes the discrepancy even worse to a 4-\(\sigma\)
level. This discrepenacy is highly robust to the choice of
\vsini{} detection limit. Unlike  
the case with the asteroseismic targets, the fraction of spectroscopic rapid
rotators is greater than the fraction of photometric rapid
rotators at a highly-significant level, which is exacerbated afer taking the
inclination correction into account.

We validated this exercise by comparing the photometric and spectroscopic
rapid rotator fractions in the asteroseismic sample. In order to have the best
overlap sample, we used \vsini{} from \citet{Bruntt12} and rotation periods
from \citet{Garcia14}, whose sample for studying stellar rotation specifically
targeted asteroseismic dwarfs and subgiants. In the asteroseismic
sample, we found the expected result that the rapid rotator
fraction as measured spectroscopically and photometrically agreed to within
1-\(\sigma\) at all \vsini{} thresholds. Therefore, we believe this
discrepancy of rotation measures in the cool dwarfs is real.

We also checked the spectroscopic rapid rotator fraction of the 
\citet{McQuillan14} nondetections. We found that the spectroscopic rapid 
rotator fraction is significantly
lower in the nondetections than the overlap sample, but inconsistent with zero. 
A non-zero rapid rotator fraction is surprising because the nondetection
sample is expected to be comprised of slow-rotating inactive stars and 
high-inclination spotted stars, both of which are incompatible with low 
\vsini{} values. 

Previous studies have found an inverse correlation between 
photometric variability and Rossby number (defined as \(R_o = P / \tau_{cz}\), where \(P\)
    is the rotation period and \(\tau_{cz}\) is the color-dependent 
convective overturn timescale) \citep{Messina01,Hartman09}. The existence of 
stars with low Rossby number and low photometric variability may either 
indicate that additional factors determine photometric variability 
aside from the Rossby number, or that \citet{McQuillan14} is not detecting 
rapid rotation in genuine, spotted rapid rotators.



\section{Results}
\label{sec:results}

\begin{figure*}
  \plotone{cool_rot}
  \caption{\emph{Left:} APOGEE \vsini{} plotted against equatorial
  velocity computed from the rotation period and radius for targets with
  detected rapid rotation. The confirmed \vsini{} detections are shown
  in dark blue while the marginal \vsini{} detections are shown in light
  blue. The solid and dashed lines correspond to values where \(\sin i =
  1, 0.5\), respectively. The hatched area represents the forbidden
  region where \(\sin i > 1\). \emph{Middle:} Symbols are similar to left
  side, except points are projected such that the DSEP-derived radius is
  plotted against the radius inferred from \vsini{} and rotation
  period. \emph{Right:} Symbols are similar to left side, except points are
  projected such that the \citet{McQuillan14} period is plotted against the
  period inferred from the \vsini{} and radius.\label{fig:rot}}
\end{figure*}

After selecting dwarfs with detectable rotation in APOGEE,
visually inspecting for DLSBs, and cross-matching with \citet{McQuillan14}, we 
end up with 26 targets with confident \vsini{}, and 26 targets with marginal
\vsini{} where we can individually compare \vsini{} and rotation period. A 
point-by-point
comparison for these targets is shown in the bottom panels of
\cref{fig:rot}. The comparison between \vsini{} and equatorial velocity shows
that the discrepancy found in Section~\ref{sec:fraction} is not due to slight
but coherent model errors, or a few discrepant points. Unlike with the
asteroseismic sample in Section~\ref{sec:astero}, all but one of the targets
lie in teh firbidden \(\sin i > 1\) region affecting both the confident and
marginal detections. This is indicative of a systematic
bias where periods are consistently (and sometimes drastically) too slow to be
consistent with \vsini{}.

The observation that this discrepancy exists for cool dwarfs and not
asteroseismic stars is also a significant clue. It indicates that the
discrepancy is only apparent for certain parts of the HR diagram. The main
differences between the two are the temperatures, with asteroseismic targets
being significantly warmer than our cool sample, and evolutionary state, where
asteroseismic targets are subgiants and the cool sample consists of dwarfs.
Better radii will be needed to extend this analysis to additional targets.

The discrepancy between spectroscopic and 
photometric rapid rotation fractions can be seen in other published work as well. 
\citet{Nielsen13} compared their measurements of \Kepler{} rotation periods to
an amalgamation of \vsini{} from various clusters. They found that the
\vsini{} distribution was consistent with the period distribution for spectral
types F--K, and then the periods dropped below the \vsini{} measurements for
K--M (top panel of their Fig. 2). They ascribe the discrepancy to the cool
stars as a shift in the \Kepler{} age distribution between the hot and cool
stars; however, they claim to reproduce rotational periods with previous work. 
Additional confirmation occured through the APOGEE M-dwarf rotation
program \citep{Gilhool18}, which surveyed a color- and proper-motion-selected 
sample of nearby M-dwarfs. Unlike in this work, the \vsini{} were measured 
directly through the spectra as opposed to through ASPCAP\@. For the \Teff{} range that
overlaps with \citet{McQuillan14} \(4000 \textrm{ K } > \Teff > 3200\) K, they 
find that 5.8\% of their targets have \(\vsini > 8 \kms\). This is 
significantly greater than our photometric rapid rotator fraction at 8 \kms{} 
of 2.5\%. 

While spectroscopic rapid rotation didn't necessarily imply photometric
rapid rotation, we find that photometric rapid rotation is generally
reflected in spectroscopy. This can be seen in the right panel of
\cref{fig:periods}, where 7/8 of the targets with \(P_{rot} < 3\) day
have detectable spectroscopic rapid rotation. This suggests two
possibilities: the \citep{McQuillan14} method misclassifies rapid
rotators as slow, or that the APOGEE \vsini{} have significant false
positives.

The most obvious systematic error that could cause this problem is
misclassification of subgiants. Despite the asteroseismic sample
predicting that contamination should be low, it is possible that 
asteroseismic subgiants are not representative of the \Kepler{} subgiant 
population. Subgiants would have underestimated the radii, causing the 
measured \vsini{} to be larger than the inferred equatorial velocity.
The radii inferred from rotation are shown on the right side of
\cref{fig:rot}. Even if this
exceptionally pathological outcome were true, subgiant contamination
would not be able to account for most of the sample because reconcicling
rotation requires radii greater than four solar radii, atwhich point
stars would be on the red giant branch. Contamination from larger red giants 
would be significantly less likely, and detection of rapid rotation on red 
giants would be a significant finding \citep{Tayar15}.

Confusion in the \Kepler{} pixels is also a potential explanation for the
discrepancy. Because the \Kepler{} pixels are relatively large (4\arcsec), a 
bright, hot primary sharing a pixel with a fainter, spottier companion 
contributing the photometric modulation in the
\Kepler{} light curve would explain the observed contradiction between
\vsini{} and rotation period. Despite this setup being unlikely, because for 
stars cooler than the Kraft break, rotation and activity are highly 
correlated, we check for bright companions
within the \Kepler{} pixel using the \Kepler{} field high-resolution UKIRT
J-band images introduced in Section~\ref{sec:astero}. We did not find a 
correlation between that the presence of a bright companion and the magnitude 
of the offset, indicating that confusion is likely not the main source of the 
problem.

One other potential explanation for the mismatch between the photometric
rotation periods and \vsini{} is the presence of SB2s with a special geometry
where the two components are offset enough that their spectral features
overlap, yet not offset enough for the two components to be visually
distinguished. Binarity would also explain the difference in behaviors
between the cool dwarfs and asteroseismic targets; stars with close, but
not identical masses would have similar luminosities on the main
sequence, but significantly different luminosities when the primary
becomes a subgiant, reducing the impact of the secondary on
\vsini{}. To determine the likelihood of this scenario, we check whether
these targets show signs of composite spectra \citep{ElBadry18}. They
report that 21/23 of the confident detection and 17/23 of the marginal detections
in the overlap sample shows signs of composite spectra based on their
data-driven classification. While this is a suggestive result, we note
that the high-resolution Pleiades sample from
\citet{Stauffer87} was a color-selected sample, and did not show widespread
systematics in \vsini{} either compared to our
lower-resolution spectroscopy, or to comparisons with rotation periods
\citep{Jackson10}. Additionally, the total
number of DLSBs identified in APOGEE spectra is lower than the total
number of rapid rotators, indicating that the number of binaries with a
special geometry eluding detection should be significantly less, and
hence cannot explain the large rapid rotator fraction. We postulate that
rapid rotation may be degenerately classifie with a composite spectrum
in the machine-learning method of \citet{ElBadry18}.

The explanation we put forward is that the periodograms used in large-scale
analyses are biased against rapid rotators. While it's unlikely there there is
a hard-limit intrinsic to the period-finding algorithms \citep{Aigrain15}, it
may be possible that in the case of real, nonidealized data, with blending of
multiple variability sources, the algorithms tend to favor long periods over
short ones. Simple problems with aliasing cannot be behind this discrepancy
because in most cases, discrepancies are much larger than a factor of two.

\section{Conclusion}
\label{sec:conclusions}

We analyzed the APOGEE sample of 408 cool dwarfs in the \Kepler{} field with 
rotation periods in \citet{McQuillan14} using DSEP-derived stellar radii. We 
found a substantial discrepancy between the two measures of rotation
among the high \vsini{} targets, with the spectroscopic rapid
rotator rate being 4--6 times the photometric rapid rotator rate. 

A point-by-point analysis of the 52 rapid rotators in our sample revealed that
rotational periods for the entire sample are universally too long to match the 
\vsini{}. This finding implies severe systematic uncertainties in either the
measurement of \vsini{}, rotation period, or in stellar classification. We
suggest several possible ways to investigate the underlying causes.

If the discrepancy lies with either concealed binarity or subgiant
misclassification, the upcoming \Gaia{} data release would be able to
test whether this is the case. Parallax-derived radii should be able to
easily distinguish any misclassified subgiants in the APOGEE spectra.
Additionally, binaries where the companion is bright enough to
substantially alter the measured \vsini{} would also be bright enough to
show a significant luminosity excess given the temperature of the star.
Therefore, rapid rotator false positives should appear anomalous in
luminosity compared to main-sequence isochrones.

Systematic uncertainties in period determinations will be significantly more
difficult to investigate. One start may be to perform a targeted period search
at short periods for the rapid rotators where the blind search returned a long
period. The presence of a short-period signal would confirm that the
spectroscopic rapid rotators are physical. Another direction would be to use an
algorithm which can successfully identify multiple periods, as has been used
for the K2 cluster campaign \citep{Rebull16,Rebull17}. If multiple signals are
present in a light curve, these methods may be able to pick both as opposed to
favoring one over the other.

Radii from the upcoming \Gaia{} Data Release 2 parallaxes will also enable 
follow-up studies to see which other portions of phase-space are affected by 
this mismatch between rotation periods and \vsini{}. Currently we only know 
that the methods yield consistent results for young clusters and 
asteroseismic targets, and yield inconsistent results for cool field dwarfs. If
this inconsistency continues to hold for all main sequence dwarfs regardless of
temperature, this would have significant consequences for the application of
gyrochronology to the \Kepler{} field, and could potentially reveal new
behavior in stellar activity at old ages \citep{vanSaders16}.

Regardless of the source of the inconsistency, we caution readers about 
interpreting large datasets of rotation in the \Kepler{} field without 
independent confirmation until the source of this inconsistency is found.


\bibliographystyle{aasjournal}
\bibliography{references}

\end{document}




