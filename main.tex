% aastex6.1 will be released as part of TeXLive 2017 around June 1. 
\documentclass[manuscript]{aastex6}

% Make the underscore behave correctly.
\usepackage[T1]{fontenc}
% Import images
\usepackage{graphicx}
% Used for consistent handling of Figures/Sections/Tables
\usepackage[capitalize]{cleveref}
% Format URLs correctly
\usepackage{url}
\usepackage{amsmath}
\usepackage[T1]{fontenc}
\usepackage{aecompl}
\usepackage{color}

% Want to have a separate storage folder for my figures.
\graphicspath{{./fig/}}

% Settings for cleveref to make Figures and Tables more appropriately
% defined.
% I want figure to be abbreviated.
\crefname{figure}{Fig.}{Figs.}

\newcommand{\vsini}{\ensuremath{v \sin i}}
\newcommand{\Kepler}{\mbox{\textit{Kepler}}}
\newcommand{\Gaia}{\mbox{\textit{Gaia}}}
\newcommand{\McQuillan}{\citep{McQuillan14}}
\newcommand{\Teff}{\ensuremath{T_{eff}}}
\newcommand{\logg}{\ensuremath{\log(g)}}
\newcommand{\kms}{\textrm{km~s}\ensuremath{^{-1}}}

\newcommand{\gvs}{\authorcomment1}

\shorttitle{Rotation Contamination in \Kepler{}}
\shortauthors{Simonian et al.}
\begin{document}

\title{Evaluating Rotation Contamination in \Kepler{} using APOGEE}
\author{Gregory V. Simonian}
\affil{Department of Astronomy, The Ohio State University}
\affil{140 West 18th Avenue, Columbus, OH 43210}
\email{simonian@astronomy.ohio-state.edu}

\begin{abstract}
To be determined.
\end{abstract}
\keywords{stars: rotation}

\section{Introduction}

One of the significant results from the \Kepler{} satellite has been the
determination of rotational periods for a large sample of old, field stars
\citep{Basri11,Affer12,Nielsen13,Reinhold13,McQuillan14,Garcia14}.
These rotation rates are expected to be extremely useful to deriving a field
gyrochronology, which can lead to a better characterization of the ages of
\Kepler{} stars, and hence a better understanding of the discovered planets.

The \Kepler{} field was designed to point outside of the plane of the galactic
disk, making the field biased toward old stars. Despite the old age of the
field, there is still a population of rapid rotators at the level of 1\% of the
population which have rotation rates significantly higher than expected from
existing gyrochronological relations. One possible explanation for these rapid
rotators are that they are tidally-synchronized binaries. The synchronized
binaries are useful because they provide insight into the formation of binaries
at scales too small for disk fragmentation to occur. 

While there have been many attempts to evaluate the determination of rotation
periods in stars throughout the \Kepler{} field, they have all been through
comparisons between different periodogram analysis, but not with independent 
measures of rotation, such as \vsini.

One of the valuable tools in performing this analysis is a uniform derivation
of the radii of these targets. The initial attempt at characterizing stars and
providing radii was the original KIC \citep{Brown11}, which fit broad-band
\textit{griz} and an intermediate-band \textit{D51} photometry to
\citet{Castelli04} stellar atmospheres and \citet{Girardi00} isochrones.
High-resolution spectroscopy found that radii from the original KIC were found 
to be systematically underestimated \citep{Everett13}. Additionally, around 5\%
of \Kepler{} targets had KIC parameters which were considered unphysical for an 
old field population \citep{Batalha13}. 

\citet{Huber14}, and the updated methodology in \citep{Mathur17} attempt to
address the shortcomings in the original stellar parameter classification by
re-adjusting the KIC parameters to conform to more current DSEP isochrones 
\citep{Dotter08}, and also used more precise input stellar parameters such as
asteroseismology and spectroscopy. In addition to refitting the original KIC 
stellar parameters, \citet{Huber14} derives stellar parameters for the 7\% of 
observed \Kepler{}  
targets without KIC stellar parameters due to missing photometric measurements. 
\citet{Huber14} classified these stars using 
asteroseismology, 2MASS colors, and \Kepler-INT Sloan photometry
\citep{Greiss12}. Because
the APOGEE-2 targeting program \citep{Zasowski17} was based on \citep{Huber14}
stellar parameters without regard to their presence in the original KIC, and
serendipitously observed many of these missing targets, providing a check on
how successful the \citet{Huber14} methodology worked in the absence of full
photometry, 

In this paper, we look at the rotation distribution of \Kepler{} targets as
seen from APOGEE \vsini{}s. In Section~\ref{sec:sample} we describe how we
selected our sample of cool dwarfs to ensure a reliable comparison between
\vsini{} and period. In Section~\ref{sec:analysis}, we describe the reliability
of the quantities we need in order to compare rotation via \vsini{} and
photometric modulation. We also apply our methodology to the APOKASC
asteroseismic sample. In Section~\ref{sec:results}, we describe our findings
from the point-by-point rotational analysis. Lastly, we summarize and describe
the impact of our findings in Section~\ref{sec:conclusions}.

\section{Sample Selection}
\label{sec:sample}

\begin{figure}
    \gridline{\fig{Bruntt_comp}{0.5\textwidth}{(a)}
    \fig{Pleiades_comp}{0.5\textwidth}{(b)}}
    \gridline{\fig{astero}{0.5\textwidth}{(c)}
    \fig{cool_sample}{0.5\textwidth}{(d)}}
    \caption{\emph{Top Left:} Behavior of fractional vsini difference for 
        \citet{Bruntt12} overlap sample with APOGEE\@. A discontinuity in the
        scatter occurs around \(\vsini=7 \kms\). Not shown is KIC11137075,
        which lies below the figure, but has a \citet{Bruntt12}
        \vsini{} of 1.0 \kms, which is significantly below the APOGEE detection 
        limit. Also not shown are targets run through the APOGEE giant grid, 
        which does not calculate \vsini. All of those targets have
    \citet{Bruntt12} \(\vsini < 5 \kms\). \emph{Top Right:} Behavior of
fractional vsini difference for Pleiades cool dwarfs \citep{Stauffer87} overlap
sample with APOGEE\@. In this case, a visible discontinuity occurs around
\(\vsini=15\kms\). Not shown is 2MASS J03475973+2443528 (HII 1653), which is
likely a SB2. \gvs{Uncertainty past 15 \kms remains at about 10\%?}. Red points
are upper limits in \citet{Stauffer87}, which provide an upper limit to the
fractional uncertainty. \gvs{Consider removing these points and just note that
they're all consistent} \emph{Bottom Left:} Stellar properties of asteroseismic
sample. The underlying APOGEE sample consists of targets with the
\texttt{APOGEE2\_SEISMO\_DWARF} flag (see \citet{Zasowski17} for specific
targeting criteria). Asteroseismically-derived \logg{}s are marked as green
stars, while spectroscopically-derived \logg{}s are marked as violet circles.
\emph{Bottom right:} Stellar properties of the cool dwarfs. The dwarf and
subgiants are shown in red and green, respectively. The rapid rotators are
denoted with cyan stars.\label{fig:sample}}
\end{figure}

We select our sample to minimize ambiguity in the stellar radius when
comparing rotation periods to \vsini. In general, large uncertainties in
determining stellar parameters propogate to the radius estimates, especially in 
the hot star regime, where a star of a given \Teff{} and \logg{} can have 
multiple different radii due to overlap with higher-mass subgiants.

In order to sidestep detailed modeling of the complex radius uncertainties 
for all \Kepler{} targets, which will soon be simplified greatly with upcoming
\Gaia{} parallaxes, we choose instead to study measures of
stellar rotation in the cool (\(\Teff < 5500\)) K dwarf regime, where the long
timescale of age evolution leads to a clean separation between dwarf and
subgiants. 

For our base sample, we choose objects observed as part of the APOGEE cool
dwarf ancillary targeting
program\footnote{\url{http://www.sdss.org/dr14/algorithms/ancillary/apogee/kepler_dwarfs/}}.
This program targeted stars observed by \Kepler{} with SDSS-corrected
\(\Teff < 5500\) K \citep{Pinsonneault12}, original KIC \(\logg > 4.0\)
\citep{Brown11} and \(7 < H < 11\). While many of these objects had the
\textit{APOGEE\_KEPLER\_COOLDWARF} targeting flag enabled, much of the rest of
the sample was observed in APOGEE-2 under the \textit{APOGEE2\_SEISMO\_DWARF}
flag. We applied the original selection criteria to the new sample in order get
the new data. As of APOGEE DR14, the program has been \gvs{Percentage} 
completed with \gvs{Check number} objects having been observed. 

Stellar classification with the original KIC photometry was very rough. There
were many cases of contamination in the dwarf sample by subgiants and giants
\citep{Mann12,Gaidos13}. Because the APOGEE Cool Dwarf sample was selected
based on photometry from the original KIC, we'd expect contamination in the
sample from subgiant and giants. APOGEE stellar parameters, on the other hand,
despite being uncalibrated for dwarfs and suffering from systematic
uncertainties, are still able to distinguish between subgiants and dwarfs. As
seen in the bottom right panel of  \cref{fig:sample}, there is a relatively clean 
separation between the dwarf and subgiant branches. We also use asteroseismic 
dwarfs, shown in the top left panel of \cref{fig:sample}, to probe the dwarf-subgiant separation 
with APOGEE stellar parameters. As will be shown in Section~\ref{sec:astero}, we see no true
subgiant masquarading as spectroscopic dwarfs. \gvs{Check Serenelli to make
sure there aren't selection effects explaining this}

A small fraction of the sample has the \textit{STAR\_BAD} quality flag enabled. 
A visual inspection over a sample of these targets show that they are largely
the result of poor subtraction or normalization of the spectrum in the
pipeline, or due to poor model fits on the cool (\(\Teff < 4250\) K) end. We
also visually inspect spectral fits for stars labeled as \textit{STAR\_WARN}.
We generally find that for cases where this flag is enabled, the model fits look 
reasonably similar to the target spectra. For this reason, we retain targets
with the \textit{STAR\_WARN} flag enabled.

Because we're interested in a point-by-point analysis of rotation according to
\vsini{} and rotation periods, we take the subset of \gvs{around 550} objects
observed in APOGEE which also have period detections in \citet{McQuillan14}.
Since both samples were largely selected using stellar parameters in the
original KIC, we expect the overlap between them to be substantial.



\section{Analysis}
\label{sec:analysis}

In order to consistently compare different measures of rotation, we require 
\vsini{}, rotational periods, and stellar radius to function as a 
proportionality constant
between the two. We perform validation to make sure that the quantities adopted
for these three measures are reasonable.


\subsection{\vsini validation}

\vsini{} measurements were only added to the ASPCAP pipeline in DR13, hence
they have not been extensively vetted and verified. We undertake 
verification for our purposes by comparing \vsini{} measured in APOGEE to \vsini{} values for
targets studied with high-resolution optical spectroscopy. 

One large collection of \vsini{}s come from efforts to study the
\Kepler{} asteroseismic targets \citep{Bruntt12}. \citet{Bruntt12} analyzed 
spectra of 93 solar-type asteroseismic targets using two R=80,000 spectrographs 
on the 3.6-meter Canada--France--Hawaii Telescope and 2-m Bernard Lyot
Telescope. While uncertainties are not quoted in \citet{Bruntt12},
previous analyses have determined \vsini{} uncertainties of about 0.6
\kms \citep{Bruntt10a,Bruntt10b}.

A comparison between the APOGEE and \citet{Bruntt12} targets is shown in
\cref{fig:sample}. Not shown in the plot is one significantly discrepant point, 
KIC 11137075, which has a \citet{Bruntt12}
\(\vsini = 1.0 \kms\) and ASPCAP \(\vsini = 3.3 \kms\) consistent with being a
nondetection in APOGEE. Also not shown are six
objects which the ASPCAP pipeline classified as giants, and skipped calculating
\vsini{}. The \citet{Bruntt12} \vsini{} is less than 5
\kms{} for all of these objects, again making them consistent with nondetections.
The rest of the sample shows a discontinuity in the scatter at \citet{Bruntt12} 
\(\vsini=7\kms\). We interpret this discontinuity as setting the detection limit 
of APOGEE, where stellar rotation is actually detected in the high-\vsini{} 
regime.  The scatter in the fractional difference is consistent with 0.1 above that
limit, so we adopt 10\% as the uncertainty in the APOGEE \vsini{} values.

Because the asteroseismic targets tend to be massive and highly populated by
subgiants, there is little overlap between our cool dwarfs and the 
\citet{Bruntt12} sample. In order to validate \vsini{}s for dwarfs with more
similar stellar parameters, we compare
APOGEE \vsini{} for targets observed by \citet{Stauffer87} in the
Pleiades. Because Pleiades targets are significantly younger than in the field,
they are expected to be rotating much more rapidly. This ends up being
advantageous because their wide
distribution of \vsini{}s allows for a wide range of \vsini{}s to be validated. 
Unlike the asteroseismic targets, the Pleiades targets seem to experience a
discontinuity in behavior at \(\vsini = 15 \kms\). \gvs{Check that scatter above
is consistent with 10\%}

For the sake of ensuring we have a reliable sample of \vsini{}, we elect to set
\(15 \kms\) as our detection limit.

\subsection{Period Validation}

The validation of rotational periods in the \Kepler{} field has been a
well-studied problem. There have been multiple different approaches to
measuring rotation periods \citep{Reinhold13,Nielsen13,McQuillan14,Garcia14}.
Based on the analysis with \citet{McQuillan14}, the overlap between their
and previous samples has period agreement within \(\sim\)95\%, with visual 
inspection of the light curve indicating that the disagreement was caused by
aliases detected by previous work. 

A comprehensive, blinded comparison of periodograms analyzed the performance of
several different algorithms \citet{Aigrain15} on injected stellar rotation
signals. The Autocorrelation Function used in \citet{McQuillan14} was found to
have an extremely high detection fraction while still maintaining around 70
percent reliability in detecting the injected signal. Given the intensive study
of different photometric period-detection routines, we refer thd reader to
\citep{Aigrain15} for a detailed discussion about the performance of
periodograms.

\subsection{Comparison for Asteroseismic Targets}
\label{sec:astero}


One of the most well-characterized subsets of the \Kepler{} field is the
APOKASC asteroseismic sample. The APOKASC asteroseismic dwarf and subgiant 
sample consists of 426 targets which presented solar-type oscillations in 
\Kepler{} short-cadence data \citep{Chaplin11}, and were observed in APOGEE DR13
\citep{Majewski17}. Using asteroseismic scaling relations, the radii
for these objects can be determined to an accuracy of around 3\%
\citep{Serenelli17}. It should be noted that subgiants are overrepresented in
the asteroseismic population because the amplitude of solar-type oscillations 
increases as stars evolve. The locations of the asteroseismic targets in an HR
diagram is shown in \cref{fig:sample}.

These are generally higher-mass, evolved dwarfs
and subgiants which show asteroseismic oscillations in their \Kepler{} light
curves. The oscillation frequencies, as well as the spectroscopic
characterization allows for extremely accurate derivations of stellar masses
and radii.

From the 426 asteroseismic dwarfs from the APOKASC v4.2.4 catalog, we removed 
three objects with the STAR\_BAD flag. We include 51 objects with 
the STAR\_WARN and 46 objects with the VSINI\_WARN flags after visual 
inspection because we observed that their modeled stellar parameters tended to
yield visually reliable fits to the spectra.  All targets with \(\vsini \ge
10\) \kms{} were visually
inspected for double-lined spectra. Four were excluded due to the obvious
presence of double-lines. 91 of the remaining stars have period detections in 
\citet{McQuillan14}, 240 were period nondetections, and 88 did not fall under 
the selection criteria in \citet{McQuillan14}. 

\begin{figure}
    \plotone{astero_rot}
    \caption{\emph{Left:} Measured \vsini{} for asteroseismic sample plotted
        against inferred \vsini{} from the period and radius. The solid line 
        denotes the one-to-one relationship. Due to the stellar inclination, most
        of the sample should fall beneath the line. \emph{Middle:} Projected
        radius inferred from \vsini{} and period plotted against the inferred
        radius from isochrones. Inclination should cause most points to lie
        below the line. \emph{Right:} Inferred period from \vsini{} and radius
        plotted against measured period. Inclination should place most points
    above the line.\label{fig:astero_rot}}
\end{figure}

Using the asteroseismic radius \citep{Serenelli17}, we compare the adopted
\vsini{}, rotational period, and radius against inferred values derived from
the other two quantities. The resulting comparison is shown in
\cref{fig:astero_rot}. For these cases, almost all of the asteroseismic 
targets have physical solutions with \(\sin i \le 1\). Only two objects have
unphysical inclinations.

Of these five objects, three (KIC 8677016, 8952800, 10518551) have
\vsini{} values close to the detection threshold. Because the errors of 
marginal \vsini{} detections are complex, 
it may be possible that these are outliers. KIC 3223000 has a
\vsini{} which is slightly above the predicted equatorial velocity. It
is a 1.7-\(\sigma\) difference based on the \vsini{} uncertainty, which
is consistent with being a statistical fluctuation. Additionally, it can
be explained by 4.9 day measured rotation period actually being a 
half-period detection alias, implying that the real equatorial velocity to be twice 
as large. 

The final outlier, KIC 9908400, is so significantly off of the relation
that it can not be explained by either uncertainties or aliasing. However,
it may be adequately explained by contamination of the \Kepler{} PSF\@.
KIC 9908400 has a \Teff{} which places it above the Kraft break. If the 
photometric variation originated in a cooler star within 
the Kepler PSF, the mismatch between the period and \vsini{} would
be explained.

For the rest of the objects, a more comprehensive upcoming analysis 
(Simonian et al\. in prep) indicates that not only are they physically 
plausible, but their distributions are consistent with each other given the 
uncertainties.

\subsection{Stellar Radii Validation}
\label{sec:radii}

We take advantage of the cool star sample for their property that age evolution
for cool dwarfs still on the main sequence occurs slowly, constraining stellar 
radii. 

We use isochrones from the Dartmouth Stellar Evolution Program (DSEP) to
estimate the impact of age ignorance on the radius uncertainties. For solar
metallicity models, the uncertainty due to age is about 8.5\% at the hot end of
our sample. 

We make our error bars for the radius by calculating the stellar radius for
isochrones at the given object's temperature and metallicity at 1 Gyr and 10
Gyr. Because the ``global'' APOGEE uncertainty for [Fe/H] is around 0.05 dex,
to minimize computational resources, we round [Fe/H] to the nearest 0.05 dex.
\gvs{Not true yet. DSEP's interpolation quirks made it so that I could only get
    metallicities to 0.05. I might have to deal with interpolating over
metallicity on my own.}

Because of the relatively small uncertainties due to stellar evolution, one of 
the chief systematic uncertainties we face is the misclassification of
subgiants as dwarfs. This uncertainty would certainly lead to underestimating
the stellar radii of dwarfs by almost a factor of two. We estimate this
contamination rate by recalling from Section~\ref{sec:astero} that the asteroseismic 
sample is heavily biased toward subgiants. This enables us to trace how
spectroscopic classification affects a population of known subgiants with
well-constrained \logg{}s. 

In the bottom left panel of \cref{fig:astero_rot}, we plot both APOGEE and 
asteroseismic \logg{}s separately to see how often subgiants are misclassified 
by APOGEE spectroscopy. Of the asteroseismic targets, \gvs{Some number} fall 
within our effective temperature range, nearly all of them on the subgiant. Of 
all of those targets, in no case does APOGEE misclassify a subgiant as a dwarf. 
There is one case where APOGEE misclassifies a dwarf as a subgiant, which isn't a
type of contamination which affects our results. As a result of this test, we
argue that this type of contamination is rare on the hot end of our sample,
where the Dwarf/Subgiant branches are nearby, it will be even rarer on the cool
end, where the dwarf and subgiants are more readily distinguishable.

\section{Results}
\label{sec:results}

\begin{figure}
    \plotone{cool_rot}
    \caption{Same as \cref{fig:astero_rot}, but with the cool dwarf
        sample.\label{fig:cool_rot}}
\end{figure}

After selecting dwarfs with detectable rotation \gvs{\(> 15 \kms\)} in APOGEE,
visually inspecting for DLSBs, and cross-matching with \citet{McQuillan14}, we 
end up with 19 targets where we can compare the results of rotation. 
The comparison for these targets is shown in 
\cref{fig:cool_rot}. Unlike what was observed for the
asteroseismic sample, the \vsini{} is too large given the observed period and 
radius for the cool stars.

This offset isn't be the result of inclination effects, but rather is
exacerbated by them because inclination effects would act to decrease the
amount of line broadening.

Confusion in the \Kepler{} pixels is also a potential explanation for the
discrepancy. Because the \Kepler{} pixels are relatively large (4\arcsec), it
may be possible to have a bright, hot primary in the APOGEE fiber, with a
fainter, spottier companion contributing the photometric modulation in the
\Kepler{} light curve. A blend with a bright, spotless rapidly spinning G/K 
primary and fainter spotty secondary is expected to be rare given our 
understanding of stellar activity, where angular momentum loss becomes less
efficient with stellar mass. Nevertheless, we check for bright companions
within the \Kepler{} pixel using the public high-resolution UK Infrared Telescope
J-band imaging of the \Kepler{} field. \gvs{And we do not find evidence of
bright companions to these stars.}

Subgiant contamination would explain results where the measured vsini would be
larger than the inferred equatorial velocity, because the radius is
underestimated. While this may explain some of the objects with
larger-than-expected \vsini{}s, it does not explain all of them. A significant
portion of the sample has \vsini{} values which are factors of several larger
than the predicted ones, a greater magnitude than subgiant contamination can 
explain.

Our most plausible explanation for the mismatch between the photometric
rotation periods and \vsini{} is the presence of SB2s with a special geometry
where the two components are offset enough that their spectral features
overlap, yet not offset enough for the two components to be visually
distinguished. 

We look at the overlap of these targets with \citep{ElBadry18}, which attempted
to use data-driven models to find APOGEE objects exhibiting composite spectra. We 
find that all of the overlap sample shows signs of composite spectra according
to their analysis. This is supported by the finding that only one 
objects shows significant RV variability \(> 15 \kms\), and it is \gvs{one of} 
the only object that has a correspondingly fast period.

% From the sample that overlaps with APOGEE, we find that the sample of
% rapidly-rotating objects with high \vsini{} follows a distribution
% consistent with the rotation periods found by Rafa. One of the ways you
% can see this is by noting that the distribution of periods follows the
% distribution predicted by \vsini{} assuming stellar radii as predicted
% by the DSEP models.

% We also find that the period distribution derived from photometric
% variability seems to follow that predicted by the \Kepler{} eclipsing
% binaries.

% The correspondence between \vsini{} and rotation period indicates that
% we are finding the rapidly rotating populations in the \Kepler{} field
% out to a contamination level of \emph{something}

\section{Conclusion}
\label{sec:conclusions}

We took a sample of 19 cool rapidly rotating dwarfs in the \Kepler{} field with 
both APOGEE \vsini{} measurements and \citet{McQuillan14} rotation periods, and 
performed a point-by-point comparison to determine if the rotation periods and
\vsini{}s were concordant. Use imposed a cut of \(\vsini > 15 \kms\) as the
criterion for rapid rotation, which corresponds to a rotation period between
2-3 days for the range of stellar radii within our sample.

Stellar radii were derived using DSEP isochrones, with uncertainties bounded by
the expected age of the \Kepler{} field. While the uncertainties due to age
evolution on the main sequence are less than 10\%, mischaracterization as
subgiants will cause a significantly larger effect. However, we expect that
APOGEE-derived stellar parameters should be able to sufficiently classify
targets such that subgiant contamination should be low.

We validate our methodology by using the \Kepler{} asteroseismic sample as a
test bed. The asteroseismic sample have radii which are precisely
characterized, and has been generally well-studied. The rotation analysis for
the asteroseismic sample indicates concordance between \vsini{} and photometric
period.

When performing the same analysis on the cool sample, we find that rotational
periods for the entire sample are far too long to match the \vsini{}. The fact
that this is universally seen among all rapid rotators in the \Kepler{} field
is troubling, meaning there are systematic mischaracterizations in rotation or
radius for the \Kepler field.

Confusion of the photometric rotation signal is expected to be unlikely because
higher-mass stars below the Kraft break are thought to experience more angular
momentum loss than lower-mass stars. Hence, the more common contaminant should
be rapid photometric periods with slow \vsini{}s.



\bibliographystyle{aasjournal}
\bibliography{references}

\end{document}




