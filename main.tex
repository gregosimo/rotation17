% aastex6.1 will be released as part of TeXLive 2017 around June 1. 
\documentclass[manuscript]{aastex6}

% Make the underscore behave correctly.
\usepackage[T1]{fontenc}
% Import images
\usepackage{graphicx}
% Used for consistent handling of Figures/Sections/Tables
\usepackage[capitalize]{cleveref}
% Format URLs correctly
\usepackage{url}
\usepackage{amsmath}
\usepackage[T1]{fontenc}
\usepackage{aecompl}
\usepackage{color}

% Want to have a separate storage folder for my figures.
\graphicspath{{./fig/}}

% Settings for cleveref to make Figures and Tables more appropriately
% defined.
% I want figure to be abbreviated.
\crefname{figure}{Fig.}{Figs.}

\newcommand{\vsini}{\ensuremath{v \sin i}}
\newcommand{\Kepler}{\mbox{\textit{Kepler}}}
\newcommand{\Gaia}{\mbox{\textit{Gaia}}}
\newcommand{\McQuillan}{\citep{McQuillan14}}
\newcommand{\Teff}{\ensuremath{T_{eff}}}
\newcommand{\logg}{\ensuremath{\log(g)}}
\newcommand{\kms}{\textrm{~km~s}\ensuremath{^{-1}}}

% Targeting flags
\newcommand{\STARBAD}{\texttt{STAR\_BAD}}
\newcommand{\STARWARN}{\texttt{STAR\_WARN}}
\newcommand{\VSINIWARN}{\texttt{VSINI\_WARN}}
\newcommand{\APOGEESEISMO}{\texttt{APOGEE2\_SEISMO\_DWARF}}

\newcommand{\gvs}{\authorcomment1}

\shorttitle{Comparison of Rotation Measurements in \Kepler{}}
\shortauthors{Simonian et al.}
\begin{document}

\title{Rapid Rotation in the \Kepler{} Field: Not a Single Star
Phenomenon}
\author{Gregory V. Simonian; Marc H. Pinsonneault; Donald M. Terndrup}
\affil{Department of Astronomy, The Ohio State University}
\affil{140 West 18th Avenue, Columbus, OH 43210}
\email{simonian@astronomy.ohio-state.edu}

\begin{abstract}
    The \Kepler{} fields have yielded a large data set of photometric rotation
    periods, which when combined with \Gaia{} parallaxes and APOGEE
    high-resolution spectroscopic data for over 4,000 dwarfs and subgiants, 
    can reveal significant insights into the evolution of rotation in stars, 
    We report on spectroscopic measurements of stellar rotation from APOGEE in an
    easily-characterized sample of over 600 cool dwarfs and subgiants in the
    \Kepler{} field. Our data are in good agreement with literature values for
    asteroseismic and Pleiades targets, with a detection threshold between 7 
    and 10 \kms. We find a rapid rotation fraction among the photometric
    binaries of \(\sim 25\%\). When comparing spectroscopic to photometric 
    rotation rates, we find that the photometric rapid rotator 
    fraction is lower than the spectroscopic fraction by a factor of 4--5, yet
    still greatly enhanced in the photometric binaries. A point-by-point 
    analysis reveals that almost all targets have rotation periods too long 
    for their measured \vsini{}. We find these results suggest that either 
    significant contamination of field \vsini{} measurements by photometric 
    binaries with blended spectral features, or current large-scale 
    photometric surveys may mis-characterize or exclude 
    cool rapid rotators in the field. 
\end{abstract}
\keywords{stars: rotation}

\section{Introduction}

One of the significant results from the \Kepler{} satellite has been the
determination of rotational periods for a large sample of old, field stars
\citep{Basri11,Affer12,Nielsen13,Reinhold13,McQuillan14,Garcia14}.
These rotation rates are expected to be extremely useful to deriving a field
gyrochronology, which can better characterize the ages of
\Kepler{} stars, and hence better understand discovered planets.


Current attempts to synthesize rotation information with known ages of
asteroseismic targets in the \Kepler{} field have consistently found a dearth
of long-period (\(P_{rot} \sim 60\) day) rotators \citep{Angus15,vanSaders16}. 
Forward modeling of the \Kepler{} field using existing rotation-activity
relations and estimates of the \Kepler{} selection function further confirm 
that models significantly under-predict the number of (\(P_{rot} > 30\) day) 
rotators in the \Kepler{} field \citep{vanSaders18}. Proposed solutions for 
relieving this tension have been modifying the activity relation in models to 
drop precipitously at a particular rotation threshold, or modifyling angular 
momentum loss to shut off at \(P_{rot} \sim 30\) days \citep{vanSaders18}.

Rotation has proven to be puzzling on the rapid end as well. The \Kepler{} 
field was chosen to point outside of the plane of the galactic
disk, biasing the field toward older stars.  Despite the old age of the
field, there is still a population of rapid rotators at the level of 10\% 
which have rotation rates significantly higher than expected from
existing gyrochronological relations \citep{McQuillan14}. Possible explanation
for these rapid rotators are a truly young stellar population, a background of
tidally-synchronized binaries, contamination of background pulsators in the
\Kepler{} PSF, or confusion with rapidly-rotating subgiants \citep{vanSaders13}. 

A substantial population of young stars would be a significant change in our
view of the \Kepler{} field. Already, more detailed models of the
\Kepler{} field are indicating that there may be more young stars than
previously thought \citep{vanSaders18}. In addition, a widespread presence of
Lithium may indicate that planet-hosting stars may be more youthful than
expected \citep{Berger18a}. Rotation will provde an independent view into youth
in the \Kepler{} field.

In addition to youth, the discovery of a significant population of
tidally-synchronized binaries would also be an exciting find.
Tidally-synchronized binaries do not form through the typical fragmentation
process, but rather from dynamical interactions on the pre-main sequence
\citep{Bate02}. However, tidally-synchronized binaries have been difficult to
characterize because they are rare \citep{Patience02,Raghavan10}. A substantial
number of new tidally-synchronized systems would be extremely useful for
characterizing the population.

Previous studies have succesfully combined rotational periods and \vsini{} 
to study systematics in the radii of pre-main sequence stars
\citep{Rhode01,Jeffries07}, the inclination distribution in clusters 
\citep{Jackson10}, and the inference of radius inflation
\citep{Jackson09,Jackson16,Jackson18}. All of these studies were done for
clusters, where large numbers of spectra could be observed and analyzed under 
the assumption of uniform distances and composition. The availability of
rotation periods and \vsini{} for field stars will be instrumental in 
illuminating other phenomena such as gyrochronology 
\citep{Barnes07,Mamajek08,Angus15}, radius inflation \citep{Jackson18}, and 
rotation on the subgiant branch \citep{vanSaders13}.

In this paper, we study the rotation distribution of \Kepler{} targets as
seen from APOGEE \vsini{}s and compare to published periods. In 
Section~\ref{sec:data} we describe the various catalogs we synthesize to
successfully determine binarity and rapid rotation.
In Section~\ref{sec:analysis}, we show how we obtain a well-characterized
sample of cool dwarfs with rotation information. In
Section~\ref{sec:vsini_check} we validate the quality of the APOGEE \vsini{}
measurements as well as demonstrate that spectroscopic and photometric rotation
measures agree in the 
APOKASC asteroseismic sample. In Section~\ref{sec:results}, we demonstrate that
rapid rotation is primarily seen in photometric binaries, and that there is
disagreement between photometric and spectroscopic rotation measures in the
cool dwarfs. In Section~\ref{sec:discussion}, we compare the two measures of 
rotation on a point-by-point basis and postulate reasons for disagreement. 
Lastly, we summarize our findings and discuss future efforts in 
Section~\ref{sec:conclusions}.

\section{Catalog Data}
\label{sec:data}

\subsection{APOGEE}

The central dataset for this analysis is from the Apache Point Observatory 
Galactic Evolution Experiment (APOGEE) \citep{Blanton17,Majewski17}. The
APOGEE instrument is a high-resolution (\(R \sim 22,000\)) multi-fiber 
near-infrared spectrograph \citep{Wilson10}, mounted on the SDSS 2.5-meter 
telescope at Apache Point Observatory \citep{Gunn06}. Observations for each
target are automatically reduced and combined as part of the APOGEE data 
pipeline \citep{Nidever15}. Stellar parameters are then determined from the 
combined spectra through the APOGEE Stellar Parameters and Chemical Abundance 
Pipeline (ASPCAP), which outputs \Teff{}, \logg{}, \([M/H]\), \vsini{}, as 
well as 15 additional chemical abundances \citep{GarciaPerez16}.

Under several major observing programs, APOGEE systematically observed over 
5,000 dwarfs and subgiants in the \Kepler{} field. Around half were
observed as part of a program to determine false positives for
asteroseismic
detections\footnote{\url{http://www.sdss.org/dr14/irspec/targets/\#AsteroseismicTargets}}. 
They were part of a magnitude-limited sample
selected according to 6500 K \(> \Teff > 5000\) K and \(\logg > 3.5\)
according to parameters derived by \citet{Huber14} \citep{Zasowski17}
which we call the hot sample. An additional 
1000 stars, which we call the cool dwarf sample, with \(\Teff < 5500\) K and 
\(\logg > 4.0\) as determined by \citet{Pinsonneault12} and \citet{Brown11},
and \(7 \textrm{ mag } < H < 11 \textrm{ mag}\) were also 
observed\footnote{\url{http://www.sdss.org/dr14/algorithms/ancillary/apogee/kepler_dwarfs/}}. 
Eclipsing
binaries\footnote{\url{http://www.sdss.org/dr14/irspec/targets/\#EclipsingBinaries}} 
and transiting planet host 
candidates\footnote{\url{http://www.sdss.org/dr14/irspec/targets/\#KeplerObjectsofInterest}} 
were also observed 
by APOGEE in the \Kepler{} field, but with more complex selection functions, 
so we exclude them from this analysis.

All objects observed by APOGEE have stellar parameters \Teff{},
[Fe/H] and \vsini{} determined by ASPCAP, which fits synthetic stellar atmosphere
models to the observed H-band spectra on a 6-dimensional grid. The
effective temperatures outputted from ASPCAP are then calibrated to the
\citep{GonzalezHernandez09} photometric temperature scale using a
metallicity-dependent correction (Holtzmann et al.\ in prep). After the
calibration, comparisons to observations in open clusters yield a
typical RMS of 100 K around solar-metallicity. While the calibration
relations are significantly more uncertain for \([M/H] < -1\), the
finding that the metallicity of the \Kepler{} field is mostly solar
makes these systematic uncertainties largely irrelevant.

We used the iron-abundance as measured by APOGEE to act as a proxy for
metallicity. Although APOGEE provides a weighted metallicity [M/H], the 
stellar isochrones used in this work use iron abundance as inputs, so we 
treated [Fe/H] as our metallicity measure for this purpose. For typical APOGEE 
dwarfs in our temperature range, the measured uncertainty is about 0.01
dex (Holtzmann et al.\ in prep).

Given the stellar parameters from APOGEE, we used isochrones from the Dartmouth
Stellar Evolution Program (DSEP) \citep{Dotter07,Dotter08} to derive
radii and predict absolute magnitudes for our samples. Individual
isochrones are specified by an age,
metallicity, and alpha abundance. For each individual target, we
selected DSEP models using APOGEE iron abundances, assumed solar alpha 
abundance, and used a fiducial age of 3 Gyr, with 1 and 14 Gyr
isochrones determining the uncertainties. The decision to fix alpha
abundance to solar was done to lower computational complexity, and is
justified \gvs{Plot a histogram to check this}. Once a given isochrone
was specified, we constructed a relation between \Teff{} and either radius
or absolute magnitude. Because these relations are double-valued, and
we're interested in the dwarf population, we split the isochrone into 
dwarf and subgiant branches, and used linear interpolation between the
points on the dwarf branch.

In order to classify the evolutionary state of our sample as well as
binarity, we used absolute K-band magnitude as a proxy for bolometric
luminosity. Bolometric luminosities in Gaia DR2 suffer from significant
uncertainties due to their reliance on bolometric corrections, which
depends sensitively on temperature, \logg{}, metallicity, and extinction. 
K-band absolute magnitudes offer the advantage of being a relatively 
well-behaved proxy for luminosity, because it measures flux far into the 
Rayleigh-Jeans tail. Additionally, the extinction in the K-band
is extemely low.

We use K-band photometry from the 2MASS survey \citep{Skrutskie06}, which has
profile-fit photometry with errors available for nearly the full sample. A 
small fraction of targets are only upper limits because of confusion from 
blending. \gvs{Calculate fraction}

To estimate the absolute K-band magnitude, we make use of distances from
\Gaia{} DR2 \citep{Gaia16,Gaia18}, derived from parallaxes using an
exponentially-decreasing volume prior \citep{Berger18b}. Of the full probability
distribution that is calculated, we make use of the mode as well as the
1-\(\sigma\) upper and lower confidence intervals.

Since the full probability distribution functions for the distances are
not available, we calculate the asymmetric uncertainty in absolute
magnitude by adding in quadrature the K-band 1-\(\sigma\) uncertainty to
the appropriately-scaled asymmetric upper and lower 1-\(\sigma\) confidence 
intervals for the distance separately. Because most of our sample is
either strongly dominated by photometric errors or distance errors, and
our results do not depend sensitively on the behavior of the K-band
absolute magnitude, this treatment does an appropriate job of
approximating the uncertainties.

\subsection{Photometric Periods}

As an independent measure of rotation, we make use of photometric rotation
periods of \Kepler{} targets measured in \citet{McQuillan14}.
\citet{McQuillan14} measured rotation periods by calculating the 
Autocorrelation Function of the light curve. The initial dataset 
consisted of 133,030 targets following cuts in \Teff, \logg, and color to
select dwarfs below the Kraft break, as advocated in 
\citet{Ciardi11} based on original KIC parameters \citep{Brown11}. Of this
initial dataset they detected periodic starspot modulations in 34,030 targets.

\citet{McQuillan14} compared their periods to those of previous work
\citep{Reinhold13,Nielsen13} to understand the uncertainties. The overlap 
samples with \citet{Reinhold13} and \citet{Nielsen13} are 20,009 and 10,381,
respectively. Within these overlaps, the different methods get
consistent periods 65\% and 94\% of the time, respectively, and detect
aliased periods 7\% and 2\% of the time. The large discrepancy with
\citet{Reinhold13} is attributed to their only using data for one
\Kepler{} quarter, in contrast to the 12 quarter required by 
\citet{McQuillan14}.

The completeness of \citet{McQuillan14}, on the other hand, is fairly
good. For targets which had periods reported by \citet{Reinhold13},
\citet{McQuillan14} reported missing 1503 of them, leading to a
detection
discrepancy of 7\%. The discrepancy with \citet{Nielsen13} is 2\%.

A comprehensive, hounds-and-hares exercise analyzed the performance of
several different periodograms on injected stellar rotation signals 
\citep{Aigrain15}. The Autocorrelation Function method used in 
\citet{McQuillan14} was found to have a 100\% detection rate on noisy
data while still maintaining around 70 percent reliability in correctly
identifying an injected signal. In contrast, \citet{Nielsen13} has a
low
detection fraction of 16\%, but 100\% reliability \citep{Aigrain15}.
\citet{Aigrain15} identifies the most prominent failure modes:
detection
of the spot lifetime instead of the rotational period, and detection of
the stellar cycle period for stars with short cycles.

Despite the comprehensiveness of the hounds-and-hares exercise, it has
several limitations. The light curves are idealized with a single
stellar signal, and do not take into account blended signals with
multiple potential periodicities. The injected signals also did not
include highly active stars and stars with persistent spots, which are
expected to be overrepresented among rapid rotators. It was also
difficult to compare the performances of each algorithm specifically in
the rapidly-rotating regime without the underlying data.


\subsection{Validation Datasets}

We use several previously studied datasets to validate the APOGEE data, and
to check the uncertainties inherent to APOGEE data. 

One of the most well-characterized subsets of the \Kepler{} field that we use 
is the APOKASC asteroseismic sample \citep{Pinsonneault14}. The APOKASC 
asteroseismic dwarf and subgiant 
sample consists of 426 targets which presented solar-type oscillations in 
\Kepler{} short-cadence data \citep{Chaplin11}, and were observed in APOGEE DR13
\citep{Majewski17}. Using asteroseismic scaling relations, the radii
for these objects can be determined to an accuracy of around 3\%
\citep{Serenelli17}. Because of their excellent parametrization, we treat the
asteroseismic catalog as a control sample, where stellar properties are
well-known. It should be noted that subgiants are overrepresented in
the asteroseismic population because the amplitude of solar-type oscillations 
increases as stars evolve. 

We use the asteroseismic sample to validate our methodologies, making
sure that they yield expected results when all stellar parameters are
directly measured. In addition to APOGEE observations, the asteroseismic
sample has also partially observed as part of a high-resolution
spectroscopic survey, which measured \vsini{} for a significant number of
\Kepler{} targets overlapping with APOGEE \citep{Bruntt12}. \citet{Bruntt12} analyzed 
spectra of 93 solar-type asteroseismic targets using two R=80,000 spectrographs 
on the 3.6-meter Canada--France--Hawaii Telescope and 2-m Bernard Lyot
Telescope. While uncertainties are not quoted in \citet{Bruntt12},
previous analyses have determined \vsini{} uncertainties of about 0.6
\kms \citep{Bruntt10a,Bruntt10b}.

\gvs{Do I want to include the \citep{Garcia14} sample? I used it to
check the rapid rotator fractions.}

In addition to the asteroseismic targets, which are largely subgiants with
\(\Teff \sim 6000 K\), we also compare to APOGEE \vsini{} measurements of cool 
dwarfs in the Pleiades \citep{Stauffer87}. Because Pleiades dwarfs are 
significantly younger than those in the field,
they rotate much more rapidly. This rapid rotation is advantageous
because allows for validation over a wide range of \vsini{}. However, unlike 
\citet{Bruntt12}, the \citet{Stauffer87} survey 
reported a detection limit of 10 \kms, which is significantly below that in
\citet{Bruntt12}. As a result, the \citet{Stauffer87} sample should
largely verify that APOGEE \vsini{} measurements do not experience severe 
systematics for cool dwarfs.

\section{Data Analysis}
\label{sec:analysis}

\begin{figure*}
    \gridline{\fig{astero}{0.3\textwidth}{(a)}
              \fig{cool_mk_sample}{0.3\textwidth}{(b)}
              \fig{binary_cut}{0.3\textwidth}{(c)}}
    \caption{\emph{Left:} Stellar properties of the full APOKASC 
    asteroseismic dwarf/subgiant control sample. The black points mark
    K-band absolute magnitudes for the full control sample. Green squares mark
    the subsample showing oscillations. 
    \emph{Middle:} The full cool dwarf sample as measured with APOGEE stellar 
    parameters. The stellar classifications of dwarfs, photometric binaries, 
    subgiants, and giants marked as teal, purple, and mustard
    dots. Photometric binaries which could also be considered subgiants are
    marked with a pentagon.  The dotted vertical line indicates the 
    \(\Teff = 5450\) K boundary.
    \emph{Right:} Magnitude excess of targets observed in the cool 
    sample above 3 Gyr DSEP models at matching \Teff{} and 
    [Fe/H]. Photometric binaries are shown in green, single stars in teal, and
    subgiants are in purple. The blue, black, and red lines denote the location of 14 Gyr
    isochrones at [Fe/H] = 0.5, 0.0, -0.5. The solid straight line denotes the 
    cut between single stars on the bottom and photometric binaries
above. The larger cutoff at higher temperature results from the faster age
evolution for hotter stars.\label{fig:sample}}
\end{figure*}

The first step to obtaining a usable sample is to perform a 
quality cut for all targets with APOGEE observations.
Depending on the reduced chi-square of the atmospheric model fit, or
disagreement with photometric temperature relations, the ASPCAP pipeline
may trigger a flag indicating that the ouputted parameters may not be
reliable. We categorized our sample into those with \STARBAD, \STARWARN,
\VSINIWARN{}, and no flag enabled to check the quality of the fits. We
visually inspected the spectra for a representative sample of these
quality classes. 
Visual inspection of spectra for targets with the \STARBAD{} flag showed that they are 
largely the result of poor subtraction or normalization of the spectrum in the
pipeline, or are due to poor model fits on the cool (\(\Teff < 4250\) K) end. 
\gvs{Calculate percentage} in the full sample have the \STARBAD{} quality 
flag enabled, which have been excluded from the rest of the analysis. 
Targets labeled as \STARWARN{} are a significantly larger fraction of the 
sample. We found that the model fits reasonably resembled the target spectra. 
For this reason, we retained targets with the \STARWARN{} flag enabled. The 
same occured for targets with the \VSINIWARN{} flags.

We selected the APOGEE hot dwarf sample simply by using the
\textit{APOGEE2\_APOKASC\_DWARF} targeting flag. The temperatures and 
absolute K-band magnitudes of the hot dwarf sample are shown in the left-hand panel of 
\cref{fig:sample}.  As is apparent that the hot 
(\(\Teff > 5500 K\)) star regime is difficult to characterize due to 
significant confusion between main-sequence age evolution,
photometric binaries, and subgiants crossing the Hertzsprung Gap.
A proper analysis with detailed modeling of the complex radius uncertainties 
due to classification errors will be necessary to interpret the rotation
for these stars. We chose instead to study measures of
stellar rotation in the cool dwarf regime, where
slow stellar evolution allows the radii to be easily characterizable through
stellar models, and the cool subgiants, photometric binaries, and main-sequence
stars are more easily separated.

In order to more easily characterize our sample, we instead used targets
observed in the APOGEE cool
dwarf ancillary targeting program.  Most of these objects were observed in 
the first phase of APOGEE under the
\textit{APOGEE\_KEPLER\_COOLDWARF} targeting flag. The remainder of
the sample was targeted in APOGEE-2 under the \textit{APOGEE2\_SEISMO\_DWARF}
flag. We applied the original selection criteria for the ancillary
program to the new sample in order get
the updated target list. As of the DR14, the program has been \(\sim 85\)\% 
completed with 1028 objects having been observed. The full cool dwarf sample is
shown in the right panel of \cref{fig:sample}.

\gvs{Need to get classification straightened out}
We use the classification in 
in \citet{Berger18b} to separate objects into dwarfs, subgiants, and giants.
Informed by solar models of solar metallicity, \citet{Berger18b} uses cuts in
stellar radius at the terminal-age main sequence, and the base of the red giant
branch. \gvs{Maybe do a custom subgiant classification too?}

\gvs{How to distinguish multiple-star systems and subgiants}
We distinguish between single and multiple-star systems by deriving K-band
luminosity excesses compared to DSEP models. Photometric binaries clearly stand
out at 0.75 mag above the main sequence. We attempt to separate multiple-star
systems by interpolating DSEP models between \Teff{} and
\(M_K\). For this calculation, we choose DSEP models with 3 Gyr as a 
reference age. The range in absolute K-band magnitude achievable by age
variation start around 0.5 mag near \(\Teff \sim
5500\) K, and drops down to 0.1 mag near \(\Teff \sim 4000\) K, as illustrated
in \cref{fig:binarity}. We use the temperature cut indicated in
\cref{fig:binarity} to distinguish photometric single systems from
multiple-star systems.

\gvs{If I calculate the maximum age given metallicity/temperature, this
will be unimportant}
By exploring the evolution of stellar parameters between 1--14 Gyr, we found
that stars hotter than \(\Teff = 5450\) K were significantly more difficult to 
characterize due to metal-rich stars having reached the subgiant branch by 14 
Gyr. Because only a 
small fraction of our sample has \(\Teff > 5450\) K, we removed targets hotter
than that cut. The effect of the cut on the sample is shown in the 
right-hand panel of \cref{fig:sample}.

To obtain a reliable sample of \vsini, we also inspected the spectra for every target with 
\(\vsini > 7 \kms\) for the presence of double lines. Most targets showed 
obvious cases of ASPCAP 
fitting a single broad line over two separate narrow features, other cases 
only revealed the second component in individual visit spectra. The targets 
showing double-lined spectra were excluded from the analysis because of
the unreliability of the catalog \vsini{}. Objects classified as having
double-lined spectra are shown in \cref{fig:sample}. More sophisticated future
analyses may be able to disentangle \vsini{} measurements from each component 
\citep{ElBadry18}. There are
also cases where the observed spectra show high RV variability
without a visible second component. We include these cases in our sample
because the measured \vsini{} shouldn't be impacted.

To control for differences between the photometric and spectroscopic sample,
we performed the bulk of our analysis on the subset of 408 objects
observed in APOGEE which also have period detections in \citet{McQuillan14}.
An additional 205 did not have significant periods in \citet{McQuillan14}, and
will be analyzed separately. The remaining 44 targets were observed in APOGEE,
but excluded by the original sample selection in \citet{McQuillan14}. 

The selection of the asteroseismic targets mirrors that of the cool dwarf
sample. From the 426 asteroseismic dwarfs from the APOKASC v4.2.4 catalog, we 
removed three objects with the STAR\_BAD flag, while including 51 objects with 
the \STARWARN{} and 46 objects with the \VSINIWARN{} flags after visual 
inspection. All targets with \(\vsini \ge 7\) \kms{} were visually
inspected for double-lined spectra leading to four being excluded. 91 of 
the remaining stars have period detections in \citet{McQuillan14}, 240 were 
period nondetections, and 88 did not fall under the selection criteria in 
\citet{McQuillan14}. 

\section{\vsini{} validation}
\label{sec:vsini_check}

\begin{figure*}
  \gridline{\fig{Bruntt_comp}{0.3\textwidth}{(a)}
  \fig{Pleiades_comp}{0.3\textwidth}{(b)} 
  \fig{astero_rot}{0.3\textwidth}{(c)}}
    \caption{\emph{Left:} \vsini{} comparison between the \citet{Bruntt12}
        overlap sample with APOGEE\@. A discontinuity in the scatter occurs
        around \(\vsini = 7 \kms\), indicated by the dotted line. The dashed
    line shows the best-fit relation between the two. Not shown are targets 
    run through the APOGEE giant grid. \emph{Middle:} \vsini{} comparison for 
    the Pleiades cool dwarfs \citep{Stauffer87} overlap sample with APOGEE\@. 
    A discontinuity in the scatter occurs around \(\vsini = 12 \kms\), 
    indicated by the dotted line. 2MASS J03475973+2443528 is not shown
    because \citet{Stauffer87} flagged it as a possible SB2. Red points are 
    upper limits in \citet{Stauffer87}.\emph{Right:} Comparison between
    \vsini{} and equatorial \(v_{eq} = \frac{2\pi R}{P}\) for the 
    asteroseismic sample. Dark blue points correspond to confirmed
    \vsini{} detections while light blue points correspond to marginal
    \vsini{} detections. The lines corresponding to \(\sin i = 1\) and
    \(\sin i = 0.5\) are denoted as solid and dashed lines. The hatch
    marks denote the forbidden region where \(\sin i > 1\).\label{fig:comps}}
\end{figure*}


The availability of published \vsini{} is relatively new to the ASPCAP pipeline, only
being included since DR13. As a result, catalog \vsini{}s have not been 
extensively vetted and verified. APOGEE does not provide a reasonable
detection limit, or an estimate on the uncertainty of \vsini{}. We
attempt to characterize the APOGEE \vsini{} measurements ourselves
by comparing to literature \vsini{} for targets studied with high-resolution 
optical spectroscopy. Our method of determining the detection limit and
uncertainty for \vsini{} is similar to that performed in
\citet{Tayar15}.


A comparison between the APOGEE and \citet{Bruntt12} targets is shown in
\cref{fig:comps}. Not shown are six
objects which the ASPCAP pipeline classified as giants, and omitted
\vsini{}. The \citet{Bruntt12} \vsini{} is less than 5 \kms{} for all of 
these objects, making them consistent with 
nondetections.  In order to characterize the APOGEE \vsini{}s, we model the 
two datasets as being linearly related above a specific detection threshold 
with constant fractional uncertainty. The sample shows a 
discontinuity in the scatter at APOGEE \(\vsini=7\kms\), which we 
interpret as the detection limit of APOGEE, above which \vsini{} yields a 
valid stellar rotation measurement. Since the \citet{Bruntt12} errors are
significantly smaller than the expected APOGEE errors, we performed a simple 
linear regression for all targets with APOGEE \vsini{} above the detection 
limit to obtain the relationship. The two datasets have slopes
consistent with one, but with a significant zero-point offset of 9\%,
APOGEE \vsini{} being lower than the \citet{Bruntt12}
\vsini{}. We estimated the measurement error by calculating the mean squared 
error of the residuals, yielding a 13\% uncertainty which we adopted as the 
uncertainty in APOGEE \vsini{}.

We plot the APOGEE \vsini{} agains the \vsini{} measured from
\citet{Stauffer87} in \cref{fig:comps} to look for systematic
uncertainties. At high \vsini{}, the two datasets agree well, even
without the offset seen in the \citet{Bruntt12} sample. Near the the 10
\kms{} \citet{Stauffer87} detection limit, they find a few targets with
much higher \vsini{} than APOGEE detects. While this may be
underestimated noise near the detection limit for \citet{Stauffer87}, we
note that this could also indicate that genuine rapid rotators may be
missed by APOGEE\@.

Our comparison with APOGEE noted a difference in behaviors, with
disagreement between APOGEE and \citet{Stauffer87} for some
targets with \(7 \kms < \vsini < 10 \kms\). In order to ensure that our 
results are not driven by potential nondetections, we split the rapid
rotators into two groups: a \vsini{} detection group, with \(\vsini > 10
\kms\), and a marginal \vsini{} group, with \(7 \kms < \vsini < 10
\kms\). Results which hold in the \vsini{} detection group as well as in
the full sample will be considered robust.

In order to validate our methodology of comparing photometric rotation
data to spectroscopic rotation data, we used the asteroseismic sample to
compared the measured APOGEE \vsini{} to the equatorial velocity 
derived from the rotational periods and the asteroseismic radii. The resulting 
comparison is shown in the right panel of \cref{fig:comps}. We find that
of the confident and marginal detections, two and three, respectively, have 
non-physical solutions with \(\sin i > 1\). 

The marginal detections with unphysical \vsini{} (KIC 8677016, 8952800, 10518551) 
have \vsini{} values close to the detection threshold. Because the errors of 
marginal \vsini{} detections are complicated, it may be possible that these 
are outliers. 

Of the confident detections with unphysical \vsini{}, KIC 3223000 has a
\vsini{} which is slightly above the predicted equatorial velocity. It
is a 1.7-\(\sigma\) difference based on the \vsini{} uncertainty, which
may be consistent as a statistical fluctuation. It can also
be explained by the 4.9 day measured rotation period actually being a 
half-period detection alias, implying a twice as large equatorial velocity. 

The final outlier, KIC 9908400, is so significantly off of the relation
that it can not be explained by either uncertainties or aliasing. However,
it may be adequately explained by contamination of the \Kepler{} PSF\@. 
High-resolution J-band imaging of the \Kepler{} field, provided by the United
Kingdom InfraRed Telescope (UKIRT) reveals a bright, nearby blended companion.
Additionally, KIC 9908400 has a high \Teff{} placing it above the Kraft
break, where convective envelopes are thin;
making spots unlikely. Photometric variation originating from the fainter 
star would explain the mismatch in \vsini{} and rotation period.

\section{Results}
\label{sec:results}

\begin{figure}[htb]
  \centering
  \includegraphics[width=0.8\textwidth]{binarity}
  \caption{The absolute K-band magnitude of rapid rotators compared to a 
  fiducial 3 Gyr isochrone. Rapid rotators are shown as dark blue points
while marginal rotators are shown as light blue points. Objects showing
double-lined spectra are denoted with salmon stars. Objects with
ambiguous stellar classification are shown as pentagons. The range of
possible K-band magnitudes achievable on the main sequence are given as
solid lines. The red, black, and blue lines correspond to [Fe/H]=0.5,
0.0, -0.5, respectively.}
  \label{fig:binarity}
\end{figure}

\begin{figure}[htb]
  \centering
  \includegraphics[width=0.8\textwidth]{mcq_binarity}
  \caption{Cumulative histograms of the K-band magnitude excess with respect to
      a 3 Gyr solar metallicity isochrone for all cool dwarfs with 
      \citet{McQuillan14} periods. The black line is the magnitude excess
      distribution for the full sample while the blue light is just the rapid
  (\(P_{rot} < 3\) day) rotator sample.\label{fig:mcqbinarity}}
\end{figure}


When plotting rapid rotators on an HR diagram, it is clear that the
overwhelming majority of them lie above the main sequence. To make this
result clearer, we plot temperature against the difference between the
observed magnitude, and the magnitude predicted by the corresponding
metallicity-calibrated DSEP model at a fiducial age of 3 Gyr. The
calculated luminosity excess is shown in \cref{fig:binarity}. The range
of luminosity differences achievable by single-star models of varying
ages only reaches a 0.5 mag excess at the hottest part of the range, and
drops significantly with temperature. Most rapid rotators are beyond the
range achievable by single-star models, and must be in multiple-star
systems.

For the subset of our sample that are photometric rapid rotators, they
also show a clear preference for being multiple-star systems systems.
This is further confirmation that rapid rotation is not common in
isolated single stars.

The tendency for rapid rotators to be in multiple star systems is also
seen in the \citet{McQuillan14} cool dwarfs. To select this sample, we
use the same absolute K-band magnitude and temperature cuts as with the
APOGEE sample. Instead of spectroscopic temperatures, we use photometric 
temperature calculated in \citet{Mathur17}. Because the full \citet{McQuillan14} 
sample does not have metallicity information, we cannot glean a sharp
photometric binary sequence as was seen with the APOGEE sample. Nonetheless,
we can note general trends just by removing the overall \Teff dependence of
\(M_K\) by subtracting off  a single 3 Gyr solar-metallicity
isochrone. Plotted in \cref{fig:mcqbinarity} is the cumulative distribution
function of \citet{McQuillan14} targets with \(P < 3\) day compared to the bulk
sample. It is clear from the figure that rapid rotators tend to be found at
larger magnitude excesses than the bulk sample.

\begin{figure}[htb]
    \plotone{detection_fraction}
    \caption{Cumulative distribution function of \vsini{} for the combined
        dwarf sample for spectroscopic and photometric rotation data. The 
        rapid rotator fraction at a given \vsini{} threshold
    would be one minus the value of the cumulative distribution function
    at that \vsini{} value. Solid lines denote the cumulative
    distribution function of APOGEE \vsini{} for the subsample with rotation
    periods. Dashed lines denote  
    the cumulative distribution function of \(v_{eq}\) derived from the
    rotation periods and stellar radii. Dotted lines denote the
    cumulative distribution function of \vsini{} for the subsample without
    rotation periods. Error bars represent 
    1-\(\sigma\) binomial confidence levels.\label{fig:detection_fraction}}
\end{figure}

Given that two rotation measures are available for our sample, we also compare 
the spectroscopic and photometric rapid rotator fractions. In order to
ensure that our comparison results are not driven by the particular
\vsini{} detection limit, we perform our comparison at a variety of
\vsini{} limits to ensure that they are robust.

\gvs{Also need a uniform method of dealing with inclination biases.}
In order to obtain a photometric rapid rotator fraction corresponding to
a \vsini{} detection limit, we transformed the \vsini{} threshold into period 
space. The relationship between rotational period and equatorial velocity is 
given by the relation 
\begin{displaymath}
    v_{eq} = \frac{2 \pi R}{P} 
\end{displaymath}
where \(R\) is the equatorial radius and \(P\) is the rotation period. As we
noted in Section~\ref{sec:data}, there is a relatively clean and narrow
relationship between radius and \Teff{} on the main sequence. We used the 
DSEP-derived radius for each individual object to transform  between
velocity and period coordinates. The fraction of objects with period
short enough to produce detectable \vsini{} is shown as the dashed 
lines in \cref{fig:detection_fraction}. The photometric rapid rotator fraction
ranges from 5-8\% for photometric binaries at 0.5-1\% for single stars.

Due to the \(\sin i\) ambiguity, there isn't a direct correlation
between a \vsini{} cut and the corresponding period cut given a stellar radius.
Many rapid rotators in period-space scatter into being non-detections in
velocity space after convolution with the inclination distribution. However,
this distribution can be statistically characterized for randomly
aligned systems showing photometric variability. Studies of spots in young open
clusters indicate that spots only produce variability when \(\sin i < 1/2\). 
Since our underlying sample was chosen based on
\Teff{} and \logg{}, we only expect a strong bias in inclination caused by spot
geometry.
Therefore, we perform a statistical correction of \(\sqrt{3}/8+\pi/6=0.740\)
to account for photometric rapid rotators scattering to low
\vsini{}. Applying this correction exacerbates the disagreement between
spectroscopic and photometric rapid rotator rates because the inferred range of
rapid rotators increases to 35-67\% for binaries.

The cumulative \vsini{} distribution of our sample is shown 
in \cref{fig:detection_fraction}, with each bin corresponding to the
fraction of targets with APOGEE \vsini{} less than the given
\vsini{}. Error bars denote  binomial 1-\(\sigma\) confidence 
intervals based on selecting the number of rapid rotators from the full 
sample.  As such, the rapid rotator fraction is the complement of that
value. 

As a result, we find that the fraction of
spectroscopic rapid rotators is significantly larger than the fraction
of spectroscopic rapid rotators by a factor of 4-13. 
Accounting for
inclination effects makes the discrepancy even worse to a 4-\(\sigma\)
level. This discrepenacy is highly robust to the choice of
\vsini{} detection limit. Unlike  
the case with the asteroseismic targets, the fraction of spectroscopic rapid
rotators is greater than the fraction of photometric rapid
rotators at a highly-significant level, which is exacerbated afer taking the
inclination correction into account.

We also checked the spectroscopic rapid rotator fraction of the 
\citet{McQuillan14} nondetections for the single and binary sequences. We 
found that the spectroscopic rapid rotator fraction is significantly
lower in the nondetections than the overlap sample for the binaries, but are 
inconsistent with zero for both the binary and single stars. 
A non-zero rapid rotator fraction is surprising because the nondetection
sample is expected to be comprised of slow-rotating inactive stars and 
high-inclination spotted stars, both of which are incompatible with low 
\vsini{} values. 

\begin{figure*}
  \plotone{cool_rot}
  \caption{\emph{Left:} APOGEE \vsini{} plotted against equatorial
  velocity computed from the rotation period and radius for targets with
  detected rapid rotation. Targets which are photometric binaries are plotted
  as large circles while targets which are on the single photometric sequence
  are plotted as small stars. The confirmed \vsini{} detections are shown
  in dark blue while the marginal \vsini{} detections are shown in light
  blue. The solid and dashed lines correspond to values where \(\sin i =
  1, 0.5\), respectively. The hatched area represents the forbidden
  region where \(\sin i > 1\). \emph{Middle:} Symbols are similar to left
  side, except points are projected such that the DSEP-derived radius is
  plotted against the radius inferred from \vsini{} and rotation
  period. \emph{Right:} Symbols are similar to left side, except points are
  projected such that the \citet{McQuillan14} period is plotted against the
  period inferred from the \vsini{} and radius.\label{fig:rot}}
\end{figure*}

To further investigate the discrepancy in rapid rotation rates, we
performed a point-by-point comparison of rotation measures for these 
targets is shown in the bottom panels of
\cref{fig:rot}. The comparison between \vsini{} and equatorial velocity shows
that the discrepancy between \vsini{} and photometric period is not due to slight
but coherent model errors, or a few discrepant points. Unlike with the
asteroseismic sample, all but one of the targets
lie in the forbidden \(\sin i > 1\) region with both the confident and
marginal detections affected. This is indicative of a systematic
bias where periods are consistently (and sometimes drastically) too slow to be
consistent with \vsini{}.

\section{Discussion}
\label{sec:discussion}

Our analysis by combining APOGEE and \Gaia{} data shows that rapid
rotation is not a single-star phenomenon. This is an indication that
consideration of multiplicity is important before applying gyrochronological
relations to the \Kepler{} field. Confirmation of photometric excesses for 
both photometric and spectroscopic rapid rotators indicates that, at least
for some systems, their rapid rotation is physical.

There are two main mechanisms to explain why objects on the photometric
sequence should produce rapid rotation. One is that the system contains
a binary where the two components undergo tidal interactions. These
systems ought to have orbital periods less than 7 day
\citep{Raghavan10}. Since these are short-period systems, they're predicted 
to be strongly biased toward having equal-mass components
\citep{Bate02}. The high rate of photometric excess among the rapid
rotators supports that view. Another explanation for rapid rotation
among objects with photometric excesses is that those systems are
interaction products, and were spun up by mass-transfer. This
explanation is much more likely for the objects above the photometric
binary sequence, which may be thermally unstable.

Our result that rapid rotators tend to be binaries supports the
predictions of efforts attempting to detect stellar multiplicity from
spectra \citep{ElBadry18}. \citet{ElBadry18} analyzed all APOGEE spectra
from DR13 before \Gaia{} parallaxes were available, including a
significant fraction of this sample. They
reported that 18/19 rapid rotators on the photometric sequence and 2/2 on 
the main sequence show signs of composite spectra based on their
data-driven classification. While \citet{ElBadry18} successfully inferred the
high binary fraction of our sample, the fact that they classified both objects
with low photometric excesses as having composite spectra suggests that 
rapid rotation may be degenerately classified with a composite spectrum
in their machine-learning method.

Our other major result is that spectroscopic measures of rapid rotation 
disagree significantly with photometric measures. In particular,
photometric rapid rotation is generally confirmed by spectroscopic rapid
rotation, but not vice versa. This suggests two
possibilities: the \citet{McQuillan14} method misclassifies rapid
rotators as slow, or that the APOGEE \vsini{} have significant false
positives.

The observation that this discrepancy exists for cool dwarfs and not
asteroseismic stars is a significant clue to its origin. It indicates 
that the discrepancy is only significant for certain parts of the HR diagram. 
The main differences between the two are temperatures, with asteroseismic 
targets being significantly warmer than our cool sample, and evolutionary 
state, where asteroseismic targets are subgiants and the cool sample consists 
of dwarfs.

The discrepancy between spectroscopic and 
photometric rapid rotation fractions can be seen in other published work as well. 
\citet{Nielsen13} compared their measurements of \Kepler{} rotation periods to
an amalgamation of \vsini{} from various clusters. They found that the
\vsini{} distribution was consistent with the period distribution for spectral
types F--K, and then the periods dropped below the \vsini{} measurements for
K--M (top panel of their Fig. 2). They ascribe the discrepancy in the cool
stars to a shift in the \Kepler{} age distribution between the hot and cool
stars; however, we consider that the photometric periods may degrade for
binaries.  Additional confirmation occured through the APOGEE M-dwarf rotation
program \citep{Gilhool18}, which surveyed a color- and proper-motion-selected 
sample of nearby M-dwarfs. Unlike in this work, the \vsini{} were measured 
directly through the spectra as opposed to through ASPCAP\@. For the \Teff{} range that
overlaps with \citet{McQuillan14} \(4000 \textrm{ K } > \Teff > 3200\) K, they 
find that 5.8\% of their targets have \(\vsini > 8 \kms\). This is 
significantly greater than our photometric rapid rotator fraction at 8 \kms{} 
of 2.5\%. 

Confusion in the \Kepler{} pixels is one potential explanation for the
discrepancy. Because the \Kepler{} pixels are relatively large (4\arcsec), a 
bright, hot primary star sharing a pixel with a fainter, spottier companion 
contributing the photometric modulation in the
\Kepler{} light curve would explain the observed contradiction between
\vsini{} and rotation period. Despite this setup being unlikely, because for 
stars cooler than the Kraft break, rotation and activity are highly 
correlated, we check for bright companions
within the \Kepler{} pixel using the \Kepler{} field high-resolution UKIRT
J-band images introduced in Section~\ref{sec:vsini_check}. We did not find a 
correlation between that the presence of a bright companion and the magnitude 
of the offset, indicating that confusion is likely not the main source of the 
problem.

The presence of high \vsini{} objects in the period nondetections is
further evidence of a systematic discrepancy. Previous studies have found an 
inverse correlation between photometric variability and Rossby number (defined 
as \(R_o = P / \tau_{cz}\), where \(P\) is the rotation period and 
\(\tau_{cz}\) is the color-dependent convective overturn timescale) 
\citep{Messina01,Hartman09}. Since activity is also inversely
proportional to the Rossby number, this predicts that the stars without
spots should also be rotating slowly, which is not consistent with high
\vsini{}. As mentioned in Section~\ref{sec:vsini_check}, inclination may
also lead to period nondetections, but the low-inclination systems which
don't show starspot modulation would also be expected to show low
\vsini{}. Therefore, we conclude that the existence of 
stars with low Rossby number and low photometric variability may either 
indicate that additional factors determine photometric variability 
aside from the Rossby number, or that \citet{McQuillan14} is not detecting 
rapid rotation in genuine, spotted rapid rotators.

One other potential explanation for the mismatch between the photometric
rotation periods and \vsini{} is the presence of SB2s with a special geometry
where the two components are offset enough that their spectral features
overlap, yet not offset enough for the two components to be visually
distinguished. Binarity would also explain the difference in behaviors
between the cool dwarfs and asteroseismic targets; stars with close, but
not identical masses would have similar luminosities on the main
sequence, but significantly different luminosities when the primary
becomes a subgiant, reducing the impact of the secondary on
\vsini{}. However, the main drawback of this explanation is that
binaries in this configuration should be rarer than observed rate of
rapid rotators. These contaminants will be dominated by systems whose
orbital velocity is in a narrow range of near half the instrumental
resolution, which should be less than 1\% of the sample. However, we are
not able to discount this explanation for the large number of
spectroscopic rapid rotators.

Lastly, the explanation we propose for the behavior of rapid rotatiors is that 
the large-scale period-finding algorithms are yielding
spurious results on short-period tidally-coupled active systems. The
reliability of these algorithms have not been tested for systems that
have a superposition of spot-modulation signals at equal or near-equal
periods with a phase offset. 

While we have two potential explanations for the discrepancy between the
photometric and spectroscopic rapid rotator rate, our currently
available data is unable to distinguish between the two. The most
effective way to break the degeneracy will be to follow-up the high
\vsini{} objects with spectroscopy at higher resolution than APOGEE. If
the line broadening is truly due to blending, the the two components
should easily separate out. If not, then we can conclude that the
spectroscopic rapid rotation is real, and the line is intrisically
broadened.

\section{Conclusion}
\label{sec:conclusions}

We analyzed the APOGEE sample of 368 cool dwarfs in the \Kepler{} field with 
rotation periods in \citet{McQuillan14} using DSEP-derived stellar radii. One
of the first results we found was that 86\% of the rapid rotators are
photometric binaries, indicating that rapid rotation is largely the product of
binary evolution. For the binaries in particular, we
found a substantial discrepancy between the two measures of rotation
among the high \vsini{} targets, with the spectroscopic rapid
rotator rate being 4--13 times the photometric rapid rotator rate. 

A point-by-point analysis of the 49 rapid rotators in our sample revealed that
rotational periods for the entire sample are universally too long to match the 
\vsini{}. This finding implies severe systematic uncertainties in either the
measurement of \vsini{} or rotation period. We suggest several possible ways 
to investigate the underlying causes.

The best way to determine how much of the discrepancy is due to blending of
spectral line is to follow-up with high-resolution spectroscopy beyond the
resolution of APOGEE\@. Whether or not the broadening is found to be real, or to
be composed of multiple components will help determine the true rapid
rotator rate in the \Kepler{} field.

Systematic uncertainties in period determinations will be significantly more
difficult to investigate. One start may be to perform a targeted period search
at short periods for the rapid rotators where the blind search returned a long
period. The presence of a short-period signal would confirm that the
spectroscopic rapid rotators are physical. Another direction would be to use an
algorithm which can successfully identify multiple periods, as has been used
for the K2 cluster campaign \citep{Rebull16,Rebull17}. If multiple signals are
present in a light curve, these methods may be able to pick both as opposed to
favoring one over the other.

Regardless of the source of the inconsistency, we caution readers about 
interpreting large datasets of rotation in the \Kepler{} field without 
independent confirmation until the source of this inconsistency is found.


\bibliographystyle{aasjournal}
\bibliography{references}

\end{document}




