% aastex6.1 will be released as part of TeXLive 2017 around June 1. 
\documentclass[manuscript]{aastex6}

% Make the underscore behave correctly.
\usepackage[T1]{fontenc}
% Import images
\usepackage{graphicx}
% Used for consistent handling of Figures/Sections/Tables
\usepackage[capitalize]{cleveref}
% Format URLs correctly
\usepackage{url}
\usepackage{amsmath}
\usepackage[T1]{fontenc}
\usepackage{aecompl}
\usepackage{color}

% Want to have a separate storage folder for my figures.
\graphicspath{{./fig/}}

% Settings for cleveref to make Figures and Tables more appropriately
% defined.
% I want figure to be abbreviated.
\crefname{figure}{Fig.}{Figs.}

\newcommand{\vsini}{\ensuremath{v \sin i}}
\newcommand{\Kepler}{\mbox{\textit{Kepler}}}
\newcommand{\Gaia}{\mbox{\textit{Gaia}}}
\newcommand{\McQuillan}{\citep{McQuillan14}}
\newcommand{\Teff}{\ensuremath{T_{eff}}}
\newcommand{\logg}{\ensuremath{\log(g)}}
\newcommand{\kms}{\textrm{ km~s}\ensuremath{^{-1}}}

% Targeting flags
\newcommand{\STARBAD}{\texttt{STAR\_BAD}}
\newcommand{\STARWARN}{\texttt{STAR\_WARN}}
\newcommand{\VSINIWARN}{\texttt{VSINI\_WARN}}
\newcommand{\APOGEESEISMO}{\texttt{APOGEE2\_SEISMO\_DWARF}}

\newcommand{\gvs}{\authorcomment1}

\shorttitle{Rotation Contamination in \Kepler{}}
\shortauthors{Simonian et al.}
\begin{document}

\title{A Comparison between Spectroscopic and Photometric Rotation in the
\Kepler{} Field}
\author{Gregory V. Simonian; Marc H. Pinsonneault; Donald M. Terndrup}
\affil{Department of Astronomy, The Ohio State University}
\affil{140 West 18th Avenue, Columbus, OH 43210}
\email{simonian@astronomy.ohio-state.edu}

\begin{abstract}
    Photometric studies of the \Kepler{} field indicate a 5\% fraction of rapid
    (\(P_{rot} < 5\) day) rotators among the cool (\(\Teff < 5500\) K) dwarfs, 
    which is inconsistent with gyrochronology of a \(> 1\) Gyr field. We use 
    APOGEE \vsini{} to independently determine the rate of rapid rotation for
    the well-characterized APOKASC asteroseismic sample, and a magnitude-limited 
    sample of cool dwarfs in the \Kepler{} field. We find that the two measures
    of rotation produce consistent results for the asteroseismic sample. For
    the cool dwarf sample, we find a larger spectroscopic rapid rotator
    fraction of 8\%, larger than the photometric rapid rotator fraction and
    inconsistent with inclination effects. In a point-by-point analysis of cool 
    dwarfs with readily-characterizable radii, we find 20 out of 21 
    spectroscopic rapid rotators in the cool dwarf sample have \vsini{} too large 
    to be explained by the rotation period. These results indicate the 
    presence of severe systematics in the measurement of rotation or stellar
    classification for cool dwarfs.
\end{abstract}
\keywords{stars: rotation}

\section{Introduction}

One of the significant results from the \Kepler{} satellite has been the
determination of rotational periods for a large sample of old, field stars
\citep{Basri11,Affer12,Nielsen13,Reinhold13,McQuillan14,Garcia14}.
These rotation rates are expected to be extremely useful to deriving a field
gyrochronology, which can better characterize the ages of
\Kepler{} stars, and hence better understand discovered planets.

The \Kepler{} field was chosen to point outside of the plane of the galactic
disk, biasing the field toward older stars. Despite the old age of the
field, there is still a population of rapid rotators at the level of 1\% of the
population which have rotation rates significantly higher than expected from
existing gyrochronological relations \citep{McQuillan14}. Possible explanation
for these rapid rotators are a truly young stellar population, a background of
tidally-synchronized binaries, contamination of background pulsators in the
\Kepler{} PSF, or confusion with rapidly-rotating subgiants \citep{vanSaders13}. 

While there have been many attempts to evaluate the determination of rotation
periods in \Kepler{} field stars, all have been 
comparisons between different periodograms, but not with independent 
measures of rotation, such as \vsini.  One valuable tool in comparing the two
measures of rotation is a uniform derivation of stellar radii. An initial attempt 
at estimating stellar radii was the original KIC \citep{Brown11}, which fit 
broad-band \textit{griz} and an intermediate-band \textit{D51} filter 
photometry to \citet{Castelli04} stellar atmospheres to derive \Teff{} and
\logg{}, and \citet{Girardi00} isochrones to infer radii based on stellar
models. High-resolution spectroscopy found that radii from the original 
KIC were found to be systematically underestimated \citep{Everett13}. 
Additionally, around 5\% of \Kepler{} targets had original KIC parameters which 
were considered unphysical for an 
old field population \citep{Batalha13}. 

\citet{Huber14}, and the updated methodology in \citet{Mathur17}, attempt to
address the shortcomings in the original stellar parameter classification by
re-adjusting the KIC parameters to conform to more current DSEP isochrones 
\citep{Dotter08}, and also incorporate more precise input stellar parameters 
such as asteroseismology and spectroscopy. The more sophisticated analysis also
fits all stellar parameters simultaneously, which reduces the number of targets
with unlikely or unphysical properties.

In this paper, we study the rotation distribution of \Kepler{} targets as
seen from APOGEE \vsini{}s and compare to published periods. In 
Section~\ref{sec:sample} we describe how we
selected our sample of cool dwarfs to ensure a reliable comparison between
\vsini{} and period. In Section~\ref{sec:analysis}, we describe the reliability
of the quantities we need in order to compare rotation via \vsini{} and
photometric modulation. We also apply our methodology to the APOKASC
asteroseismic sample. In Section~\ref{sec:fraction}, we compare the rapid
rotator fraction from photometric studies to that found using APOGEE spectra.  
In Section~\ref{sec:results}, we compare the two measures of rotation on a
point-by-point basis. Lastly, we summarize and describe
the impact of our findings in Section~\ref{sec:conclusions}.

\section{Sample Selection}
\label{sec:sample}

\begin{figure}
    \gridline{\fig{cool_sample}{0.5\textwidth}{(a)}
        \fig{astero}{0.5\textwidth}{(b)}}
    \gridline{\fig{Bruntt_comp}{0.5\textwidth}{(c)}
        \fig{Pleiades_comp}{0.5\textwidth}{(d)}}
    \caption{\emph{Top Left:} Stellar properties of the cool dwarf sample as
        measured by APOGEE\@. Stars classified as dwarf and subgiants are shown in 
        red and green, respectively. The rapid rotators are denoted with cyan 
        stars.
        \emph{Top Right:} Stellar properties of asteroseismic sample. The 
        underlying APOGEE sample consists of targets with the 
        \APOGEESEISMO{} flag, used to collect statistics on 
        asteroseismic non-detections (see \citet{Zasowski17} for specific 
        targeting criteria). Asteroseismically-derived \logg{}s are marked 
        as green stars, while spectroscopically-derived \logg{}s are marked as 
        violet circles.
        \emph{Bottom Left:} \vsini{} comparison for the \citet{Bruntt12} overlap 
        sample with APOGEE\@. A discontinuity in the
        scatter occurs around APOGEE \(\vsini=7 \kms\). The thick solid line 
        marks the one-to-one line while the thick dashed line is a linear fit in 
        log space to the detections. Thin dashed lines mark 1-\(\sigma\) offsets
        off of the best-fit line of 13\%. Not shown are targets run through the 
        APOGEE giant grid, which does not calculate \vsini. All of those targets 
        have \citet{Bruntt12} \(\vsini < 5 \kms\). \gvs{Errorbar looks too
        small. Maybe errors are correlated.}
        \emph{Bottom Right:} 
        \vsini{} comparison for the Pleiades cool dwarfs \citep{Stauffer87} 
        overlap sample with APOGEE\@. In this case, a visible discontinuity 
        occurs around \(\vsini=12\kms\). Not shown is 2MASS J03475973+2443528 (HII 1653), which 
        is likely a SB2. \gvs{Uncertainty past 15 \kms remains at about 10\%?}. 
        Red points are upper limits in \citet{Stauffer87}.\label{fig:sample}}
\end{figure}

We select our sample to minimize ambiguity in the stellar radius when
comparing rotation periods to \vsini. In general, large uncertainties in
determining stellar parameters propogate to the radius estimates. This is 
especially prominent in the hot star regime, where higher-mass subgiants and 
lower-mass dwarfs have \Teff{} and \logg{} values that are easily confused.

In order to sidestep detailed modeling of the complex radius uncertainties 
for all \Kepler{} targets, which will be simplified greatly with upcoming
\Gaia{} parallaxes, we choose instead to study measures of
stellar rotation in the cool (\(\Teff < 5500\) K) dwarf regime, where
slow stellar evolution maintains a clean separation between dwarf and
subgiants. 

For our base sample, we choose objects observed as part of the APOGEE cool
dwarf ancillary targeting
program\footnote{\url{http://www.sdss.org/dr14/algorithms/ancillary/apogee/kepler_dwarfs/}}.
This program targeted stars observed by \Kepler{} with SDSS-corrected
\(\Teff < 5500\) K \citep{Pinsonneault12}, original KIC \(\logg > 4.0\)
\citep{Brown11}, and 2MASS \(7 < H < 11\). While many of these objects
were observed in the first phase of APOGEE with the
\textit{APOGEE\_KEPLER\_COOLDWARF} targeting flag enabled, much of the rest of
the sample was observed in APOGEE-2 under the \textit{APOGEE2\_SEISMO\_DWARF}
flag. We applied the original selection criteria to the new sample in order get
the updated target list \gvs{Use Jen's original list for this?}. As of the DR14, 
the program has been \(\sim 85\)\% completed with 1028 objects having been 
observed. 

We also selected based on APOGEE stellar parameter fits. We categorized the 
sample into those with \STARBAD, \STARWARN,
\VSINIWARN{}, and no flag enabled to check the quality of the fits. We
visually inspected the spectra for a representative sample of these
quality classes. 
Visual inspection over these targets show that they are largely
the result of poor subtraction or normalization of the spectrum in the
pipeline, or are due to poor model fits on the cool (\(\Teff < 4250\) K) end. 123
targets in the full sample has the \STARBAD{} quality flag enabled. 
However, stars labeled as \STARWARN{} are a significantly larger fraction of the 
sample. We generally find that the model fits look reasonably similar to the 
target spectra. For this reason, we retain targets
with the \STARWARN{} flag enabled. We find that the same occurs for targets with 
the \VSINIWARN{} flags.

Systematic uncertainties in \logg{} permeated the original KIC stellar
classification. Later analyses found many cases of contamination in the dwarf 
sample by subgiants 
and giants \citep{Mann12,Gaidos13}. Because the APOGEE cool dwarf sample was 
selected based on photometry from the original KIC, we expect contamination 
from subgiant and giants. However, APOGEE stellar parameters, 
despite being uncalibrated for dwarfs and suffering from 
systematic uncertainties, are readily able to distinguish between subgiants and 
dwarfs. We illustrate this in the top left panel of \cref{fig:sample}; there is a 
relatively clean separation between the dwarf and subgiant branches. We use the
temperature-dependent \logg{} cut shown in \cref{fig:sample} to select 657
dwarfs. We also use asteroseismic dwarfs, shown in the top right panel of
\cref{fig:sample}, as a well-characterized test sample, to probe the 
dwarf-subgiant separation with APOGEE stellar parameters in
Section~\ref{sec:radii}.

We also inspected the spectra for every target with \(\vsini > 7 \kms\)
for double-lined spectra. Some targets were obvious cases where ASPCAP fit a 
single broad line over two narrow features, other cases only revealed the
second component in individual visit spectra. There are
also cases where the observed spectrum experiences high RV variability
without a visible second component. We include these cases in our sample
because the measured \vsini{} shouldn't be impacted.

When performing the point-by-point analysis of spectroscopic rapid rotators, 
we take the subset of 408 objects
observed in APOGEE which also have period detections in \citet{McQuillan14}.
An additional 205 are period nondetections. To check for biases between the
photometric and spectroscopic sample, we measure the rapid rotator fraction in
both the full and overlap rapid rotation samples.

\subsection{Asteroseismic Targets}

One of the most well-characterized subsets of the \Kepler{} field is the
APOKASC asteroseismic sample. The APOKASC asteroseismic dwarf and subgiant 
sample consists of 426 targets which presented solar-type oscillations in 
\Kepler{} short-cadence data \citep{Chaplin11}, and were observed in APOGEE DR13
\citep{Majewski17}. Using asteroseismic scaling relations, the radii
for these objects can be determined to an accuracy of around 3\%
\citep{Serenelli17}. Because of their excellent parametrization, we treat the
asteroseismic catalog as a control sample, where stellar properties are
well-known. It should be noted that subgiants are overrepresented in
the asteroseismic population because the amplitude of solar-type oscillations 
increases as stars evolve. The locations of the asteroseismic targets in an HR
diagram is shown in the top right panel of \cref{fig:sample}.

From the 426 asteroseismic dwarfs from the APOKASC v4.2.4 catalog, we removed 
three objects with the STAR\_BAD flag. We include 51 objects with 
the \STARWARN{} and 46 objects with the \VSINIWARN{} flags after visual 
inspection because we observed that their modeled stellar parameters tended to
yield visually reliable fits to the spectra.  All targets with \(\vsini \ge
7\) \kms{} were visually
inspected for double-lined spectra. Four were excluded due to the obvious
presence of double-lines. 91 of the remaining stars have period detections in 
\citet{McQuillan14}, 240 were period nondetections, and 88 did not fall under 
the selection criteria in \citet{McQuillan14}. 

\section{Analysis}
\label{sec:analysis}

In order to consistently compare different measures of rotation, we require 
reliable \vsini{}, rotational periods, and stellar radii. We perform validation
exercises to make sure that the quantities adopted for these three measures are 
reasonable.


\subsection{\vsini{} validation}
\label{sec:vsini_check}

\vsini{} measurements are relatively new additions to the ASPCAP pipeline, only
being included since DR13. As a result, they have not been extensively vetted and 
verified. We undertake verification for our purposes by comparing \vsini{} 
measured in APOGEE to \vsini{} values for
targets studied with high-resolution optical spectroscopy. 

One large collection of \vsini{}s comes from an effort to study the
\Kepler{} asteroseismic targets \citep{Bruntt12}. \citet{Bruntt12} analyzed 
spectra of 93 solar-type asteroseismic targets using two R=80,000 spectrographs 
on the 3.6-meter Canada--France--Hawaii Telescope and 2-m Bernard Lyot
Telescope. While uncertainties are not quoted in \citet{Bruntt12},
previous analyses have determined \vsini{} uncertainties of about 0.6
\kms \citep{Bruntt10a,Bruntt10b}.

A comparison between the APOGEE and \citet{Bruntt12} targets is shown in
\cref{fig:sample}. Not shown are six
objects which the ASPCAP pipeline classified as giants, and omitted
\vsini{}. The \citet{Bruntt12} \vsini{} is less than 5
\kms{} for all of these objects, again making them consistent with nondetections.
In order to characterize the APOGEE \vsini{}s, we model the two datasets as
being linearly related above a specific detection threshold with a constant
fractional uncertainty. The sample shows a 
discontinuity in the scatter at APOGEE \(\vsini=6\kms\), which we 
interpret as the detection limit of APOGEE, above which \vsini{} yields a valid 
stellar rotation measurement. Since the \citet{Bruntt12} errors are
significantly smaller than the APOGEE errors, we perform a simple linear
regression for all targets with APOGEE \vsini{} above the detection limit to
obtain the relationship. We get that the two datasets are consistent with each
other but with a significant zero-point offset of 9\%. We estimate the
measurement error by calculating the mean squared error of the residuals, which
is 13\%. We adopt this as the uncertainty in the APOGEE \vsini{} values.

There is little overlap between our cool dwarfs and the 
\citet{Bruntt12} sample because the asteroseismic targets tend to be massive and 
highly populated by subgiants. We also check the APOGEE \vsini{}s against
observations of cool dwarfs in the Pleiades \citep{Stauffer87}. Because
Pleiades dwarfs are significantly younger than those in the field,
they rotate much more rapidly. This ends up being advantageous because their 
wide distribution of \vsini{}s allows for a wide range of \vsini{}s to be 
validated. We plot the overlap sample in the bottom right panel of
\cref{fig:sample}.  Unlike the asteroseismic targets, the Pleiades targets seem to experience a
discontinuity in behavior at \(\vsini = 12 \kms\). \gvs{Include other Pleiades
    targets. Check that scatter above is consistent with 10\%}

The difference in detection limit between the Pleiades and asteroseismic sample
indicates that the ability to measure \vsini{} changes is a function of stellar
parameters. In order to ensure that nondetections do not affect our results, we
perform our analyses with different detection thresholds in order to ensure
that the results are robust.

\subsection{Period Validation}

The validation of rotational periods in the \Kepler{} field has been a
well-studied problem. There have been multiple different approaches to
measuring rotation periods \citep{Reinhold13,Nielsen13,McQuillan14,Garcia14}.
Based on the analysis with \citet{McQuillan14}, their periods agree with
previous analyses for \(\sim\)95\% of the overlap sample, with visual 
inspection of the light curve indicating that much of the disagreement was 
caused by aliases detected by previous works. 

A comprehensive, blinded comparison of periodograms analyzed the performance of
several different algorithms on injected stellar rotation signals 
\citep{Aigrain15}. The Autocorrelation Function method used in \citet{McQuillan14} was 
found to have an extremely high detection fraction while still maintaining 
around 70 percent reliability in detecting the injected signal. Given the 
intensive study of different photometric period-detection routines, we refer 
the reader to \citet{Aigrain15} for a detailed discussion about the 
performance of periodograms.

\subsection{Comparison for Asteroseismic Targets}
\label{sec:astero}

\begin{figure}
    \plotone{astero_rot}
    \caption{\emph{Left:} Measured \vsini{} for asteroseismic sample plotted
        against inferred \vsini{} from the period and radius. The solid line 
        denotes the one-to-one relationship. Due to the stellar inclination, most
        of the sample should fall beneath the line. \emph{Middle:} Projected
        radius inferred from \vsini{} and period plotted against the inferred
        radius from isochrones. Inclination should cause most points to lie
        below the line. \emph{Right:} Inferred period from \vsini{} and radius
        plotted against measured period. Inclination should place most points
    above the line.\label{fig:astero_rot}}
\end{figure}

For the asteroseismic targets, we compare the measured \vsini{}, rotational 
period, and radius against inferred values derived from
the other two quantities. The resulting comparison is shown in
\cref{fig:astero_rot}. We distinguish between confident \vsini{} detections
with \(\vsini > 12 \kms\), and marginal detections with \(7 \kms < \vsini < 12
\kms\). Of the confident detections, two  
have non-physical solutions with \(\sin i \le 1\). Only two objects have
unphysical inclinations.

The marginal detections with unphysical \vsini{} (KIC 8677016, 8952800, 10518551) 
have \vsini{} values close to the detection threshold. Because the errors of 
marginal \vsini{} detections are complex, 
it may be possible that these are outliers. 

Of the confident detections with unphysical \vsini{}, KIC 3223000 has a
\vsini{} which is slightly above the predicted equatorial velocity. It
is a 1.7-\(\sigma\) difference based on the \vsini{} uncertainty, which
may be consistent with being a statistical fluctuation. Additionally, it can
be explained by the 4.9 day measured rotation period actually being a 
half-period detection alias, implying that the real equatorial velocity to be twice 
as large. 

The final outlier, KIC 9908400, is so significantly off of the relation
that it can not be explained by either uncertainties or aliasing. However,
it may be adequately explained by contamination of the \Kepler{} PSF\@. UKIRT
high-resolution J-band imaging reveals a bright, nearby blended companion.
Additionally, KIC 9908400 has a high \Teff{} placing it above the Kraft break;
making spots unlikely. Photometric variation originating from the fainter 
star would explain the mismatch in \vsini{} and rotation period.

For the rest of the objects, a more comprehensive upcoming analysis 
(Simonian et al.\ in prep) indicates that not only are the \vsini{} values 
physically plausible, but their distributions are consistent with the
rotational periods within the given uncertainties.

\subsection{Stellar Radii Validation}
\label{sec:radii}

The main property of cool stars we exploit is that their slow age evolution
constrains stellar radii. We quantify this uncertainty using isochrones from the 
Dartmouth Stellar Evolution Program (DSEP) to
estimate the impact of age ignorance on the radius uncertainties. We define our
uncertainty as the difference in radius between stars of similar
\Teff{} between 1 and 10 Gyr ages. For solar metallicity models, the uncertainty 
in radius due to age is about 8.5\% at the hot end of
our sample, and decreasing sharply with \Teff. 

We make our error bars for the radius by calculating the stellar radius for
isochrones at the given object's temperature and metallicity at 1 Gyr and 10
Gyr. Because the ``global'' APOGEE uncertainty for [Fe/H] is around 0.05 dex,
to minimize computational resources, we round [Fe/H] to the nearest 0.05 dex.
\gvs{Not true yet. DSEP's interpolation quirks made it so that I could only get
    metallicities to 0.05. I might have to deal with interpolating over
metallicity on my own.}

A significantly larger systematic uncertainty than the impact of age is the 
misclassification of
subgiants as dwarfs. This uncertainty would certainly lead to underestimating
the stellar radii of dwarfs by several factors. We estimate this
contamination rate by using the asteroseismic sample, shown in the top right
panel of \cref{fig:sample}, which mainly consists of subgiants. 

In order to obtain a representative sample of contaminating subgiants, we
select only those with \(\Teff < 5500\) K, resulting in a sample of 317
targets. Of these, all but eight have asteroseismic \logg{} on the subgiant
branch. None have APOGEE \logg{}s which scatter onto the dwarf branch. We do
not concern ourselves with the aswteroseismic dwarfs which APOGEE places on the
subgiant branch because those will not act as contaminants in our dwarf sample. 
Assuming that evolutionary-state classification is a random
process following a binomial distribution, we can infer a 1-\(\sigma\) upper
limit of 2.4 subgiant contaminants in the full cool dwarf sample
\citep{Gehrels86}. 

\section{Rapid Rotator Fractions}

One of the broadest checks that we perform is to see whether the spectroscopic
and photometric rotation data agree in the rapidly rotating regime.
Spectroscopic rapid rotators are defined in this exercise as being those being
above the detection threshold of the APOGEE spectrograph. As mentioned in 
Section~\ref{sec:vsini_check}, APOGEE \vsini{} measurements have not been fully
characterized, and the detection limit could reasonably be placed anywhere 
between 7-12 \kms{}. We perform the analysis over multiple detection thresholds
to ensure the robustness of our results.

In order to obtain a corresponding photometric rapid rotator fraction, we need to 
transform the \vsini{} cut into period space. The relationship between rotational 
period and equatorial velocity is given by the relation
\begin{displaymath}
    v_{eq} = \frac{2 \pi R}{P} 
\end{displaymath}
where \(R\) is the equatorial radius and \(P\) is the rotation period. As we
saw in Section~\ref{sec:radii}, there is a relatively clean and narrow
relationship between radius and \Teff{} on the main sequence. When transforming
between velocity and period-space, we use the DSEP-derived radius for each
individual object, as described in Section~\ref{sec:radii}.

Due to the \(\sin i\) ambiguity, there isn't a simple one-to-one correlation
between a \vsini{} cut and the corresponding period cut given a stellar radius.
Many rapid rotators in period-space should scatter into being non-detections in
velocity space after convolution with the inclination distribution. However,
this distribution can be statistically-characterized, and for random
alignments, the the number of rapid rotators in period space should be increased 
by a factor of \(\pi/4=0.785\).\footnote{We note that the presence of starspot 
    variability biases the
    inclination distribution against face-on alignments. Measurements
    in clusters indicate that spots are not seen for inclinations where 
    \(\sin i < 1/2\) \citep{Jackson10}, whose exclusion only changes the 
correction factor to 0.740} When
comparing the photometric to spectroscopic rapid rotator fraction, we reduce
the photometric rapid rotator fraction by 0.785 to account for potential low
-inclination effects.

\begin{figure}[htb]
    \centering
    \includegraphics[width=0.8\textwidth]{detection_fraction}
    \caption{Rapid rotator fraction as a function of \vsini{} detection limit.
    The black line connects spectroscopic rapid rotator fractions for different
    \vsini{} thresholds. Error bars represent 1-\(\sigma\) Poisson confidence
    levels. The red line marks the fraction of targets with \citet{McQuillan14} 
periods short enough to be detectable over the given \vsini{} threshold. The
photometric rapid rotator fractions are all multiplied by 0.785 to account for
inclination effects.\label{fig:detection_fraction}}
\end{figure}

We validate our methodology by comparing the photometric and spectroscopic
rapid rotator fractions in the asteroseismic sample. In order to have the best
overlap sample, we use the \vsini{} from \citet{Bruntt12} and rotation period 
from \citet{Garcia14}, whose sample for studying stellar rotation specifically
targeted asteroseismic dwarfs and subgiants. We get that the rapid rotator
fraction as measured spectroscopically and photometrically agree well to within
1-\(\sigma\) at all \vsini{} thresholds for rapid rotators.

The results of performing the analysis using just the cool dwarf sample is
shown in \cref{fig:detection_fraction}. This time, the \vsini{} measurements
are from APOGEE and the rotation periods are from \citet{McQuillan14}. Error
bars were chosen to be binomial 1-\(\sigma\) confidence intervals based on
selecting the number of rapid rotators from the full sample. Unlike  
the case with the asteroseismic targets, the fraction of spectroscopic rapid
rotators is greater than the fraction of photometric rapid
rotators at a highly-significant level, which is made worse by taking the
inclination correction into account.

We check for biases in the \citet{McQuillan14} sample by calculating the rapid 
rotator fraction of the \citet{McQuillan14} nondetections. Because no periods
are available for these targets, we can only compare the spectroscopic rapid
rotator fractions. We find that the rapid rotator fraction is significantly
lower in the nondetections than the overlap sample. A smaller rapid rotator
fraction is expected for cases of starspot modulation, because highly-inclined
targets are expected to both have lower \vsini{} values as well as show
diminished modulation amplitudes. Another source of \citet{McQuillan14} are
low-activity stars that don't have detectable starspots. Determining the
relative contributions of each 

\gvs{Make deeper claims? This must indicate that there are slow rotators in the
\Kepler{} field, but that doesn't seem like a useful thing to say. Also, most
of the nondetections are for hot stars, not cool stars.}

We also check the rapid rotation sample in the subgiants. We classify subgiants
as those stars observed by the \Kepler{} cool dwarf program with
\logg{} too low to be a dwarf, but also less than 3.5. Again, since there
isn't a reliable mapping from velocity to period space for these objects, we
only report the spectroscopi rapid rotator fraction. The subgiants behave
similarly to the cool dwarfs, with a rapid rotator fraction of around 10\%.

Lastly we calculate the rapid rotator fraction for the warmer dwarfs.

\section{Results}
\label{sec:results}

\begin{figure}
    \plotone{cool_rot}
    \caption{Same as \cref{fig:astero_rot}, but with the cool dwarf
        sample.\label{fig:cool_rot}}
\end{figure}

After selecting dwarfs with detectable rotation in APOGEE,
visually inspecting for DLSBs, and cross-matching with \citet{McQuillan14}, we 
end up with 19 targets where we can compare the results of rotation. 
The comparison for these targets is shown in 
\cref{fig:cool_rot}. Unlike what was observed for the
asteroseismic sample, the \vsini{} is too large given the observed period and 
radius for the cool stars.

This offset isn't be the result of inclination effects, but rather is
exacerbated by them because inclination effects would act to decrease the
amount of line broadening.

Confusion in the \Kepler{} pixels is also a potential explanation for the
discrepancy. Because the \Kepler{} pixels are relatively large (4\arcsec), it
may be possible to have a bright, hot primary in the APOGEE fiber, with a
fainter, spottier companion contributing the photometric modulation in the
\Kepler{} light curve. A blend with a bright, spotless rapidly spinning G/K 
primary and fainter spotty secondary is expected to be rare given our 
understanding of stellar activity, where angular momentum loss becomes less
efficient with stellar mass. Nevertheless, we check for bright companions
within the \Kepler{} pixel using the public high-resolution UK Infrared Telescope
J-band imaging of the \Kepler{} field. We do not find bright companions near
most of the targets, indicating that confusion is likely not the main source of
the problem.

Subgiant contamination would explain results where the measured vsini would be
larger than the inferred equatorial velocity, because the radius is
underestimated. While this may explain some of the objects with
larger-than-expected \vsini{}s, it does not explain all of them. A significant
portion of the sample has \vsini{} values which are factors of several larger
than the predicted ones, a greater magnitude than subgiant contamination can 
explain.

Our most plausible explanation for the mismatch between the photometric
rotation periods and \vsini{} is the presence of SB2s with a special geometry
where the two components are offset enough that their spectral features
overlap, yet not offset enough for the two components to be visually
distinguished. 

We look at the overlap of these targets with \citep{ElBadry18}, which attempted
to use data-driven models to find APOGEE objects exhibiting composite spectra. We 
find that 28/30 of the overlap sample shows signs of composite spectra according
to their analysis. This is supported by the finding that only one 
objects shows significant RV variability \(> 15 \kms\), and it is one of
the only objects that have a correspondingly fast period.

\section{Conclusion}
\label{sec:conclusions}

We took a sample of 21 cool rapidly rotating dwarfs in the \Kepler{} field with 
both APOGEE \vsini{} measurements and \citet{McQuillan14} rotation periods, and 
performed a point-by-point comparison to determine if the rotation periods and
\vsini{}s were concordant. Use imposed a cut of \(\vsini > 15 \kms\) as the
criterion for rapid rotation, which corresponds to a rotation period between
2-3 days for the range of stellar radii within our sample.

Stellar radii were derived using DSEP isochrones, with uncertainties bounded by
the expected age of the \Kepler{} field. While the uncertainties due to age
evolution on the main sequence are less than 10\%, mischaracterization as
subgiants will cause a significantly larger effect. However, we expect that
APOGEE-derived stellar parameters should be able to sufficiently classify
targets such that subgiant contamination should be low.

We validate our methodology by using the \Kepler{} asteroseismic sample as a
test bed. The asteroseismic sample have radii which are precisely
characterized, and has been generally well-studied. The rotation analysis for
the asteroseismic sample indicates concordance between \vsini{} and photometric
period.

When performing the same analysis on the cool sample, we find that rotational
periods for the entire sample are far too long to match the \vsini{}. The fact
that this is universally seen among all rapid rotators in the \Kepler{} field
is troubling, meaning there are systematic mischaracterizations in rotation or
radius for the \Kepler field.

Confusion of the photometric rotation signal is expected to be unlikely because
higher-mass stars below the Kraft break are thought to experience more angular
momentum loss than lower-mass stars. Hence, the more common contaminant should
be rapid photometric periods with slow \vsini{}s.



\bibliographystyle{aasjournal}
\bibliography{references}

\end{document}




