% aastex6.1 will be released as part of TeXLive 2017 around June 1. 
\documentclass[manuscript]{aastex6}

% Make the underscore behave correctly.
\usepackage[T1]{fontenc}
% Import images
\usepackage{graphicx}
% Format URLs correctly
\usepackage{url}
\usepackage{amsmath}
\usepackage[T1]{fontenc}
\usepackage{aecompl}
\usepackage{color}
% Used for consistent handling of Figures/Sections/Tables
\usepackage[capitalize]{cleveref}

% Want to have a separate storage folder for my figures.
\graphicspath{{./fig/}}

% Settings for cleveref to make Figures and Tables more appropriately
% defined.
% I want figure to be abbreviated.
\crefname{figure}{Fig.}{Figs.}

\newcommand{\vsini}{\ensuremath{v \sin i}}
\newcommand{\Kepler}{\mbox{\textit{Kepler}}}
\newcommand{\Gaia}{\mbox{\textit{Gaia}}}
\newcommand{\McQuillan}{\citep{McQuillan14}}
\newcommand{\Teff}{\ensuremath{T_{\rm eff}}}
\newcommand{\logg}{\ensuremath{\log g}}
\newcommand{\kms}{\textrm{~km~s}\ensuremath{^{-1}}}
\newcommand{\MK}{\ensuremath{M_K}}
\newcommand{\feh}{\textrm{[Fe/H]}}

% Targeting flags
\newcommand{\STARBAD}{\texttt{STAR\_BAD}}
\newcommand{\STARWARN}{\texttt{STAR\_WARN}}
\newcommand{\VSINIWARN}{\texttt{VSINI\_WARN}}
\newcommand{\APOGEESEISMO}{\texttt{APOGEE2\_SEISMO\_DWARF}}

\newcommand{\gvs}{\authorcomment1}

\shorttitle{Rapid Rotator Binarity}
\shortauthors{Simonian et al.}
\begin{document}

\title{Rapid Rotation in the \Kepler{} Field: Not a Single Star
Phenomenon}
\author{Gregory V. A. Simonian; Marc H. Pinsonneault; Donald M. Terndrup}
\affil{Department of Astronomy, The Ohio State University \\ 140 West 18th Avenue, Columbus, OH 43210}
\email{simonian@astronomy.ohio-state.edu}

\begin{abstract}
    Tens of thousands of rotation periods have been measured in the
    \Kepler{} fields, including a substantial fraction of rapid rotators. We 
    use \Gaia{} parallaxes to distinguish photometric binaries (PBs) from
    single stars on the unevolved lower main sequence, and compare their
    distribution of rotation properties to those of single stars both with and 
    without APOGEE spectroscopic characterization. We find that 59\% of stars
    with \(1.5 \textrm{ day } < P < 7\) lie 0.3 mag above the main
    sequence, compared with 28\% of the full rotation sample. The fraction of 
    stars in the same period range is \(1.7 \pm 0.1\%\) of the total sample 
    analyzed for rotation periods. Both the fraction of photometric binaries
    and the fraction of  also consistent with the population of non-eclipsing short period 
    binaries inferred from \Kepler{} eclipsing binary data after correcting for 
    inclination, implying \(2.0 \pm 0.1\%\) of the total \Kepler{} sample. This suggests that the rapid rotators 
    are dominated by tidally-synchronized binaries rather than single-stars 
    obeying traditional angular momentum evolution. Combining both populations
    yields that the total fraction of \Kepler{} targets which are binaries with 
    periods between 1.5--7 days is \(2.0 \pm 0.1\%\). We caution against 
    interpreting rapid rotation in the \Kepler{} field as a signature of youth. 
    Following up this new sample of 217 candidate tidally-synchronized binaries 
    will help further understand tidal processes in stars.
\end{abstract}
\keywords{binaries: close, stars: late-type, stars: rotation}

\section{Introduction}

Rotation is a fundamental property of stars; it impacts their lifetimes, can
induce mixing, and can serve as a population diagnostic. Stars are also born 
with a wide range of rotation rates \citep{Attridge92, Herbst00, Henderson12}.
The observed rotation rates are also strikingly different in high and low mass 
stars. \citet{Kraft67} noted a dichotomy in rotation rates: with field stars 
hotter than 6200 K rotating with a wide range of \vsini, and those cooler than 
6200 K, including the Sun, rotating with largely undetectable \vsini. This abrupt transition is thought to
be due to the onset of convective envelopes for lower-mass stars, which 
enables efficient angular momentum loss through magnetized winds 
\citep{Parker58,Weber67}.

The strong dependence of angular momentum loss on rotation \citep{Kawaler88}
leads to the rapid convergence of a wide range of rotation rates to a unique
mass-dependent value; for solar-type stars convergence is nearly complete by
0.5 Gyr \citep{Pinsonneault89}. Rotation is therefore an age indicator for 
stellar populations \citep{Skumanich72}; and correlations between rotation and
age been used to derive empirical relationships for field populations, 
sometimes referred to as gyrochronology \citep{Barnes07, Mamajek08, Meibom09}.  
A full theoretical treatment of gyrochronology requires evolutionary stellar 
models that include a variety of physical effects (see \citet{Gallet13} for an
overview). Models of angular momentum evolution have usually been calibrated 
using rotation periods in star clusters, which have known ages and are easily 
characterized \citep{Krishnamurthi97, Gallet13, Somers17}. The calibrating 
clusters are typically young (\(< 1\) Gyr), with the behavior at old ages anchored by the Sun. 

The \Kepler{} satellite \citep{Borucki10,Koch10}, which observed a single field
for over four years, has revolutionized our ability to measure stellar 
rotation in old field populations. Large-scale analyses have extracted 
rotation periods from the light curves of tens of thousands of field stars
\citep{Nielsen13, Reinhold13, Garcia14, McQuillan14}. These rotation periods, 
along with a field gyrochronology, are expected to provide insights on the age 
distribution of the field, and thus provide more accurate representation of 
transiting exoplanets.

The new observations of old field stars enabled by \Kepler{} have challenged 
our theories of angular momentum evolution on multiple fronts. Existing 
gyrochronological models anchored at the Sun have not been able to explain
the relatively rapid rotation of old field stars with ages inferred from
asteroseismic data \citep{Angus15, VanSaders16}. The \Kepler{} population at 
long periods shows a sharp drop-off which is not predicted by stellar models 
\citep{VanSaders18}.  Nearby K- and M-dwarfs show a bimodality in the period 
distribution which has no clear explanation in stellar physics 
\citep{Davenport18}. There is also a modest, but real, population of rapid 
rotators that would be interpreted as young if they were single main sequence 
stars.

The \Kepler{} field also contains a mixture of stellar populations, not all of 
which have been calibrated in clusters. Subgiants, while being rare in the 
clusters used to calibrate gyrochronology, 
make up 20\% of \Kepler{} targets \citep{Berger18b}. Since the 
subgiant population contains stars expanding off the main sequence from 
above the Kraft break, they obey a different gyrochronology relation than that 
calibrated for cluster dwarfs \citep{vanSaders13}. 

Another population which is not included in traditional angular momentum
evolution models is tidally-synchronized binaries. These binaries have orbital 
periods short enough that tidal interactions force the rotational period to
synchronize to the orbital period. Existing models of tidal theory 
\citep{Zahn77} predict a strong dependence of the synchronization timescale 
on period, leading to rapid transition between
synchronized and unsynchronized systems. Understanding the 
age distribution of the \Kepler{} field at the young end will require 
characterizing the background population of tidally-synchronized binaries.

Tidally-synchronized binaries themselves are in the center of interesting
dynamical phenomena. These systems are thought to be dynamically-formed 
through three-body interactions \citep{Tokovinin06, Fabrycky07}. Once formed, they can experience significant
evolution in orbital period due to angular momentum loss \citep{Andronov06}.
Depending on the rate of angular momentum loss, they can merge to form blue 
stragglers or pre-CV systems.

Tidally-synchronized binaries have been difficult to study because they are 
intrinsically rare and usually require significant spectroscopic resources 
to discover and characterize \citep{Mathieu90, Raghavan10, Geller15}. A review 
of the observational state of tidal interactions is given in \citet{Mazeh08}. 
Most observations efforts to understand stellar tides have focused on 
measuring the ``cut-off period'' for orbital circularization \citep{Mayor84},
which usually requires intense spectroscopic monitoring to derive orbits.

The largest homogeneous study of synchronicity to date measured starspot 
variability in the eclipsing binary sample \citep{Lurie17}, which found a
cutoff in synchronicity at 10 days, in agreement with previous studies. A novel
result found in their sample was the existence of a population of
subsynchronous rotators (around 15\%) with orbital periods between 2--10 days.
Confirmation of this finding may provide unexpected results in the theory of
three-body interactions.

The availability of \Gaia{} DR2 parallaxes \citep{Gaia18,Lindegren18} has
enabled us to better characterize sources of rapid rotation beyond what is
achievable through photometry and spectroscopy. Young single stars would be
observed as rapid rotators within the field main sequence. Subgiants would lie 
in the Hertzprung Gap, on the way to the Red Giant Branch. Lastly, binaries 
should be up to twice as luminous as the main sequence, extending up to 0.75 
mag above the single-star sequence.  While these populations overlap for hot 
stars, the subgiants and binaries separate cleanly on the lower main sequence.

An additional constraint on the populations comes from the substantial sample
of spectroscopically-characterized dwarfs from the Apache Point Observatory
Galactic Evolution Experiment (APOGEE) \citep{Majewski17}. APOGEE contains a 
total of 15,724 objects in the \Kepler{} field with usable stellar parameters,
K-band photometry, and parallaxes.  This sample will be valuable for 
understanding the importance of metallicity to identify populations of 
binaries.

This paper will attempt to compare binarity between the rapid and slow rotators 
in \citet{McQuillan14}. In Section~\ref{sec:data}, we will describe our 
sample, as well as the data used to select binaries. In Section~\ref{sec:analysis}, we demonstrate the regime where 
binaries can be successfully distinguished from single stars with both the 
presence and absence of metallicity information. We'll also characterize the 
uncertainty in our measurements. Section~\ref{sec:results} illustrates our 
results regarding the prominence of binaries among the rapid rotators. Lastly, 
we discuss the implications of our results for the population of rapid rotators 
and lay out future avenues to explore.

\section{Catalog Data}
\label{sec:data}

The base sample of this study consists of \Kepler{} dwarfs that were analyzed
for photometric starspot modulations. All targets in the \Kepler{} field have 
stellar parameters determined at the very least by KIC photometry 
\citep{Brown11}. The total number of objects that were analyzed was 133,030, 
of which 34,030 have period detections.

We also examine a spectroscopically-characterized subsample from APOGEE\@. 
This sample has spectroscopically-determined temperatures and metallicities. 
The overlap between this sample and the number of objects inspected by 
\citet{McQuillan14} for rotational period modulation is 3,023.

To calculate vertical displacements above the main sequence, we use photometry
from 2MASS \citep{Skrutskie06}, parallaxes from \Gaia{} DR2 \citep{Gaia18},
\Teff{} from \citet{Pinsonneault12}, extinction values estimated from the 
\Kepler{} Stellar Parameter Catalog (KSPC) DR25 \citep{Huber14,Mathur17}, and
MIST isochrones \citep{Choi16} (described further in
Section~\ref{sec:analysis}). For the spectroscopic sample, \Teff{} and [Fe/H] 
are taken from APOGEE DR14 \citep{Abolfathi18}. More detail on these catalogs 
is given below.

\subsection{Default Stellar Parameters}

Stellar parameters for the full \Kepler{} sample are available as part of the
\Kepler{} Stellar Parameter Catalog (KSPC) DR25 \citep{Mathur17}. The KSPC
compiles stellar parameters from the literature and simultaneously fits them
using DSEP stellar evolutionary tracks \citep{Dotter08} via the methodology of
\citet{Huber14}. Revised stellar parameters, such as \Teff, as well as inferred
quantities, such as extinction, are output along with asymmetric
1-\(\sigma\) uncertainties. 

The sophisticated machinery of the KSPC imprints artifacts in the \Teff{}
distribution of cool dwarfs by attempting to stich together the temperature
scales of \citet{Pinsonneault12} and \citet{Dressing13}. Instead of using
\Teff{} from the KSPC, we therefore opted to use temperatures from 
\citet{Pinsonneault12} to ensure uniformly analyzed temperatures; this data is
close to the KSPC scale globally \citep{Huber17}. We note that 
\citet{Pinsonneault12} only provides temperatures for colors where the
infrared-flux method has been well-studied, which corresponds to between
4000--7500 K.

For this paper, we also make use of extinctions from the KSPC\@. The
extinctions are inferred from the predicted absolute magnitude given KSPC 
parameters, and a 3D extinction map \citep{Amores05}. The catalog extinction 
is given as \(A_V\), which we convert to other bands using the 
\citet{Cardelli89} extinction law. 

\subsection{Astrometric Data}

The ability to distinguish between different evolutionary states of stars is
enabled by parallaxes derived from the \Gaia{} mission's Data Release 2
\citep{Gaia18}. The \Gaia{} DR2 performed a fully consistent single-star
5-parameter (\(\alpha\), \(\delta\), \(\mu_\alpha\), \(\mu_\delta\), \(\pi\))
solution to 1.3 billion sources over 22 months of observations
\citep{Lindegren18}. For targets which could not be adequately fit with a
single-star 5-parameter solution, a 2-parameter (\(\alpha\), \(\delta\))
solution is performed instead \citep{Michalik15}. The criteria for a
2-parameter fall-back solution include targets with Gaia \(G > 21.0\), having
fewer than 6 visibility periods of observations, and an error ellipse larger
than a magnitude-dependent limit in its largest dimension \citep{Lindegren18}.

% APOGEE Targeted sample APOKASC (APOKASC + Cool Dwarf - faint dwarfs) - 7642
% Not bad APOGEE targeted sample - 7395
% Good APOGEE with good K detection - 7373
% Good APOGEE in Gaia - 6961

We use the cross-matched database of \citet{Berger18b} to match \Kepler{} targets
against \Gaia{} DR2 detections.
\citet{Berger18b} cross-matched targets in the KSPC DR25 \citep{Mathur17}, 
with \Gaia{} DR2 sources using both position and G-band flux. Of the 
successfully cross-matched targets, \citet{Berger18b} excluded objects with 
fractional parallax errors greater than 0.2, \(\Teff < 3000\) K, 
\(\logg < 0.1\), and low quality 2MASS photometry.  Because all of our 
targets have high-quality parallaxes, we use the traditional 
formula for deriving distance modulus from parallax instead of the Bayesian 
method advocated by \citet{Luri18} to reduce computational complexity. We
include the zero-point offset of 0.05 mas \citep{Zinn18}.
% Include when this calculation is actually *DONE*.
%We converted parallax to distance modulus by using an
%exponentially-decreasing volume density prior with a 1.35 kpc length scale
%\citep{BailerJones15,Astraatmadja16}, and correcting for the zero-point offset
%of 0.5 mas \citep{Zinn18}. \gvs{Should we include a table of this?} 

\subsection{Photometric Data}

In order to characterize our stellar sample, we chose to use absolute K-band
magnitude as a proxy for luminosity to minimize the impact of extinction. 
K-band apparent magnitudes were measured by the 2MASS survey 
\citep{Skrutskie06}, which has profile-fit photometry with uncertainties 
available for nearly the full sample.  The 
typical K-band photometric uncertainty for the \citet{McQuillan14} sample is 
0.03 mag. The typical extinction from the KSPC and its uncertainty for cool 
dwarfs is \(A_V = 0.33\), and \(\sigma_{A_V}=0.025\). Applying the 
\citet{Cardelli89} relation that \(A_K/A_V = 0.114\), the effect of extinction 
itself shrinks to the level of photometric errors.  A small fraction of 
targets (1.5\%) of the \citet{McQuillan14} sample have 2MASS photometry 
flagged due to blending. To avoid contaminating the binary sequence with 
unrelated blends, we exclude these targets.

\subsection{Rotation Data}

The rotation periods for our sample come from \citet{McQuillan14}, which
is large and homogenously measured. \citet{McQuillan14} selected their sample 
to have photometric \(\Teff < 6500\) K \citep{Brown11,Dressing13}, and 
implemented the color-color and \logg{} cuts advocated in \citet{Ciardi11}, 
ranging from \(\logg = 3.5\) at 6000 K to \(\logg = 4.0\) at 4250 K. 
In addition to the temperature and gravity cuts, \citet{McQuillan14} also 
excluded known \Kepler{} eclipsing binaries and \Kepler{} Objects of Interest. 

Rotational periods in \citet{McQuillan14} were measured by calculating the 
autocorrelation function and fitting the location of multiple peaks. In lieu 
of visual verification of the autocorrelation function, \citet{McQuillan14}
performed two automated tests to distinguish physical periodicity from
instrumental artifacts or other sources of variability. First, the 
periodicity must be consistent in different segments of the light curve.
Secondly, the height of the first peak must be larger than a temperature- and
period-dependent threshold. In order to reduce contamination from pulsators, 
\citet{McQuillan14} only considered periods between 0.2--70 days.

\begin{figure*}[htb]
    \centering
    \epsscale{1.1}
    \plotone{mcquillan_selection}
    \caption{\emph{Left:} \Teff-\(M_K\) density plot of the sample of
        \citet{McQuillan14} period detections. Color represents the number of
        objects in each bin. A binary sequence is clearly 
        visible above the lower main sequence. Temperatures are from
        \citet{Pinsonneault12}. The bin size is 100 K in
        temperature and 0.02 mag in K-band absolute magnitude. A 
        representative error bar is shown on the bottom right corner, although
        the vertical error bar is too small to be easily visible. 
        \emph{Right:} The variation in the \citet{McQuillan14} period 
        detection fraction across the \Teff-\(M_K\) 
    diagram.}\label{fig:mcquillan_selection}
\end{figure*}

The \citet{McQuillan14} sample is shown in a \Teff-\(M_K\)
diagram in \cref{fig:mcquillan_selection}. One of the new results enabled by 
\Gaia{} is that a binary sequence,
located above the main sequence, is clearly seen. Period detections become
relatively rare in evolved stars. The bulk of the period sample are solar
analogs, reflecting the underlying \Kepler{} sample.

\subsection{Spectroscopic Parameters}

We draw our spectroscopic sample from the Data Release 14 \citep{Abolfathi18}
of the APOGEE survey \citep{Majewski17}. The workhorse behind the survey is the
APOGEE spectrograph, a high-resolution (\(R \sim 22,000\)) multi-fiber
near-infrared spectrograph \citep{Wilson10} mounted on the SDSS 2.5-meter
telescope at Apache Point Observatory \citep{Gunn06}.

Reduction of the APOGEE data takes place in three main stages. First,
individual visit spectra are reduced, including detector calibration, bad pixel
masking, wavelength calibration, sky substraction, and determination of
individual radial velocities through cross-correlation of template spectra.
Second, the individual spectra are combined by correcting for radial velocity
differences between the exposures, either by cross-correlating with each other,
or with template spectra, whichever leads to a smaller scatter
\citep{Holtzman18}. The full process is detailed in \citet{Nidever15}.

The final step is the extraction of stellar parameters and chemical abundances
by the APOGEE Stellar Parameter and Chemical Abundances Pipeline (ASPCAP)
\citep{GarciaPerez16}. ASPCAP measures stellar parameters by performing a
chi-squared minimization \citep{AllendePrieto06} over a 6-dimensional space of
\Teff, \logg, [M/H], [\(\alpha\)/M], \vsini, and microturbulent velocity.

After determination of the stellar parameters, the effective temperatures are 
calibrated to the photometric system of \citet{GonzalezHernandez09} using a 
metallicity-dependent offset. After calibration, the scatter between the 
calibrated and photometric temperatures  were found to be a function of 
\Teff{}, ranging from 130 K at 5500 K down to 85 K at 4000 K
\citep{Holtzman18}.

We inspected the fits to a set of spectra representative of targets flagged 
by the ASPCAP pipeline as having potential quality problems. A small fraction
(2.5\%) of targets have the \STARBAD{} quality flag enabled. Visual inspection
of a representative sample of spectra indicated that the flag was largely the
result of poor subtraction or normalization of the spectrum in the pipeline, or
due to poor model fits on the cool (\(\Teff < 4250 K\)) end. We exclude these
from our analysis. A substantially larger fraction of targets (14\%) have the 
\STARWARN{} quality flag enabled. Visual inspection of a representative sample 
of these spectra indicated that the fits reasonably resembled the underlying 
spectra. We included all targets with the \STARWARN{} quality flag enabled.

\begin{figure*}[htb]
    \centering
    \epsscale{1.1}
    \plotone{apogee_selection}
    \caption{\emph{Left:} \Teff-\(M_K\) diagram for the APOGEE observations of
        \Kepler{} targets. Asteroseismic targets are shown as black dots. The
        dwarf sample is shown as brown dots. The light blue dots indicate 
        eclipsing binary targets and purple dots are \Kepler{} Objects of 
        Interest. A binary sequence is clearly visible on the lower main
        sequence. Temperatures are spectrosopic APOGEE temperatures. A 
        representative error bar for the cool dwarf sample is 
        shown in the bottom-right corner. \emph{Right:} A density plot of the 
        full APOGEE sample. To preserve the dynamic range of the dwarf 
        sequence, the red clump was allowed to saturate. The bin size is 100 K
    horizontally and 0.02 mag vertically.}\label{fig:apogee_selection}
\end{figure*}

\Cref{fig:apogee_selection} shows all \Kepler{} targets with observed
spectroscopic temperatures in the APOGEE DR14, color-coded by observing
program. One of the driving efforts for \Kepler{} field observations has been
the spectroscopic characterization of asteroseismic targets, shown as small
black dots \citep{Zasowski17,Pinsonneault18}; 11,734 targets are part of that
sample. The asteroseismic sample is substantially biased toward giants 
and subgiants due to their high amplitude oscillations. A complementary 
program proposed to observe the remaining portion of the \Kepler{} sample 
cooler than 6500 K and brighter than \(H < 11\), make up an additional 3193
targets, shown as small red dots. An additional 695 targets were part of an 
effort to characterize \Kepler{} planet hosts with \(H < 14\), shown as purple 
diamonds. The remaining 102 targets were follow-up of eclipsing binaries 
\citep{Prsa11,Slawson11}, and are shown as light blue circles.  

\section{Data Analysis}
\label{sec:analysis}

The fundamental classification in this analysis is to distinguish photometric
binaries from the rest of the sample. The crucial quantity to make this
distinction is the 
vertical displacement above the single-star sequence. Vertical 
displacement is a standard measure of binarity in clusters 
\citep{Mermilliod92}, where the single-star sequence can be described by a
single-age stellar population with a single metallicity. Equal-mass binaries
would lie 0.75 mag above the single-star sequence, while equal-mass triples would be
1.25 mag above. Measuring the vertical displacement for a field population is 
more challenging, because the
field samples heterogeneous ages and metallicities. There is a
field turn-off, however, and for sufficiently cool dwarfs there is a
well-defined unevolved main sequence that is an analog of the cluster case. We
therefore search for the temperature domain where age effects are minimized, 
and the natural width of the unevolved main sequence can be confidently
measured to identify field binaries.

The vertical displacement is a difference of two quantities: the measured 
K-band luminosity of a star, and the inferred K-band luminosity of a single 
star with the same temperature and metallicity as the original one at a reference age. The 
former is a well-constrained and easily calculated quantity given the 2MASS 
photometry and the \Gaia{} parallax; the latter requires a single-star model.

We use the MIST \citep{Dotter16,Choi16} isochrones to model the 
single-star lower main sequence. Bolometric corrections translating from
evolutionary tracks to photometric bands are generated from ATLAS12/SYNTHE
stellar atmosphere models \citep{Kurucz70,Kurucz93}. For a full description of the MIST isochrones, we 
refer the reader to the source papers, as well as the MESA instrument papers 
\citep{Paxton11, Paxton13, Paxton15}, which is the stellar model on which the 
isochrones are based. In order to characterize the vertical displacement, we select 1 Gyr isochrones
as a baseline case. As age effects in the domain of interest are by definition
small, the precise choice of baseline age does not significantly impact our 
results (see below).

\begin{figure}[htb]
    \centering
    \plotone{sample_dk}
    \caption{APOGEE temperature and K luminosity excess above a 1 Gyr MIST isochrone 
        matched to the APOGEE metallicity for the APOGEE-\Kepler{} sample. 
        In order to exclude evolved stars while preserving binaries and
        triples, we define the dwarf sample as targets less than 1.3 magnitudes
        more luminous than the metallicity-adjusted MIST isochrone (shown as
        violet points). A representative error bar for the 
        full APOGEE dwarf sample assuming Gaussian uncertainties is shown in 
        the lower right corner.}\label{fig:sample_dk}
\end{figure}


The vertical displacement of the full APOGEE sample above a 1 Gyr isochrone is 
shown in \cref{fig:sample_dk}. For every star, the displacement is calculated 
by subtracting the luminosity of the 1 Gyr isochrone with the APOGEE-measured 
metallicity. F and G stars spread above the isochrone, which is 
expected because these stars evolve substantially over the age of the Milky
Way. With the main sequence detrended a visible binary sequence emerges at 
\(\sim0.75\) mag above the main sequence. In order to exclude red 
giant branch stars, we draw a limit at 1.3 magnitudes above the main sequence. 
This limit cleanly excludes red giant branch stars while providing generous 
inclusion limits for binaries and potential triple systems.

\begin{figure}[htb]
    \centering
    \plotone{ages}
    \caption{Age-imposed behavior for the dwarf sample with APOGEE
        temperatures. The purple line denotes the predicted luminosity excess of 
        a 9 Gyr solar composition isochrone relative to a 1 Gyr baseline at
        fixed metallicity. The dashed line represents 0.75 
        mag, the limit for photometric binaries. The 25th percentile of the 
        dwarfs in 15 temperature bins is shown in brown. The median absolute 
        deviation of the least-luminous 50\% of the bin is shown as the error 
        bar. We draw a temperature threshold for the unevolved lower main 
        sequence at 5250 K. A representative statistical error bar assuming 
        Gaussian uncertainties is shown in the lower right corner.}
    \label{fig:ages}
\end{figure}

For hot stars, the increase in luminosity due to age is degenerate with
increased luminosity due to a binary companion. To reduce the impact of age,
we restrict ourselves to the unevolved lower main sequence, where age
effects are small. We quantify the impact of age by predicting the vertical
displacement of a 9 Gyr solar-composition MIST isochrone with respect to our 
1 Gyr solar-composition MIST isochrone, shown in \cref{fig:ages}. The
predicted displacement between the two isochrones is -0.20 mag at 5500 K, 
-0.11 mag at 5250 K, and -0.08 mag at 5000 K, showing a natural break at 5250
K.

We parametrize the behavior of the main sequence by splitting it into bins
and calculating the 25th percentile of the luminosity excess in each bin. Each 
temperature bin contains a population of both binaries and single stars, with 
binaries having substantially higher vertical displacements. The 25th 
percentile statistic characterizes the main sequence in a way that is robust 
to the population of binaries, and has been used successfully in clusters 
\citep{An06}. Even large departures this assumed binary fraction should 
generally only affect the zero-point, but not the shape of the main sequence.

The detrended main sequence is generally flat at temperatures cooler than 5000 K 
with age effects becoming more important at hotter temperatures. The number of stars
also increases substantially with temperature above 5000 K. As a compromise
between age effects and statistical power, we adopt a threshold \Teff{} of 
5250 K to denote the lower main sequence.

\subsection{Corrections for the Spectroscopic Sample}
\label{sec:speccor}

There are modest, but real, differences between the theoretically predicted
positions of isochrones and the data which impose scatter on the detrended main 
sequence. Residual trends are not surprising because the NIR behavior of MIST 
isochrones have not been comprehensively tested at a wide range of metallicities 
\citep{Choi16}.  In order to achieve a truly detrended main sequence, we include 
both metallicity and temperature-dependent terms.

\begin{figure}[htb]
    \centering
    \epsscale{0.9}
    \plotone{metallicity_correction}
    \caption{\emph{Top:} Residual metallicity trend in the MIST 
        metallicity-corrected displacement. A quadratic trend remains over 
        metallicity. The trend is characterized by fitting the sample to 
        5 bins with equal numbers of points.  The 25th percentile of points 
        denote the y-value of the bin while the mean temperature within each
        bins denotes the x-value. The best-fit relation is \(\Delta
            \MK{} = -1.14\feh{}^2 + 0.448 \feh{} + 0.0411\). Red points mark 
            the bins used for the fit.  The median absolute deviation of the 
            50\% of the sample showing the
        least luminosity excess within each bin is shown as an error bar. Three 
        very low-metallicity points are not shown.  \emph{Bottom:} Residuals 
        after correcting for the quadratic trend.}\label{fig:met_trend}
\end{figure}

We find and correct for a residual quadratic trend in the vertical 
displacement of the main sequence over metallicity, shown in 
\cref{fig:met_trend}.  We fit the trend to 5 equally populated bins. We denote 
the metallicity of the bin as the mean metallicity of all stars in that bin, 
and the vertical displacement as the 25th percentile of all stars in that bin. 
For the purpose of selecting the optimal binning style, we quantified the 
goodness-of-fit using the median absolute deviation of the top 50th percentile 
of the vertical displacements. The correction was mostly robust to the choice 
of binning; only the sparsely populated low-metallicity regime 
(\([Fe/H] < -0.6\)) was sensitive to the details of binning. Because there are 
only three objects in this metallicity range, our conclusions are not 
sensitive to their treatment.

\begin{figure}[htb]
    \centering
    \plotone{spec_teff_correction}
    \caption{\emph{Top}: Vertical displacement above a 1 Gyr,
        metallicity-adjusted MIST isochrone, empirically corrected for
        metallicity trends, for the cool sample with APOGEE temperatures. 
        A slight linear trend in \Teff{} was measured using five bins (shown in
        red). The best-fit relation is \(\Delta \MK{} = -0.000114
        \Teff{} + 0.546\) The median absolute deviation of the 50\% of the 
        sample showing the least luminosity excess within each bin is shown as 
        an error bar. \emph{Bottom:} The residuals after subtracting the 
        linear trend.}\label{fig:apogee_teff_trend}
\end{figure}

After correcting the trend in metallicity, we correct a residual trend in
\Teff{}, shown in \cref{fig:apogee_teff_trend} using a similar procedure. We
take five equally-populated bins, and define the bin temperature and 
displacement using the mean within the bin  and 25th percentile displacement
within the bin.  This trend is slight and can be subtracted using a linear 
fit. After performing these two corrections, we consider the main sequence to be 
flattened to its measured natural width, as seen in the bottom panel of 
\cref{fig:apogee_teff_trend}. The persistence of the binary sequence in the 
residuals indicates that our corrections are meaningful.  

\subsection{Corrections for Photometric Sample}

The photometric sample in \citet{McQuillan14} differs from the spectroscopic
sample in two main ways: the lack of metallicity diagnostics, and the necessity
of using photometric instead of spectroscopic \Teff{}. Both of these
differences will increase the uncertainty of the inferred single-star K-band
luminosity, and we measure their impact in the following analysis.

For the full \citet{McQuillan14} sample, where individual metallicities are not
available, we need to select an isochrone with a metallicity that is 
representative of the full sample. We do this by assuming that metallicity 
distribution of the spectroscopic targets is representative of the full 
\Kepler{} field. 

\begin{figure*}[htb]
    \centering
    \plotone{metallicity}
    \caption{\emph{Left:} Metallicity distribution of the \Kepler{} cool 
        dwarfs in the APOGEE sample, selected as having 
        \(\Teff < 5250 \textrm{ K }\) and a luminosity less than 1.3 mag above 
        the single-star sequence. The 
        thick black line denotes the median metallicity while the thin black 
        lines denote the 1-\(\sigma\) confidence intervals. The hatched region
        to the left of [Fe/H]=-0.5 denotes the metallicity beneath which the 
        empirical correction in \cref{fig:met_trend} is poorly constrained due 
        to too few metal-poor stars. \emph{Right:} The vertical displacement 
        caused by assuming a single metallicity isochrone.  The blue line 
        represents the difference between the derived \(M_K\) assuming the 
        median field metallicity and that for the true metallicity. The 
        orange line includes our empirical shape correction for metallicity, 
        which removes most of the predicted width. The hatched region
        to the left of [Fe/H]=-0.5 denotes the metallicity beneath which the 
        empirical correction is poorly constrained due to too few metal-poor 
        stars.}\label{fig:metallicity}
\end{figure*}

Because we will mostly be concerned with the cool dwarfs, we check the
metallicity distribution of dwarfs cooler than 5250 K, which is the 
temperature range of our sample. The metallicity distribution for that
sample is shown in \cref{fig:metallicity}. The median metallicity is 
0.08, consistent with that observed for planet hosts by 
the California \Kepler{} survey \citep{Petigura17}.  We therefore treat the \([Fe/H] =
0.08\) isochrone as the base isochrone for the photometric sample. However, the
metallicity of the base isochrone does not substantially affect the analysis
because any zero-point offset will be subtracted off in the following
correction.

\begin{figure}[htb]
    \centering
    \plotone{Teff_relation}
    \caption{Comparison between the APOGEE spectroscopic \Teff{} and the
        \citet{Pinsonneault12} \Teff{} for objects with \(4000 \textrm{ K } \le
    \textrm{ APOGEE \Teff{} } \le 5250 \textrm{ K }\). A linear fit for the
    temperature is shown as the solid line. The best-fit model is
    \(T_{\textrm{eff,phot}} = 1.153 T_{\textrm{eff,spec}} - 682.6\). The 
    one-to-one line is shown as a dashed line. The scatter between the two 
    datasets is 135 K. A representative error for both datasets is shown in 
the bottom-right corner.}\label{fig:teffdiff}
\end{figure}

We also estimate the impact of the photometric temperatures by directly 
comparing the spectroscopic and photometric temperatures for the set of 
overlapping points shown in \cref{fig:teffdiff}. Overall, the two temperature scales are reasonably
correlated. There is a temperature scatter of 135 K between the two scales.
Taking into account the 100 K uncertainty in the APOGEE \Teff{} scale, this
implies a 90K uncertainty in the \citet{Pinsonneault12} temperatures, which
agrees with the uncertainties expected from the photometric method. 

\begin{figure}[htb]
    \centering
    \plotone{phot_teff_correction}
    \caption{Same as \cref{fig:apogee_teff_trend}, except using
        \citet{Pinsonneault12} temperatures, and subtracting off a
    constant [Fe/H]=0.08 isochrone for all objects instead of a 
    metallicity-adjusted isochrone. The best-fit relation is \(\Delta \MK{} =
-0.0000937 \Teff{} + 0.434\). }\label{fig:photuncor}
\end{figure}

Combining the two changes: subtracting only a single-metallicity isochrone and
using the photometric temperatures yields the points in \cref{fig:photuncor}.
This sample still has a trend with temperature, that we remove using a linear
fit, similar to that done for the spectroscopic sample.

\begin{figure*}[htb]
    \centering
    \plotone{excess_hist}
    \caption{\emph{Left:} Histogram showing the distribution of vertical 
        displacements with isochrones corresponding to APOGEE metallicities.
        The blue line denotes the best-fit double-Gaussian model to the
        distribution of vertical displacements. The red line is a fit to the
        best-fit double-Gaussian model assuming a single metallicity (see right
        panel). Conservative and inclusive photometric binary thresholds at 
        \(\Delta K < -0.3\) mag and \(\Delta K < -0.2\) mag are shown as violet and green dashed 
        lines, respectively. The best-fit dispersion of the single-star 
        Gaussian is 0.086 mag. \emph{Right:} Histogram showing the 
        distribution of vertical displacements with a \([Fe/H] = 0.08\) 
        isochrone. The red line denotes the best-fit double-Gaussian model to
        the distribution of vertical displacements. The blue line is a fit to
        the best-fit double-Gaussian model for isochrones adjusted for the
        APOGEE metallicity (see left panel). Conservative and inclusive
        photometric binary thresholds are shown as violet and green dashed 
        lines, respectively. The best-fit dispersion of the single-star 
        Gaussian is 0.117 mag. The blue points at the top of the plot show the
    mean and standard deviation of the single and binary Gaussian fits for the
sample with metallicity information. The red points show the mean and standard
deviation of the single and binary Gaussian fits for the sample without
metallicity information.}\label{fig:histcompare}
\end{figure*}

Now that we have corrected vertical displacements over the 4000--5250 K
temperature range, we can now compare the main-sequence scatter for the two
datasets. We do this by generating histograms for the spectroscopic and
photometric samples over vertical displacement, shown in
\cref{fig:histcompare}. We try to characterize the width of the two samples by
fitting a double Gaussian to each histogram, and taking the width of the
main-sequence Gaussian as the uncertainty in the vertical displacement in the
vertical displacement. Our fits indicate that the scatter for the spectroscopic
sample is 0.086 mag while the scatter for the photometric sample is 0.117 mag.

\begin{figure*}[htb]
    \centering
    \plotone{metallicity_scatter}
    \caption{\emph{Left:} The distribution of vertical displacements imposed
    by the metallicity distribution of the APOGEE sample using the raw MIST
    isochrones. All stars are assumed to have a known temperature of 5000 K. 
    Not shown are two very metal-poor stars which are off the plot to the 
    right. \emph{Right:} Same as the left, except the MIST K-band magnitudes
are corrected by the empirical correction in
\cref{fig:met_trend}.}\label{fig:met_spread}
\end{figure*}

This modest increase in the scatter for the photometric sample is surprising,
especially given the strong dependence of K-band magnitude on metallicity shown
in \cref{fig:metallicity} predicted by the MIST isochrones. We believe that the
reduced scatter is real, and that the MIST bolometric corrections overestimate 
the dependence of K-band luminosity on metallicity. We demonstrate that the
metallicity dependence of the isochrones by measuring the scatter in vertical
displacement of an artificial population of stars which differ only by 
metallicity. We assume all stars have a temperature of 5000 K, have an age
of 1 Gyr, and have metallicities drawn from the APOGEE cool dwarf distribution
shown in \cref{fig:metallicity}.

We display two realizations of this population in \cref{fig:met_spread}: one
using K-band magnitudes predicted by MIST, and another applying the
metallicity-dependent correction in \cref{fig:met_trend}.
The left-hand panel illustrates the distribution of vertical displacements 
using the raw MIST K-band absolute magnitudes while the right-hand panel
uses the empirically-corrected K-band magnitudes. The scatter of the two
populations are 0.14 mag and 0.10 mag, respectively, indicating that the
corrected distribution has a lower scatter due to metallicity.
As is evident from the corrected sample on the right side of 
\cref{fig:met_spread}, the reduction in the width of the main 
sequence is driven by a cutoff at large, positive vertical displacements.
The origin of this cutoff comes from the metallicity-dependence of the empirical 
correction in \cref{fig:metallicity}; metal-poor stars are forced to smaller 
vertical displacements compared to the MIST predictions. We note that the MIST 
models predict that the metallicity dependence K-band is similar to the V-band,
which indicates that this is a feature of hte luminosity dependence of the main
sequence locus, not a peculiar feature of the K-band bolometric corrections.

\section{Results and Discussion}
\label{sec:results}

\begin{figure}[htb]
    \centering
    \plotone{ebdist}
    \caption{The observed period distribution of all eclipsing binaries with
    \Teff{} < 5250 K from \citet{Pinsonneault12} and vertical displacement less
than 1.3 mag is shown in red. The period distribution of the full binary
distribution inferred from the eclipsing binaries is shown in black. The
normalization factor is the total number of Kepler cool dwarfs. Error bars
represent 1-\(\sigma\) Poisson confidence intervals.}\label{fig:ebdist}
\end{figure}

The most fundamental sample for understanding binarity in the \Kepler{} field
is the \Kepler{} Eclipsing Binary sample \citep{Prsa11,Kirk16}. With
\Kepler{}'s continuous monitoring and a long baseline of 4 years, the eclipsing 
binary sample for orbital periods on the order of 20 days should be complete
\citep{Kirk16}.
Whether a binary is observed to eclipse or not depends solely on its geometry,
with an eclipse probability for circular orbits given as \((R_1 + R_2)/a\), where \(R_1\) and 
\(R_2\) are the primary and secondary radii, and \(a\) is the semimajor axis of
the orbit. Given the period distribution of the eclipsing binaries, as well as
assumptions about the masses and radii of the component stars, the period
distribution of the full binary sample can be derived using just the eclipse
probability. This expression assumes circular orbits, which is justified because
short-period, low-mass systems are overwhelmingly observed to be circularized
\citep{Raghavan10,VanEylen16}. To simplify the calculation, we
assume that the secondaries are half as massive as the primaries, which is
generally consistent with a flat mass-ratio distribution \citep{Raghavan10}. 
More massive secondaries would reduce the predicted number of non-eclipsing 
systems. Given these assumptions, an estimate of the orbital period distribution 
for all short-period \Kepler{} binaries on the unevolved lower main sequence is 
shown in \cref{fig:ebdist}, assuming radii and masses inferred from \Teff{} 
using MIST isochrones.

The eclipsing binaries range from making up almost 15\% of the total population 
of binaries in the in the shortest-period bin in \cref{fig:ebdist} to 6\% in the 
longest-period bin, hence they are not a negligible segment of the
population. \citet{McQuillan14} excluded eclipsing binaries from their analysis
of rotation in the \Kepler{} field due to concerns that the eclipse signal would
dominate the period. In order to obtain an unbiased view of the binary
population at short periods, we include the \Kepler{} eclipsing binaries in our
statistical analysis. This treatment assumes that orbital period is identical
to the rotation period, which should certainly be true for synchronized
systems; we address the question of synchronicity directly later on.

After successfully calculating the vertical displacement of the full sample, we
now attempt to distinguish between single and binary stars. As noted in the
previous section, the 1-\(\sigma\) uncertainties of the single-star sequence 
were 0.086 mag with known metallicity, and 0.117 mag when metallicity was unknown. 
To implement a uniform photometric binary cut for both the spectroscopic and
photometric samples, we will assume a measured width of the main sequence of 
0.1 mag. To show that our findings are robust to the choice of photometric
binary threshold, we perform our analysis for two photometric binary thresholds: 
an inclusive 2-\(\sigma\) threshold and a conservative 3-\(\sigma\) threshold,
which turn out to be displacement cuts of 0.2 and 0.3 mag. To prevent age effects 
from biasing the results, we only performed statistics in the 4000--5250 K 
range.

\begin{figure*}[htb]
    \centering
    \epsscale{1.1}
    \plotone{apogee_rapid_excess}
    \caption{\emph{Top Left to Bottom Right:} Vertical displacement of cool 
        APOGEE targets with \citet{McQuillan14} periods >10 days, between 
        7--10 days, 1.5--7 days, and <1.5 days. Pink stars denote 
        eclipsing binaries with orbital periods within the same ranges. The 
        green and purple lines denote the inclusive and conservative 
        photometric binary thresholds, respectively. The temperatures are from
    APOGEE.}\label{fig:apogee_rapid_excess}
\end{figure*}

For both binarity thresholds, the well-characterized APOGEE sample 
illustrates the differences in binarity between slow and rapid rotators. For
the purpose of illustration, we divide the sample into four bins, with edges at 
1.5, 7, and 10 days. We draw a short-period limit for our analysis at 1.5 days 
because we want to eliminate potential contamination from ellipsoidal variables 
and semidetached systems \citep{VanEylen16}. We find 7 days to mark the
transition from a binary-dominated synchronized population to a single-star
dominated population (see below). And 10 days is the theoretically and
observationally-motivated boundary in orbital period where synchronization
should take place \citep{Claret97,Lurie17}.

Vertical displacements in those bins of rotation period are shown in
\cref{fig:apogee_rapid_excess}. Nearly all of the rapid rotators with periods 
between 1.5--7 days are photometric binaries, in contrast to the slow rotators
with periods greater than 10 days, which have a distinct single star sequence. 
Angular momentum loss due to stellar winds acting over a billion years is
expected to deplete the short-period regime of single stars. The transitional 
periods between 7--10 days have a binary fraction intermediate to the 
rapid and slow regimes. The bin with periods shorter than 1.5 day is not expected
to behave like detached, eclipsing binaries due to potential contamination from 
contact binaries, ellipsoidal variables and blended pulsators \citep{VanEylen16}. 

To quantify the significance the excess binary fraction in the rapid rotators, 
we compare the photometric binary fraction of the short-period bin to the
binary fraction of the full sample. In order to accurately characterize the 
binarity of short-period systems, we include the \Kepler{} eclipsing binaries, 
which were excluded by \citet{McQuillan14} to avoid contaminating signal from 
the eclipses. The shortest-period systems are expected to be synchronized, so
we assume that the rotational period is identical to the orbital period. 
The eclipse probability for these short-period binaries can be as high as 40\% 
\citep{Kirk16}, hence ignoring this population would severely
bias the underlying sample. For systems where the orbital period is too long
for synchronization, this assumption would break down and the orbital period
distribution would behave differently from the rotational period distribution.
We will perform a check for this effect further below.

We find that the photometric binary fraction with the inclusive and conservative 
threshold for the full APOGEE sample with temperatures between 4000--5250 K is 
38\% and 23\%, respectively. For the bin with rotational periods between 
1.5--7 days the binary fraction using the inclusive and conservative cuts is
71\% and 67\%, which are larger than the binary fraction of all cool APOGEE 
dwarfs by a p-value of \(5 \times 10^4\) and \(4 \times 10^4\), respectively. 
The binary fraction for stars with periods between 7--10 days is 48\% for both 
the inclusive and conservative cuts; they are marginally excluded with p-values
of 0.24 and 0.01. For a long-period interval from 10--15 days, the binary 
fractions for the inclusive and conservative thresholds 37\% for both with 
p-value of 0.7 and 0.3, indicating that the binary fraction is representative 
of the full sample.

\begin{figure*}[htb]
    \centering
    \epsscale{1.1}
    \plotone{full_mcquillan_rr_excess}
    \caption{\emph{Top Left to Bottom Right:} Vertical displacement of all 
        cool \citet{McQuillan14} targets in period bins >10 days, 
        7-10 days, 1.5-7 days, and <1.5 days. Pink stars 
        denote eclipsing binaries with orbital periods within the same ranges. 
        The green and purple lines denote the inclusive and conservative 
    photometric binary thresholds, respectively. The temperatures are from
\citet{Pinsonneault12}.}\label{fig:mcq_rapid_excess}
\end{figure*}

While the spectroscopic sample can show trends between rotation and binarity in
large bins, the substantially larger photometric sample is needed to 
effectively characterize the rapid rotators. The vertical displacement of stars 
in each period bin is shown in \cref{fig:mcq_rapid_excess}. With the 
substantially larger number of objects in each bin, the absence of the
single-star sequence in the rapid rotators is starker.

\begin{figure*}[htb]
    \centering
    \epsscale{1.1}
    \plotone{mcquillan_transition}
    \caption{Same as \cref{fig:mcq_rapid_excess} except with the period ranges
    7--9 days, 9--11 days, 11--13 days, and 13--15 days.}
    \label{fig:mcquillan_transition}
\end{figure*}

The appearance of the single-star sequence is shown in
\cref{fig:mcquillan_transition}, where we zoom in on a narrower range of
periods. The single-star sequence appears in the
hottest bin first, with cooler stars having a longer-period tail. The trend
that hotter stars in the field rotate more quickly than the cooler stars 
is consistent with the overall trend in temperature seen in 
\citet{McQuillan14}.

\begin{figure*}[htb]
    \centering
    \plotone{binary_fraction}
    \caption{\emph{Top Left:} The photometric binary fraction as a function of 
        period for the inclusive photometric binary threshold (\(\Delta K <
        -0.2\) mag). The photometric binary fraction
        for the full sample analyzed by \citet{McQuillan14} is shown as the
        dotted line. Error bars are 1-\(\sigma\) binomial confidence
        intervals. \emph{Top Right:} Same as the previous plot, except using
        the more conservative threshold (\(\Delta K < -0.3\) mag). \emph{Bottom
        left:} The period distribution of the photometric binary (solid green) 
        and photometric single (dashed purple) samples, each normalized by the 
        total number of photometric binaries and photometric single stars 
        analyzed by \citet{McQuillan14}, respectively, using the inclusive
        photometric binary threshold. Error bars are 1-\(\sigma\) Poisson 
        confidence intervals. \emph{Bottom Right:}
        Same as the previous plot, except with the more conservative vertical
        displacement threshold.}\label{fig:binary_fraction}
\end{figure*}

In order to understand the low-period edge of the single-star population, as
well as the cutoff period for the tidally-synchronized binaries, we need to
quantify the behavior illustrated in
\cref{fig:mcq_rapid_excess,fig:mcquillan_transition}. One measure to trace
changes in behavior between the binaries and single stars is the photometric
binary fraction as a function of rotation period, shown in the top row of 
\cref{fig:binary_fraction}. The rapid 
rotators show a uniquely high binary fraction not seen in any other period 
range, which is robust to the choice of detection threshold. As was 
qualitatively shown in \cref{fig:mcquillan_transition}, the binary fraction 
begins to decrease around 7 days, which is expected for the contribution of the
single-star tail. However, it does not reach the mean binarity of the full 
sample until around 9--11 days, after which the single-stars dominate the
population.

We also find that the period dependence of the photometric binaries and
photometric singles differ in the short-period domain. The fraction of
photometric binaries as well as the fraction of
photometric single stars is given in the bottom row of
\cref{fig:binary_fraction} as a function of period for both the inclusive and 
conservative photometric binary thresholds. The number of photometric binaries 
and photometric single stars both decrease in the short-period regime compared to the 
long-period regime; however the photometric single stars decrease much more 
steeply than the photometric binary stars. The sudden drop in photometric 
single stars implies a mixed photometric binary/single population at long 
periods, and a binary-dominated population at short periods. As seen with the
photometric binaries, a tail of single stars appears at rotation periods 
greater than 7 days for both the inclusive and conservative photometric binary 
thresholds. The relative contributions of photometric and single stars then 
become fixed past 11 days, consistent with the flattening of the photometric 
binary fraction. 

Given the radical difference in behavior between the photometric binaries and
photometric single stars in the short-period regime, we propose that the 
rapid rotators are dominated by tidally-synchronized binaries. The 
short-period systems that do not show significant vertical displacement are 
likely low mass-ratio systems. This interpretation of the photometric singles 
in the rapid rotators is supported by comparing the photometric binary 
fraction of the eclipsing binaries to that of the rapid rotators. The 
photometric binary fraction of the eclipsing binaries with orbital periods 
between 1.5--7 days is \(83^{+7}_{-10}\%\) with the inclusive threshold and 
\(76^{+8}_{-11}\%\) with the conservative threshold, while the photometric 
binary fraction of the rapid rotators with rotation periods between 1.5--7
days is \(67^{+3.3}_{-3.5}\%\) with the inclusive threshold and
\(59^{+3.6}_{-3.6}\%\) with the conservative threshold.  Given the
uncertainties, the two photometric binary fractions are just outside of one 
sigma away from each other.

\begin{figure}[htb]
    \centering
    \plotone{eclipseprob}
    \caption{Comparison of rapid rotator period distribution and eclipsing
    binary. The period distribution of the \citet{McQuillan14} sample is shown
as a blue solid histogram, while the period distribution of the eclipsing binaries
is shown as the red histogram. Both distributions are normalized to the total
number of objects analyzed by \citet{McQuillan14} and in the EB catalog. The predicted EB 
distribution assuming the \citet{McQuillan14} sample consists of binaries 
inclined enough to show starspot modulation, but not inclined enough to eclipse. 
Error bars represent 1-\(\sigma\) Poisson confidence 
intervals.}\label{fig:eclipseprob}
\end{figure}

The short-period rapid rotators also behave like a population of non-eclipsing
binaries, providing further evidence that the rotation period and orbital
periods are identical. We test the assumption of synchronicity by comparing the
rapid rotator population to the non-eclipsing population predicted by the
eclipsing binaries. We predict the population of rapid rotators using the same
assumptions as used in \cref{fig:ebdist}. We additionally correct the rotational 
period distribution for inclination effects. K2 observations of the Pleiades 
found that 8\% of the sample did not show variations due to starspots 
\citep{Rebull17}, which is similar to the fraction of objects previously found 
to have inclinations too high to show starspots \citep{Jackson10}. Since 
rotation periods in the Pleiades are on the order of several days, comparable to 
the rotation periods in this sample, we treat 8\% of the total sample as being 
unobservable to starspot periods. Given these assumptions, and the observed 
population of eclipsing binaries, we predict the period distribution of 
non-eclipsing binaries in \cref{fig:eclipseprob}. 

The two period distributions match well up to 7 days, indicating
that the rotational period distribution is reflects the orbital period
distribution for the eclipsing binaries. The period at which the data diverges
from the prediction is also similar to the period where the binary fraction 
drops in \cref{fig:binary_fraction}. The concordance of these two periods 
suggests that the short-period systems follow a 
rotational period distribution compatible with the eclipsing binaries, while 
the long-period systems become dominated by a different 
period distribution. Our physical interpretation for this phenomenon is that the
short-period systems are tidally-synchronized binaries, which remain 
synchronized at least through an orbital period of 7 days. At periods longer
than 7 days, the sample becomes a mix of synchronized binaries and a 
rapidly-rotating tail of single stars and wide binaries following single-star angular momentum evolution.

\begin{figure}[htb]
    \centering
    \plotone{photometric_massratio}
    \caption{Vertical displacement of a binary with a 0.7 M\(_\sun\), solar
    metallicity primary as a function of mass ratio using MIST models. The
violet and green dashed lines denote the mass-ratio at which the vertical
displacement exceeds -0.3 and -0.2  mag, which are mass-ratios of 0.79 and
0.68.}\label{fig:photometric_massratio}
\end{figure}

We attempt to generalize our results from just the photometric binaries to the
full binary population. This generalization requires a mapping between the
observed photometric binary fraction and the true binary fraction. We use the
MIST models to predict the fraction of binaries which will be detected as
photometric binaries under the assumption of a flat mass-ratio distribution
\citep{Raghavan10}. As illustrated in \cref{fig:photometric_massratio}, for a 
0.7 solar mass star only mass-ratios greater than 0.70 result in photometric
binaries according to the conservative binarity threshold, implying that only
30\% of the binary population in our sample should be detected as a photometric 
binary, even if the secondary is not drawn from a steep relative IMF. This rough 
calculation by itself is incompatible with the observed 
short-period photometric binary fraction in \cref{fig:binary_fraction} of 0.75
This contradiction requires that our assumption of a flat mass-ratio function 
cannot hold, or that there are selection effects in the sample. We find that
the median distance of the photometric single stars is 509 pc, while the median
distance of the photometric binaries is 653 pc, consistent with photometric 
binaries being detected to a larger volume than photometric singles. Attempting 
to reconstruct the intrinsic mass-ratio distribution of tidally-synchronized 
binaries is beyond the scope of this analysis.

We validate many of these trends by revisiting them in the smaller, but more
well-characterized spectroscopic sample. The spike in the binary fraction
between 1--7 days (top rows of \cref{fig:binary_fraction}) can be seen in the 
spectroscopic sample as well as the drop-off in photometric single stars at
short periods (bottom rows of \cref{fig:binary_fraction}). However, the
spectroscopic sample is so small that all rotational periods end up being
consistent with the orbital period distribution of eclipsing binaries because
the binomial error bars are too large.

Under the assumption that the rapid rotators with rotation periods less than 7
days are tidally-synchronized binaries, while longer-period systems contain a
mix of synchronized binaries and single stars, we can infer a lower limit
to the synchronization period of 7 days. The original formulation by
\citep{Zahn77} included an expression for the synchronization timescale for a
binary system given as \(t_{sync} = 10^4 ((1+q)/2q)^2 P^4 / R^6\) where \(t_{sync}\) is
in years, \(q\) is the mass-ratio, \(P\) is in days, and \(R\) is the radius in
solar radii. For a 4 Gyr old equal-mass field binary, this
corresponds to a synchronization period of 15 days, which is well within the
constraints. \citet{Claret97} revisited the formulation of \citet{Zahn77} and
instead advocated substituting a different expression for the apsidal motion 
constant which made tidal dissipation on the main sequence a factor of 5
smaller. The new expression corresponds to a synchronization period of 10 days,
which is also consistent with our findings.

\begin{figure}[htb]
    \centering
    \plotone{ElBadry_Excess}
    \caption{The sample of APOGEE targets analyzed by \citet{ElBadry18b} for
    spectroscopic signs of a companion. Black points indicate the APOGEE DR14 
values for the targets. Violet points indicate the revised temperatures, and the
K-band luminosity excess derived from the revised temperature and metallicity
in \citet{ElBadry18b}.}
    \label{fig:elbadry_excess}
\end{figure}

While we assumed that objects showing large vertical displacement were
binaries, our sample includes an abundance of systems with vertical
displacements greater than 0.75 mag, some greater than even 1.3 mag. One
obvious explanation for these systems is that they are multiple systems of
comparable mass. Some of these targets may be multiple systems where each
component contributes substantially to the luminosity. One other explanation is 
that the secondary causes biases the observed temperature to be cooler than the 
actual primary temperature causing the single-star luminosity to be 
underestimated. This phenomenon occurs for both photometric and spectroscopic 
temperatures \citep{Pinsonneault12,ElBadry18a}. 

We can test the effect of binarity on the estimated temperature by using the
APOGEE DR13 targets whose spectra were fit for multiple solutions
\citep{ElBadry18b}. In cases where a two-component fit was favored, temperature
were published for both the primary and secondary. With \Gaia{} DR2 parallaxes,
we can actually test how often two photometric components were successfully
detected. All the DR13 objects analyzed by \citet{ElBadry18b} are shown in
\cref{fig:elbadry_excess} with both the ASPCAP and decomposed temperatures.

The overall behavior in the \citet{ElBadry18b} sample is that the effect of 
binarity is substantial, leading to incorrectly-inferred magnitude excesses of
up to 0.3 mag. For many of the intermediate mass-ratio objects, the 
ASPCAP-inferred primary temperature may be off by several hundred Kelvin, 
leading an overestimation of the K-band vertical displacement by up to 0.3 
mag. The difference in temperature is less severe for targets with high 
luminosity excesses, which is to be expected because the two stellar 
components would have similar spectra. However, the effect is still 
non-negligible. The overestimation of vertical displacements may also contribute 
to the overabundance of photometric binaries with respect to a flat mass 
distribution seen above. 

While the two-component fit produces results that overall seem reasonable, many
of its decompositions were not compatible with the vertical displacement. Many
of the photometric binaries were not flagged as requiring two-component fits.
Additionally, two of the targets showing essentially no vertical displacement
were flagged as having substantial modifications to the temperature from a
companion. Nonetheless, the results of the analysis by \citet{ElBadry18b} are
impressive given that it did not make use of \Gaia{} parallaxes. Future
analyses may want to make use of the vertical displacement when flagging
spectra for multiple-component fits.

\section{Conclusion}
\label{sec:conclusions}

We calculated K-band luminosity excesses above the main sequence for the full
\citet{McQuillan14} sample. As seen in both a spectroscopically-characterized
subset as well as in the general sample, rapid rotators with periods between
1--7 days show substantially larger luminosity excesses than the slower
rotators. At periods longer than 7 days, the rapidly rotating tail of the
single-star population grows more numerous before dominating the sample at
periods longer than 9 days. 

We propose that the excess of binaries among rapid rotators is a sign that
tidally-synchronized binaries dominate the short-period end of the rotation
distribution. When comparing the rotational period distribution of the rapid 
rotators to the orbital period distribution of \Kepler{} eclipsing binaries, we
find that the rapid rotators are consistent with being synchronized, 
non-eclipsing binary systems showing starspot modulation, up to periods of 7
days.

Our hypothesis that the rapid rotators are tidally-synchronized systems can be
readily tested using time-resolved moderate-resolution spectroscopy. A binary
with two 0.7 M\(_\sun\) stars in a 7 day period orbit should have an RV
semiamplitude of 62 \kms. If these systems were truly synchronized, the
RV-derived orbital period should be identical to the photometric rotation
period. A list of the rapid rotators with 1 day < \(P_{rot}\) < 7 day, along
with relevant information is shown in \cref{tab:rapidrot}.

Assuming that the rapid rotators are tidally-synchronized systems, we find a
lower limit to the synchronization period of 7 days imposed by the rapid tail
of the population experiencing single-star angular momentum evolution, which is
consistent with most current theories of stellar tides. To derive a more
stringent upper limit, either the rapidly-rotating tail of the single-star
distribution needs to be modeled and subtracted, or stars undergoing 
single-star angular momentum evolution need to be excluded. The latter can be
done with either radial velocity monitoring to measure radial velocity
variability and to confirm the orbital period, or by measuring rotation periods
for the eclipsing binaries, as done by \citet{Lurie17}.

RV follow-up of our non-eclipsing close binaries will provide a good resource
for understanding these systems. RV orbital periods for non-eclipsing systems 
will provide crucial confirmation of the subsynchronous binary population 
discovered in \citet{Lurie17}. Our sample would also make an ideal sample for
studies of the mass-ratio distribution for close binaries. Because it was
detected through rotation, it should only be weakly biased with mass-ratio,
which has long been a problem with studies of stellar multiplicity
\citep{Halbwachs03}.

A background population of tidally-synchronized binaries requires caution for
interpreting gyrochronology in the \Kepler{} field. Attempting to characterize
the age distribution of the \Kepler{} field through gyrochronology without 
taking synchronized binaries into account would under-estimate the age, 
potentially biasing future population studies of \Kepler{} stars or planets. 
In order to successfully short-period systems into a gyrochronology, either age 
diagnostics independent of rotation or activity would have to be used, or
tidally-synchronized binaries would have to be individually excluded, or
modeled as a population.

We emphasize that a tidally-synchronized background is not unique to the
\Kepler{} field. All studies of rotation, including previous work in clusters,
will eventually have to incorporate tidally-synchronized binaries to
successfully calibrate gyrochronology and angular momentum evolution models.
The \Kepler{} field, which is rich in synchronized binaries, is an excellent 
source of additional systems to study.

We also note that the K-band bolometric corrections in the MIST isochrones
overpredict the dependence on metallicity, even near solar metallicity. We
derived corrections for trends imposed by the MIST isochrones which reduced the
scatter in the single-star sequence. In the era of \Gaia{} parallaxes,
additional testing can be done to validate bolometric corrections in the
sparsely-calibrated NIR region.

\acknowledgments

G.S, M.P. and D.T acknowledge support from NASA ADP Grant NNX15AF13G and from
the National Sscience Foundation via grant AST-1411685 to The Ohio State
University. 
Funding for the Sloan Digital Sky Survey IV has been provided by the Alfred P.
Sloan Foundation, the U.S. Department of Energy Office of Science, and the
Participating Institutions. SDSS-IV acknowledges support and resources from the
Center for High-Performance Computing at the University of Utah. The SDSS web
site is www.sdss.org. SDSS-IV is managed by the Astrophysical Research
Consortium for the Participating Institutions of the SDSS Collaboration
including the Brazilian Participation Group, the Carnegie Institution for
Science, Carnegie Mellon University, the Chilean Participation Group, the
French Participation Group, Harvard-Smithsonian Center for Astrophysics,
Instituto de Astrof\'isica de Canarias, The Johns Hopkins University, Kavli
Institute for the Physics and Mathematics of the Universe (IPMU) / University
of Tokyo, Lawrence Berkeley National Laboratory, Leibniz Institut f\"ur
Astrophysik Potsdam (AIP), Max-Planck-Institut f\"ur Astronomie (MPIA
Heidelberg), Max-Planck-Institut f\"ur Astrophysik (MPA Garching),
Max-Planck-Institut f\"ur Extraterrestrische Physik (MPE), National
Astronomical Observatories of China, New Mexico State University, New York
University, University of Notre Dame, Observat\'ario Nacional / MCTI, The Ohio
State University, Pennsylvania State University, Shanghai Astronomical
Observatory, United Kingdom Participation Group, Universidad Nacional
Aut\'onoma de M\'exico, University of Arizona, University of Colorado Boulder,
University of Oxford, University of Portsmouth, University of Utah, University
of Virginia, University of Washington, University of Wisconsin, Vanderbilt
University, and Yale University.  This publication makes use of data products
from the Two Micron All Sky Survey, which is a joint project of the University
of Massachusetts and the Infrared Processing and Analysis Center/California
Institute of Technology, funded by the National Aeronautics and Space
Administration and the National Science Foundation. 

\facility{Kepler, Gaia, CTIO:2MASS, ARC}

\software{MIST \citep{Choi16}, Astropy \citep{astropy}, IPython \citep{PER-GRA:2007}, Scipy
\citep{jones_scipy_2001}, NumPy \citep{van2011numpy}, Matplotlib \citep{Hunter:2007} }

\onecolumngrid
\input{tables/table1}

\bibliographystyle{aasjournal}
\bibliography{references}

\end{document}




