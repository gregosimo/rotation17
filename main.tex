% aastex6.1 will be released as part of TeXLive 2017 around June 1. 
\documentclass[manuscript]{aastex6}

% Make the underscore behave correctly.
\usepackage[T1]{fontenc}
% Import images
\usepackage{graphicx}
% Used for consistent handling of Figures/Sections/Tables
\usepackage[capitalize]{cleveref}
% Format URLs correctly
\usepackage{url}
\usepackage{amsmath}
\usepackage[T1]{fontenc}
\usepackage{aecompl}
\usepackage{color}

% Want to have a separate storage folder for my figures.
\graphicspath{{./fig/}}

% Settings for cleveref to make Figures and Tables more appropriately
% defined.
% I want figure to be abbreviated.
\crefname{figure}{Fig.}{Figs.}

\newcommand{\vsini}{\ensuremath{v \sin i}}
\newcommand{\Kepler}{\mbox{\textit{Kepler}}}
\newcommand{\Gaia}{\mbox{\textit{Gaia}}}
\newcommand{\McQuillan}{\citep{McQuillan14}}
\newcommand{\Teff}{\ensuremath{T_{eff}}}
\newcommand{\logg}{\ensuremath{\log(g)}}
\newcommand{\kms}{\textrm{ km~s}\ensuremath{^{-1}}}

% Targeting flags
\newcommand{\STARBAD}{\texttt{STAR\_BAD}}
\newcommand{\STARWARN}{\texttt{STAR\_WARN}}
\newcommand{\VSINIWARN}{\texttt{VSINI\_WARN}}
\newcommand{\APOGEESEISMO}{\texttt{APOGEE2\_SEISMO\_DWARF}}

\newcommand{\gvs}{\authorcomment1}

\shorttitle{Comparison of Rotation Measurements in \Kepler{}}
\shortauthors{Simonian et al.}
\begin{document}

\title{A Comparison between Spectroscopic and Photometric Rotation in the
\Kepler{} Field}
\author{Gregory V. Simonian; Marc H. Pinsonneault; Donald M. Terndrup}
\affil{Department of Astronomy, The Ohio State University}
\affil{140 West 18th Avenue, Columbus, OH 43210}
\email{simonian@astronomy.ohio-state.edu}

\begin{abstract}
    The \Kepler{} fields have yielded a large data set of photometric rotation
    periods, which when combined with \Gaia{} parallaxes and APOGEE
    high-resolution spectroscopic data for over 4,000 dwarfs and subgiants, 
    will reveal significant insights into the evolution of rotation in stars, 
    In anticipation of radii derived from \Gaia{} DR2 parallaxes, we report on 
    spectroscopic measurements of stellar rotation from APOGEE in an
    easily-characterized sample of over 600 cool dwarfs in the
    \Kepler{} fields. Our data are in good agreement with literature values for
    targets in common, with a detection threshold between 6 and 12 \kms. We
    find a rapid rotation fraction of \(\sim 10\%\), a larger than expected
    population of young stars. When comparing spectroscopic to photometric
    rotation rates, we find that the photometric rapid rotator fraction is
    lower than the spectroscopic fraction by a factor of 4-5. A point-by-point
    analysis reveals that almost all targets have rotation periods too long for
    their measured \vsini{}. We find these results suggest that current 
    large-scale photometric surveys may mis-characterize or exclude cool rapid 
    rotators in the field. \gvs{I may need to add some info to the conclusion
revisiting this.}
\end{abstract}
\keywords{stars: rotation}

\section{Introduction}

One of the significant results from the \Kepler{} satellite has been the
determination of rotational periods for a large sample of old, field stars
\citep{Basri11,Affer12,Nielsen13,Reinhold13,McQuillan14,Garcia14}.
These rotation rates are expected to be extremely useful to deriving a field
gyrochronology, which can better characterize the ages of
\Kepler{} stars, and hence better understand discovered planets.

The \Kepler{} field was chosen to point outside of the plane of the galactic
disk, biasing the field toward older stars. Despite the old age of the
field, there is still a population of rapid rotators at the level of 1\% 
which have rotation rates significantly higher than expected from
existing gyrochronological relations \citep{McQuillan14}. Possible explanation
for these rapid rotators are a truly young stellar population, a background of
tidally-synchronized binaries, contamination of background pulsators in the
\Kepler{} PSF, or confusion with rapidly-rotating subgiants \citep{vanSaders13}. 

All attempts to evaluate the determination of rotation periods in
\Kepler{} field stars have been comparisons between different pereiodograms and
light-curve processing algorithms. The combination of large-scale
high-resolution spectroscopy from APOGEE and uniform radii derived from precise
\Gaia{} parallaxes \citep{Stevens17} will allow for more powerful tests of our
ability to measure stellar rotation, and our interpretation of stellar rotation
data, especially in old, inactive stars where data is sparse.

Previous studies have succesfully combined rotational periods and \vsini{} 
to study systematics in the radii of pre-main sequence stars
\citep{Rhode01,Jeffries07}, the inclination distribution in clusters 
\citep{Jackson10}, and the inference of radius inflation
\citep{Jackson09,Jackson16,Jackson18}. All of these studies were done for
clusters, where large numbers of spectra could be observed and analyzed under 
the assumption of uniform distances and composition. The availability of
rotation periods and \vsini{} for field stars will be instrumental in 
illuminating other phenomena such as gyrochronology 
\citep{Barnes07,Mamajek08,Angus15}, radius inflation \citep{Jackson18}, and 
rotation on the subgiant branch \citep{vanSaders13}.

In this paper, we study the rotation distribution of \Kepler{} targets as
seen from APOGEE \vsini{}s and compare to published periods. In 
Section~\ref{sec:sample} we describe the
selection of cool dwarfs to ensure a reliable comparison between
\vsini{} and period. In Section~\ref{sec:analysis}, we describe the reliability
of the quantities we need in order to compare rotation via \vsini{} and
photometric modulation. We also demonstrate our methodology works with the 
APOKASC asteroseismic sample. In Section~\ref{sec:fraction}, we compare the 
rapid rotator fraction from photometric studies to that found using APOGEE 
spectra. In Section~\ref{sec:results}, we compare the two measures of 
rotation on a point-by-point basis and postulate reasons for disagreement. 
Lastly, we summarize our findings and discuss future efforts in 
Section~\ref{sec:conclusions}.

\section{Sample Selection}
\label{sec:sample}

\begin{figure}
    \plottwo{cool_sample}{astero}
    \caption{\emph{Left:} Stellar properties of the cool dwarf sample as
        measured by APOGEE\@. Stars classified as dwarf and subgiants are 
        shown in red and purple, respectively. Rapid rotators are denoted with 
        cyan stars. The dotted vertical line indicates the \(\Teff = 5450\) K
        boundary.
        \emph{Right:} Stellar properties of the asteroseismic dwarf sample. 
        For the asteroseismic dwarfs showing oscillations, 
        asteroseismically-derived \logg{}s are marked as green stars, while 
        spectroscopically-derived \logg{}s are marked as orange circles. The 
        black points mark spectroscopic \logg{}s for the full APOGEE
        asteroseismic dwarf control sample. \label{fig:sample}}
\end{figure}


The central dataset for this analysis is from the Apache Point Observatory 
Galactic Evolution Experiment (APOGEE) \citep{Blanton17,Majewski17}. The
APOGEE instrument is a high-resolution (\(R \sim 22,000\)) multi-fiber 
near-infrared spectrograph \citep{Wilson10}, mounted on the SDSS 2.5-meter 
telescope at Apache Point Observatory \citep{Gunn06}. Observations for each
target are automatically reduced and combined as part of the APOGEE data 
pipeline \citep{Nidever15}. Stellar parameters are then determined from the 
combined spectra through the APOGEE Stellar Parameters and Chemical Abundance 
Pipeline (ASPCAP), which outputs \Teff{}, \logg{}, \([M/H]\), \vsini{}, as 
well as 15 additional chemical abundances \citep{GarciaPerez16}.

Under several major observing programs, APOGEE systematically observed over 
5,000 dwarfs and subgiants in the \Kepler{} field. Around half were
observed as part of a program to determine false positives for
asteroseismic detections. They were part of a magnitude-limited sample
selected according to 6500 K \(> \Teff > 5000\) K and \(\logg > 3.5\)
according to parameters derived by \citet{Huber14} \citep{Zasowski17}. These
are shown in the right panel of \cref{fig:sample}. An additional 1000 cool 
dwarfs with (\(\Teff < 5500\) K) and \(\logg > 4.0\) as determined
by \citet{Pinsonneault12} and \citet{Brown11} were also observed, and the
dwarf/subgiant subset are shown in the left panel of \cref{fig:sample}. 
Eclipsing binaries and transiting planet host candidates were also observed 
by APOGEE in the \Kepler{} field, but with more complex selection functions, 
which we didn't include in this analysis.

We selected our sample to minimize ambiguity in the stellar radius when
comparing rotation periods to \vsini. Large uncertainties in
determining stellar parameters propogate to the radius estimates. This is 
especially prominent in the hot star regime, where higher-mass subgiants and 
lower-mass dwarfs have easily confused \Teff{} and \logg{}. 

In order to sidestep detailed modeling of the complex radius uncertainties 
for all \Kepler{} targets, which will be simplified greatly with upcoming
\Gaia{} parallaxes, we chose instead to study measures of
stellar rotation in the cool  dwarf regime, where
slow stellar evolution maintains a clean separation between dwarf and
subgiants. 

For our base sample, we chose objects observed as part of the APOGEE cool
dwarf ancillary targeting
program\footnote{\url{http://www.sdss.org/dr14/algorithms/ancillary/apogee/kepler_dwarfs/}}.
This program targeted stars observed by \Kepler{} with SDSS-corrected
\(\Teff < 5500\) K \citep{Pinsonneault12}, original KIC \(\logg > 4.0\)
\citep{Brown11}, and 2MASS \(7 < H < 11\). While many of these objects
were observed in the first phase of APOGEE with the
\textit{APOGEE\_KEPLER\_COOLDWARF} targeting flag enabled, much of the rest of
the sample was observed in APOGEE-2 under the \textit{APOGEE2\_SEISMO\_DWARF}
flag. We applied the original selection criteria to the new sample in order get
the updated target list \gvs{Use Jen's original list for this?}. As of the DR14, 
the program has been \(\sim 85\)\% completed with 1028 objects having been 
observed. 

We also selected targets based on APOGEE stellar parameter fits. We categorized the 
sample into those with \STARBAD, \STARWARN,
\VSINIWARN{}, and no flag enabled to check the quality of the fits. We
visually inspected the spectra for a representative sample of these
quality classes. 
Visual inspection of targets with the \STARBAD{} flag show that they are 
largely the result of poor subtraction or normalization of the spectrum in the
pipeline, or are due to poor model fits on the cool (\(\Teff < 4250\) K) end. 123
targets in the full sample has the \STARBAD{} quality flag enabled. 
Stars labeled as \STARWARN{} are a significantly larger fraction of the 
sample. We found that the model fits reasonably resembled the target spectra. 
For this reason, we retained targets with the \STARWARN{} flag enabled. The 
same occured for targets with the \VSINIWARN{} flags.

Systematic uncertainties in \logg{} permeated the original KIC stellar
classification. Later analyses found many cases of contamination in the dwarf 
sample by subgiants 
and giants \citep{Mann12,Gaidos13}. Because the APOGEE cool dwarf sample was 
selected based on photometry from the original KIC, we expected contamination 
from subgiant and giants. However, APOGEE stellar parameters, 
despite being uncalibrated for dwarfs and suffering from 
systematic uncertainties, are readily able to distinguish between subgiants and 
dwarfs. This discrimination is demonstrated in the left panel of 
\cref{fig:sample}; there is a relatively clean separation between the dwarf 
and subgiant branches. We used the temperature-dependent \logg{} cut shown in 
\cref{fig:sample} to select 657 dwarfs. 

To control for differences between the photometric and spectroscopic sample,
we performed the bulk of our analysis on the subset of 408 objects
observed in APOGEE which also have period detections in \citet{McQuillan14}.
An additional 205 did not have significant periods in \citet{McQuillan14}, and
will be analyzed separately. The remaining 44 targets were excluded from the
sample selection in \citet{McQuillan14}. 

We also inspected the spectra for every target with \(\vsini > 6 \kms\)
for double-lined spectra. Some targets showed obvious cases of ASPCAP fitting 
a single broad line over two separate narrow features, other cases only revealed the
second component in individual visit spectra. The targets showing double-lined
spectra were excluded from the analysis. There are
also cases where the observed spectra show high RV variability
without a visible second component. We include these cases in our sample
because the measured \vsini{} shouldn't be impacted.


We also found that stars hotter than
\(\Teff = 5450\) K were significantly more difficult to characterize due to
rapid age evolution and overlapping with subgiants. Because only a small 
fraction of our sample is that hot, we removed targets beyond that cut. The
effect of the cut on the sample is shown in \cref{fig:sample}.

\subsection{Asteroseismic Targets}

One of the most well-characterized subsets of the \Kepler{} field is the
APOKASC asteroseismic sample. The APOKASC asteroseismic dwarf and subgiant 
sample consists of 426 targets which presented solar-type oscillations in 
\Kepler{} short-cadence data \citep{Chaplin11}, and were observed in APOGEE DR13
\citep{Majewski17}. Using asteroseismic scaling relations, the radii
for these objects can be determined to an accuracy of around 3\%
\citep{Serenelli17}. Because of their excellent parametrization, we treat the
asteroseismic catalog as a control sample, where stellar properties are
well-known. It should be noted that subgiants are overrepresented in
the asteroseismic population because the amplitude of solar-type oscillations 
increases as stars evolve. The locations of the asteroseismic targets in an HR
diagram is shown in the right panel of \cref{fig:sample}.

The selection of the asteroseismic targets mirrors that of the cool dwarf
sample. From the 426 asteroseismic dwarfs from the APOKASC v4.2.4 catalog, we 
removed three objects with the STAR\_BAD flag, while including 51 objects with 
the \STARWARN{} and 46 objects with the \VSINIWARN{} flags after visual 
inspection. All targets with \(\vsini \ge 6\) \kms{} were visually
inspected for double-lined spectra and four showing them were excluded. 91 of 
the remaining stars have period detections in \citet{McQuillan14}, 240 were 
period nondetections, and 88 did not fall under the selection criteria in 
\citet{McQuillan14}. 

\section{Analysis}
\label{sec:analysis}

In order to consistently compare different measures of rotation, we require 
reliable \vsini{}, rotational periods, and stellar radii. We perform validation
exercises to make sure that the quantities adopted for these three measures are 
reasonable.


\subsection{\vsini{} validation}
\label{sec:vsini_check}

\begin{figure}
    \plottwo{Bruntt_comp}{Pleiades_comp}
    \caption{\emph{Left:} \vsini{} comparison between the \citet{Bruntt12}
        overlap sample with APOGEE\@. A discontinuity in the scatter occurs
        around \(\vsini = 6 \kms\), indicated by the dotted line. The dashed
    line shows the best-fit relation between the two. Not shown are targets run
    through the APOGEE giant grid. \emph{Right:} \vsini{} comparison for the
    Pleiades cool dwarfs \citep{Stauffer87} overlap sample with APOGEE\@. A
    discontinuity in the scatter occurs around \(\vsini = 12 \kms\), indicated by
    the dotted line. 2MASS J03475973+2443528 is not shown. Red points are upper
    limits in \citet{Stauffer87}.\label{fig:comps}}
\end{figure}

The ability to measure \vsini{} is relatively new to the ASPCAP pipeline, only
being included since DR13. As a result, they have not been extensively vetted and 
verified. We undertake verification for our purposes by comparing \vsini{} 
measured in APOGEE to \vsini{} values for
targets studied with high-resolution optical spectroscopy. 

One large collection of \vsini{}s comes from an effort to study the
\Kepler{} asteroseismic targets \citep{Bruntt12}. \citet{Bruntt12} analyzed 
spectra of 93 solar-type asteroseismic targets using two R=80,000 spectrographs 
on the 3.6-meter Canada--France--Hawaii Telescope and 2-m Bernard Lyot
Telescope. While uncertainties are not quoted in \citet{Bruntt12},
previous analyses have determined \vsini{} uncertainties of about 0.6
\kms \citep{Bruntt10a,Bruntt10b}.

A comparison between the APOGEE and \citet{Bruntt12} targets is shown in
\cref{fig:comps}. Not shown are six
objects which the ASPCAP pipeline classified as giants, and omitted
\vsini{}. The \citet{Bruntt12} \vsini{} is less than 5
\kms{} for all of these objects, again making them consistent with 
nondetections.  In order to characterize the APOGEE \vsini{}s, we model the 
two datasets as being linearly related above a specific detection threshold 
with a constant fractional uncertainty. The sample shows a 
discontinuity in the scatter at APOGEE \(\vsini=6\kms\), which we 
interpret as the detection limit of APOGEE, above which \vsini{} yields a 
valid stellar rotation measurement. Since the \citet{Bruntt12} errors are
significantly smaller than the expected APOGEE errors, we performed a simple 
linear regression for all targets with APOGEE \vsini{} above the detection 
limit to obtain the relationship. We got that the two datasets have slopes
consistent with one, but with a significant zero-point offset of 9\%. We 
estimated the measurement error by calculating the mean squared error of the 
residuals, yielding a 13\% uncertainty which we adopted as the 
uncertainty in APOGEE \vsini{}.

There is little overlap between our cool dwarfs and the 
\citet{Bruntt12} sample because the asteroseismic targets tend to be massive and 
highly populated by subgiants. In order to perform a check with a more
representative sample, we compared APOGEE \vsini{}s with  
measurements of cool dwarfs in the Pleiades \citep{Stauffer87}. Because
Pleiades dwarfs are significantly younger than those in the field,
they rotate much more rapidly. This ends up being advantageous because their 
wide distribution of \vsini{}s allows for a wide range of \vsini{}s to be 
validated. We plot the overlap sample with \citet{Stauffer87} in 
\cref{fig:comps}.  Unlike the asteroseismic targets, the Pleiades targets 
seem to experience a discontinuity in behavior at \(\vsini = 12 \kms\). 
Nonetheless, the errors and fit seem consistent above the detection threshold.

The difference in detection limit between the Pleiades and asteroseismic sample
indicates that the ability to measure \vsini{} changes is a function of stellar
parameters. In order to ensure that our results are robust to the choice of
detection threshold, we
perform our analyses with different detection thresholds to show that any
results are persistent.

\subsection{Period Validation}

The validation of rotational periods in the \Kepler{} field has been a
well-studied problem. There have been multiple different approaches to
measuring rotation periods \citep{Reinhold13,Nielsen13,McQuillan14,Garcia14}.
Based on the analysis with \citet{McQuillan14}, their periods agree with
previous analyses for \(\sim\)95\% of the overlap sample, with visual 
inspection of the light curve indicating that much of the disagreement was 
caused by aliases detected by previous works. 

A comprehensive, blinded comparison of periodograms analyzed the performance of
several different algorithms on injected stellar rotation signals 
\citep{Aigrain15}. The Autocorrelation Function method used in \citet{McQuillan14} was 
found to have an extremely high detection fraction while still maintaining 
around 70 percent reliability in detecting an injected signal. Given the 
intensive study of different photometric period-detection routines, we refer 
the reader to \citet{Aigrain15} for a detailed discussion about the 
performance of periodograms.

\subsection{Comparison for Asteroseismic Targets}
\label{sec:astero}

\begin{figure*}
    \plotone{astero_rot}
    \plotone{cool_rot}
    \caption{\emph{Top Left:} Measured \vsini{} for asteroseismic sample plotted
        against inferred \vsini{} from the period and radius. The gray area
        denotes where stars with nonzero inclination should lie. 
        \emph{Top Middle:} Projected radius inferred from \vsini{} and period 
        plotted against the inferred radius from isochrones. Inclination should
        scatter stars into the gray region. \emph{Top Right:} Inferred period 
        from \vsini{} and radius plotted against measured period. Inclination
        should scatter stars into the gray region. \emph{Bottom:} Same as top 
        plots, except using the APOGEE cool dwarf 
    sample.\label{fig:rot}}
\end{figure*}

We first validated our comparison methodology on asteroseismic targets. We 
compared the measured \vsini{}, rotational period, and radius against 
inferred values derived from the other two quantities. The resulting 
comparison is shown in the top panels of \cref{fig:rot}. We distinguished 
between confident \vsini{} detections with \(\vsini > 12 \kms\), and marginal 
detections with \(7 \kms < \vsini < 12 \kms\). Of the confident detections, 
two  have non-physical solutions with \(\sin i \le 1\). 

The marginal detections with unphysical \vsini{} (KIC 8677016, 8952800, 10518551) 
have \vsini{} values close to the detection threshold. Because the errors of 
marginal \vsini{} detections are complex, 
it may be possible that these are outliers. 

Of the confident detections with unphysical \vsini{}, KIC 3223000 has a
\vsini{} which is slightly above the predicted equatorial velocity. It
is a 1.7-\(\sigma\) difference based on the \vsini{} uncertainty, which
may be consistent with being a statistical fluctuation. It can also
be explained by the 4.9 day measured rotation period actually being a 
half-period detection alias, implying a twice as large equatorial velocity. 

The final outlier, KIC 9908400, is so significantly off of the relation
that it can not be explained by either uncertainties or aliasing. However,
it may be adequately explained by contamination of the \Kepler{} PSF\@. 
High-resolution J-band imaging of the \Kepler{} field, provided by the United
Kingdom InfraRed Telescope (UKIRT) reveals a bright, nearby blended companion.
Additionally, KIC 9908400 has a high \Teff{} placing it above the Kraft break;
making spots unlikely. Photometric variation originating from the fainter 
star would explain the mismatch in \vsini{} and rotation period.

For the rest of the objects, a more comprehensive upcoming analysis 
(Simonian et al.\ in prep) indicates that not only are the \vsini{} values 
physically plausible, but their distributions are consistent with the
rotational periods within the given uncertainties.

\subsection{Stellar Radii Validation}
\label{sec:radii}

The main property of cool stars we exploited is that their slow age evolution
constrains stellar radii. We quantified this uncertainty using isochrones from 
the Dartmouth Stellar Evolution Program (DSEP) to estimate the impact of age 
ignorance on the radius uncertainties. We defined our uncertainty as the 
difference in radius between stars of similar \Teff{}, with ages between 1 
and 10 Gyr. For solar metallicity models, the uncertainty in radius due to 
age is about 8.5\% at the hot end of our sample, and decreases sharply with 
\Teff. 

Error bars for the individual stellar radii follow the same convention as
above, except are based on the individual star's temperature and metallicity 
at 1 Gyr and 10 Gyr. Because the ``global'' APOGEE uncertainty for [Fe/H] is 
around 0.05 dex, we rounded [Fe/H] to the nearest 0.05 dex.

A significantly larger systematic uncertainty than the impact of age is the 
misclassification of
subgiants as dwarfs. This uncertainty would certainly lead to underestimating
the stellar radii of dwarfs by several factors. We estimated this
contamination rate by using the asteroseismic sample, shown in the top right
panel of \cref{fig:sample}, which mainly consists of subgiants. 

In order to obtain a representative sample of contaminating subgiants, we
selected only those with \(\Teff < 5500\) K, resulting in a sample of 317
targets. Of these, all but eight have asteroseismic \logg{} on the subgiant
branch. None have APOGEE \logg{}s which scatter from the subgiant onto the 
dwarf branch. The asteroseismic dwarfs which APOGEE places on the
subgiant branch are not concerning because those will not act as contaminants 
in our dwarf sample. Assuming that evolutionary-state classification is a 
random process following a binomial distribution, we can infer a 1-\(\sigma\) 
upper limit of 2.4 subgiant contaminants in the full cool dwarf sample
\citep{Gehrels86}. 

\section{Rapid Rotator Fractions}
\label{sec:fraction}

The broadest check to determine agreement between the spectroscopic
and photometric rotation data is comparing the rapid rotator fraction
in both datasets. We defined spectroscopic rapid rotators in this exercise as 
stars having \vsini{} above the detection threshold of the APOGEE 
spectrograph. As mentioned in 
Section~\ref{sec:vsini_check}, APOGEE \vsini{} measurements have not been fully
characterized, and the detection limit could reasonably be placed anywhere 
between 6-12 \kms{}. We repeated the analysis over multiple detection 
thresholds to ensure the robustness of our results.

In order to obtain a corresponding photometric rapid rotator fraction, we 
transformed the \vsini{} threshold into period space. The relationship between 
rotational period and equatorial velocity is given by the relation
\begin{displaymath}
    v_{eq} = \frac{2 \pi R}{P} 
\end{displaymath}
where \(R\) is the equatorial radius and \(P\) is the rotation period. As we
noted in Section~\ref{sec:radii}, there is a relatively clean and narrow
relationship between radius and \Teff{} on the main sequence. When transforming
between velocity and period-space, we used the DSEP-derived radius for each
individual object.

Due to the \(\sin i\) ambiguity, there isn't a simple one-to-one correlation
between a \vsini{} cut and the corresponding period cut given a stellar radius.
Many rapid rotators in period-space scatter into being non-detections in
velocity space after convolution with the inclination distribution. However,
this distribution can be statistically characterized for randomly
aligned systems. Since our underlying sample was chosen based on
\Teff{} and \logg{}, we don't expect a bias in inclination.
Therefore, we perform a statistical correction of \(\pi/4=0.785\)\footnote{We 
  note that the presence of starspot variability biases the
    inclination distribution against face-on alignments. Measurements
    in clusters indicate that spots are not seen for inclinations where 
    \(\sin i < 1/2\) \citep{Jackson10}, whose exclusion only changes the 
correction factor to 0.740} 
to account for photometric rapid rotators scattering to low
\vsini{}. 

\begin{figure}[htb]
    \centering
    \includegraphics[width=0.8\textwidth]{detection_fraction}
    \caption{Rapid rotator fraction as a function of \vsini{} detection limit.
    The solid black line connects spectroscopic rapid rotator fractions for 
    different \vsini{} thresholds. Error bars represent 1-\(\sigma\) binomial 
    confidence levels. The dashed black line marks the inclination-corrected 
    fraction of targets with \citet{McQuillan14} periods short enough to be 
    detectable over the given \vsini{} threshold. Also plotted is the 
\citet{McQuillan14} nondetection sample (red).\label{fig:detection_fraction}}
\end{figure}

We validated this exercise by comparing the photometric and spectroscopic
rapid rotator fractions in the asteroseismic sample. In order to have the best
overlap sample, we used \vsini{} from \citet{Bruntt12} and rotation periods
from \citet{Garcia14}, whose sample for studying stellar rotation specifically
targeted asteroseismic dwarfs and subgiants. We found that the rapid rotator
fraction as measured spectroscopically and photometrically agreed to within
1-\(\sigma\) at all \vsini{} thresholds. 

The analysis for the cool dwarf sample is shown in 
\cref{fig:detection_fraction}. This time, the \vsini{} measurements
are from APOGEE and the rotation periods are from \citet{McQuillan14}. Error
bars were chosen to be binomial 1-\(\sigma\) confidence intervals based on
selecting the number of rapid rotators from the full sample. Unlike  
the case with the asteroseismic targets, the fraction of spectroscopic rapid
rotators is greater than the fraction of photometric rapid
rotators at a highly-significant level, which is exacerbated afer taking the
inclination correction into account.

We also checked the spectroscopic rapid rotator fraction of the 
\citet{McQuillan14} nondetections. We found that the spectroscopic rapid 
rotator fraction is significantly
lower in the nondetections than the overlap sample, but inconsistent with zero. 
The fact that the rapid rotator fraction is lower than the detection sample in
\citet{McQuillan14} is not surprising because the former is expected to be 
comprised of slow-rotating inactive stars and high-inclination spotted stars, 
both of which are expected to yield low \vsini{}. 

The finding that 
the nondetection sample contains a relatively large numbers of rapid rotators 
is surprising. Previous studies have found an inverse correlation between 
photometric variability and Rossby number (defined as \(R_o = P / \tau_{cz}\), where \(P\)
    is the rotation period and \(\tau_{cz}\) is the color-dependent 
convective overturn timescale) \citep{Messina01,Hartman09}. The existence of 
stars with low Rossby number and low photometric variability may either 
indicate that an additional factor determines the photometric variability 
aside from the Rossby number, or that \citet{McQuillan14} is not detecting 
rapid rotation in genuine, spotted rapid rotators.



\section{Results}
\label{sec:results}

After selecting dwarfs with detectable rotation in APOGEE,
visually inspecting for DLSBs, and cross-matching with \citet{McQuillan14}, we 
end up with 26 targets with confident \vsini{}, and 26 targets with marginal
\vsini{} where we can individually compare two vsini and period. 
A point-by-point
comparison for these targets is shown in the bottom panels of
\cref{fig:rot}. The comparison between \vsini{} and equatorial velocity shows
that the discrepancy found in Section~\ref{sec:fraction} is not due to slight
but coherent model errors, or a few discrepant points. Unlike with the
asteroseismic sample in Section~\ref{sec:astero}, all but one of the targets
lie in teh firbidden \(\sin i > 1\) region affecting both the confident and
marginal detections. This is indicative of a systematic
bias where periods are consistently (and sometimes drastically) too slow to be
consistent with \vsini{}.

The observation that this discrepancy exists for cool dwarfs and not
asteroseismic stars is also a significant clue. It indicates that the
discrepancy is only apparent for certain parts of the HR diagram. The main
differences between the two are the temperatures, with asteroseismic targets
being significantly warmer than our cool sample, and evolutionary state, where
asteroseismic targets are subgiants and the cool sample consists of dwarfs.
Better radii will be needed to extend this analysis to additional targets.

The discrepancy between spectroscopic and 
photometric rapid rotation fractions can be seen in other published work as well. 
\citet{Nielsen13} compared their measurements of \Kepler{} rotation periods to
an amalgamation of \vsini{} from various clusters. They found that the
\vsini{} distribution was consistent with the period distribution for spectral
types F--K, and then the periods dropped below the \vsini{} measurements for
K--M (top panel of their Fig. 2). They ascribe the discrepancy to the cool
stars as a shift in the \Kepler{} age distribution between the hot and cool
stars; however, they claim to reproduce rotational periods with previous work. 
Additional confirmation occured through the APOGEE M-dwarf rotation
program \citep{Gilhool18}, which surveyed a color- and proper-motion-selected 
sample of nearby M-dwarfs. Unlike in this work, the \vsini{} were measured 
directly through the spectra as opposed to through ASPCAP\@. For the \Teff{} range that
overlaps with \citet{McQuillan14} \(4000 \textrm{ K } > \Teff > 3200\) K, they 
find that 5.8\% of their targets have \(\vsini > 8 \kms\). This is 
significantly greater than our photometric rapid rotator fraction at 8 \kms{} 
of 2.5\%. 

While spectroscopic rapid rotation didn't necessarily imply photometric rapid
rotation, we find that photometric rapid rotation is reflected in spectroscopy.
This can be seen in \cref{fig:

The most obvious systematic error that could cause this problem is
misclassification of subgiants. Subgiant contamination would explain 
results where the measured vsini would be
larger than the inferred equatorial velocity, because the radius is
underestimated. Despite predicting from the asteroseismic sample that
contamination should be low, it may be possible that asteroseismic subgiants
are not representative of the \Kepler{} subgiant population. Even if this
unlikely possibility were true, it would only be able to explain those rapid
rotators with inferred radii up to four solar radii, not all of them.
Contamination from larger red giants would be significantly less likely, and
detection of rapid rotation on red giants would be a significant finding
\citep{Tayar15}.

Confusion in the \Kepler{} pixels is also a potential explanation for the
discrepancy. Because the \Kepler{} pixels are relatively large (4\arcsec), a 
bright, hot primary in the APOGEE fiber with a fainter, spottier companion 
contributing the photometric modulation in the
\Kepler{} light curve would explain the observed contradiction between
\vsini{} and rotation period. Despite this setup is unlikely, because for 
stars cooler than the Kraft break, rotation and activity are highly 
correlated, we check for bright companions
within the \Kepler{} pixel using the \Kepler{} field ihigh-resolution UKIRT
J-band introduced in Section~\ref{sec:astero}. We did not find a correlation 
between that the presence of a bright companion and the magnitude of the 
offset, indicating that confusion is likely not the main source of the 
problem.

One other potential explanation for the mismatch between the photometric
rotation periods and \vsini{} is the presence of SB2s with a special geometry
where the two components are offset enough that their spectral features
overlap, yet not offset enough for the two components to be visually
distinguished.  To determine the likelihood of this scenario, we check whether
these targets show signs of composite spectra \citep{ElBadry18}. We 
find that 21/23 of the confident detection and 17/23 of the marginal detections
in the overlap sample shows signs of composite spectra based on their
data-driven classification. However, the Pleiades sample from
\citet{Stauffer87} was a color-selected sample, and did not show widespread
systematics in \vsini{} \citep{Jackson10}.

One last potential explanation is that the periodograms used in large-scale
analyses are biased against rapid rotators. While it's unlikely there there is
a hard-limit intrinsic to the period-finding algorithms \citep{Aigrain15}, it
may be possible that in the case of real-nonidealized data, with blending of
multiple variability sources, the algorithms tend to favor long periods over
short ones. Simple problems with aliasing cannot be behind this discrepancy
because in most cases, discrepancies are much larger than a factor of two.\ to
better understand potential biases in the two samples of rapid rotators, we
inverted our sample by restricting ourselves to photometric rapid rotators with
\(P_{rot} < 3\) day, and calculated the fraction of photometric rapid rotators
which are also spectroscopic rapid rotators. We found that 7/8 objects
with \(P < 3\) day are spectroscopic rapid rotators. So it appears that the
\citet{McQuillan14} periods are incomplete for the rapid rotators. This is in
agreement with the presence of spectroscopic rapid rotators in the
\citet{McQuillan14} nondetection sample.

\section{Conclusion}
\label{sec:conclusions}

We analyzed the APOGEE sample of 400 cool in the \Kepler{} field with 
rotation periods in \citet{McQuillan14} using DSEP-derived stellar radii. We 
found a substantial discrepancy between the two, with the spectroscopic rapid
rotator rate being 4-5 times the photometric rapid rotator rate. 

A point-by-point analysis of the 52 rapid rotators in our sample revealed that
rotational periods for the entire sample are universally too long to match the 
\vsini{}. This finding implies severe systematic uncertainties in either the
measurement of \vsini{}, rotation period, or in stellar classification. We
suggest several possible ways to investigate the underlying causes.

If the discrepancy lies with either concealed binarity or subgiant
misclassification, the upcoming \Gaia{} data release would be able to
test whether this is the case. Parallax-derived radii should be able to
easily distinguish any misclassified subgiants in the APOGEE spectra.
Additionally, binaries where the companion is bright enough to
substantially alter the measured \vsini{} would also be bright enough to
show a significant luminosity excess given the temperature of the star.
Therefore, rapid rotator false positives should appear anomalous in
luminosity compared to main-sequence isochrones.

Systematic uncertainties in period determinations will be significantly more
difficult to investigate. One start may be to perform a targeted period search
at short periods for the rapid rotators where the blind search returned a long
period. The presence of a short-period signal would confirm that the
spectroscopic rapid rotators are physical. Another direction would be to use an
algorithm which can successfully identify multiple periods, as has been used
for the K2 cluster campaign \citep{Rebull16,Rebull17}. If multiple signals are
present in a light curve, these methods may be able to pick both as opposed to
favoring one over the other.

Radii from the upcoming \Gaia{} Data Release 2 parallaxes will also enable 
follow-up studies to see which other portions of phase-space are affected by 
this mismatch between rotation periods and \vsini{}. Currently we only know 
that the methods yield consistent results for young clusters and 
asteroseismic targets, and yield inconsistent results for cool field dwarfs. If
this inconsistency continues to hold for all unevolved dwarfs regardless of
temperature, this would have significant consequences for the application of
gyrochronology to the \Kepler{} field, and could potentially reveal new
behavior in stellar activity at old ages \citep{vanSaders16}.

Regardless of the source of the inconsistency, we caution readers about 
interpreting large datasets of rotation in the \Kepler{} field without 
independent confirmation until the source of this inconsistency is found.


\bibliographystyle{aasjournal}
\bibliography{references}

\end{document}




