% aastex6.1 will be released as part of TeXLive 2017 around June 1. 
\documentclass[manuscript]{aastex6}

% Make the underscore behave correctly.
\usepackage[T1]{fontenc}
% Import images
\usepackage{graphicx}
% Used for consistent handling of Figures/Sections/Tables
\usepackage[capitalize]{cleveref}
% Format URLs correctly
\usepackage{url}
\usepackage{amsmath}
\usepackage[T1]{fontenc}
\usepackage{aecompl}
\usepackage{color}

% Want to have a separate storage folder for my figures.
\graphicspath{{./fig/}}

% Settings for cleveref to make Figures and Tables more appropriately
% defined.
% I want figure to be abbreviated.
\crefname{figure}{Fig.}{Figs.}

\newcommand{\vsini}{\ensuremath{v \sin i}}
\newcommand{\Kepler}{\mbox{\textit{Kepler}}}
\newcommand{\Gaia}{\mbox{\textit{Gaia}}}
\newcommand{\McQuillan}{\citep{McQuillan14}}
\newcommand{\Teff}{\ensuremath{T_{eff}}}
\newcommand{\logg}{\ensuremath{\log(g)}}
\newcommand{\kms}{\textrm{~km~s}\ensuremath{^{-1}}}
\newcommand{\MK}{\ensuremath{M_K}}

% Targeting flags
\newcommand{\STARBAD}{\texttt{STAR\_BAD}}
\newcommand{\STARWARN}{\texttt{STAR\_WARN}}
\newcommand{\VSINIWARN}{\texttt{VSINI\_WARN}}
\newcommand{\APOGEESEISMO}{\texttt{APOGEE2\_SEISMO\_DWARF}}

\newcommand{\gvs}{\authorcomment1}

\shorttitle{Comparison of Rotation Measurements in \Kepler{}}
\shortauthors{Simonian et al.}
\begin{document}

\title{Rapid Rotation in the \Kepler{} Field: Not a Single Star
Phenomenon}
\author{Gregory V. Simonian; Marc H. Pinsonneault; Donald M. Terndrup}
\affil{Department of Astronomy, The Ohio State University}
\affil{140 West 18th Avenue, Columbus, OH 43210}
\email{simonian@astronomy.ohio-state.edu}

\begin{abstract}
    Tens of thousands of rotation periods have been measured in the
    \Kepler{} fields, including a substantial fraction of rapid rotators. We 
    use \Gaia{} parallaxes to distinguish photometric binaries (PBs) from
    single stars on the unevolved lower main sequence, and compare their
    distribution of rotation properties to those of single stars both with and 
    without APOGEE spectroscopic characterization. Virtually all stars 
    with rotation periods less than 7 days are PBs, and the fraction of stars
    with \(P < 7\) days (0.029\% \pm 0.002\%) \gvs{(For lower limit of 1 day,
    fraction is 0.019\% \pm 0.001\%)} is also consistent with the
    population of short period binaries inferred by \Kepler{} eclipsing binary
    data after correcting for inclination \gvs{Insert predicted fraction.}
    This indicates that the rapid rotators are dominated by 
    tidally-synchronized binaries rather than single-stars obeying traditional 
    angular momentum evolution. As a result, caution should be taken before 
    interpreting rapid rotation in the \Kepler{} field as a signature of 
    youth. This new sample of 168 candidate tidally-synchronized binaries 
    should also be characterized to understand tidal processes in stars.
\end{abstract}
\keywords{stars: rotation}

\section{Introduction}

Rotation is a fundamental property of stars; it impacts their lifetimes, can
induce mixing, and can serve as a population diagnostic. Stars are also born 
with a wide range of rotation rates \citep{Attridge92, Herbst00, Henderson12}.
With a strong dependence on stellar mass. The observed rotation rates are also
strikingly different in high and low mass stars. \citet{Kraft67} noted a 
dichotomy in rotation rates among field stars hotter than 6200 K rotating 
with a wide range of \vsini, and those cooler than 6200 K, including the Sun, 
rotation with largely undetectable \vsini. This abrupt transition is thought to
be due to the onset of convective envelopes for lower-mass stars, which 
enables efficient angular momentum loss through magnetized winds 
\citep{Parker58,Weber67}.

The strong dependence of angular momentum loss on rotation \citep{Kawaler88}
leads to the rapid convergence of a wide range of rotation rates to a unique
mass-dependent value; for solar-type stars convergence is nearly complete by
0.5 Gyr \citep{Pinsonneault89}. Rotation is therefore an age indicator for 
stellar populations \citep{Skumanich72}; and correlations between rotation and
age been used to derive empirical relationships for field populations, 
sometimes referred to as gyrochronology \citep{Barnes07, Mamajek08, Meibom09}.  
A full theoretical treatment of gyrochronology requires evolutionary stellar 
models that include a variety of physical effects (see \citet{Gallet13} for an
overview). Models of angular momentum evolution have usually been calibrated 
using rotation periods in star clusters, which have known ages and are easily 
characterized \citep{Krishnamurthi97, Gallet13, Somers17}. The calibrating 
clusters are typically young (\(< 1\) Gyr), with the behavior at old ages anchored by the Sun. 

The \Kepler{} satellite \citep{Borucki10,Koch10}, which observed a single field
for over four years, has revolutionized our ability to measure stellar 
rotation in old field populations. Large-scale analyses have extracted 
rotation periods from the light curves of tens of thousands of field stars
\citep{Nielsen13, Reinhold13, Garcia14, McQuillan14}. These rotation periods, 
along with a field gyrochronology, are expected to provide insights on the age 
distribution of the field, and thus provide more accurate representation of 
transiting exoplanets.

The new observations of old field stars enabled by \Kepler{} have challenged 
our theories of angular momentum evolution on multiple fronts. Existing 
gyrochronological models anchored at the Sun have not been able to explain
the relatively rapid rotation of old field stars with ages inferred from
asteroseismic data \citep{Angus15, VanSaders16}. The \Kepler{} population at 
long periods shows a sharp drop-off which is not predicted by stellar models 
\citep{VanSaders18}.  Nearby K- and M-dwarfs show a bimodality in the period 
distribution which has no clear explanation in stellar physics 
\citep{Davenport18}. There is also a modest, but real, population of rapid 
rotators that would be interpreted as young if they were single main sequence 
stars.

The \Kepler{} field contains a mixture of stellar populations, not all of which
have been calibrated in clusters. Subgiants, while being rare in the clusters used to calibrate gyrochronology, 
make up 20\% of \Kepler{} targets \citep{Gaidos13, Berger18b}. Since the 
subgiant population contains stars expanding off the main sequence from 
above the Kraft break, they obey a different gyrochronology relation than that 
calibrated for cluster dwarfs \citep{vanSaders13}. 

Another population which has not been calibrated in clusters is
tidally-synchronized binaries. These binaries have orbital periods short enough
that tidal interactions force the orbital period to match the rotation period.
Existing models of tidal theory \citep{Zahn77} predict a strong dependence of 
the synchronization timescale on period, leading to rapid transition between
synchronized and unsynchronized systems. Current observations place the
synchronization period between 5--9 days \citep{Raghavan10}. Understanding the 
age distribution of the \Kepler{} field at the young end will require 
characterizing the background population of tidally-synchronized binaries.

Tidally-synchronized binaries themselves are in the center of interesting
dynamical phenomena. These systems are thought to be dynamically-formed 
through three-body interactions \citep{Tokovinin06, Fabrycky07}. Once formed, they can experience significant
evolution in orbital period due to angular momentum loss \citep{Andronov06}.
Depending on the rate of angular momentum loss, they can merge to form blue 
stragglers or pre-CV systems.

Tidally-synchronized binaries have been difficult to study because they are 
intrinsically rare and usually require significant spectroscopic resources 
to discover and characterize \citep{Mathieu90, Raghavan10, Geller15}. 
Characterizing the close, tidally-synchronized binary population will enable 
us to predict the prevalence of mergers in the field as well as compare merger
rates to those observed in clusters.

The availability of \Gaia{} DR2 parallaxes \citep{Gaia18,Lindegren18} has
enabled us to better characterize sources of rapid rotation beyond that
available through photometry and spectroscopy. Young single stars would be
observed as rapid rotators within the field main sequence. Subgiants would be
substantially more luminous than the main sequence due to the expansion of
their stellar envelopes. Lastly, binaries should be up to twice as luminous as
the main sequence, extending up to 0.75 mag above the single-star sequence.
While these populations overlap for hot stars, the subgiants and binaries
separate cleanly on the lower main sequence.

An additional constraint on the populations comes from the substantial sample
of spectroscopically-characterized dwarfs from the Apache Point Observatory
Galactic Evolution Experiment (APOGEE) \citep{Majewski17}. A total of 
approximately 5000 \gvs{Exact number} dwarfs and subgiants
have been observed in the \Kepler{} field with reliable effective temperature
and metallicities \citep{Holtzmann18}. This sample will be valuable for
understanding the importance of metallicity to identify populations of
binaries.

This paper will attempt to measure the prevalence of photometric binaries among
the rapid rotators in \citet{McQuillan14}. In Section~\ref{sec:data}, we will 
describe our sample, as well as the data used to select binaries from the 
sample. In Section~\ref{sec:analysis}, we demonstrate the regime where 
binaries can be successfully distinguished from subgiants with both the 
presence and absence of metallicity information. We'll also characterize the 
uncertainty in our measurements. Section~\ref{sec:results} illustrates our 
results of the prominence of binaries among the rapid rotators. Lastly, we 
discuss the implications of our results for the population of rapid rotators 
and lay out future avenues to explore.

\section{Catalog Data}
\label{sec:data}

The base sample of this study consists of \Kepler{} targets with photometric
period detections. All targets in the \Kepler{} field have stellar parameters
determined at the very least by KIC photometry \citep{Brown11}. The total
number of objects with period detections is 34,030.

We also examine a spectroscopically-characterized subsample from APOGEE\@. 
This sample has spectroscopically-determined temperatures and metallicities. 
The overlap between this sample and the number of objects inspected by 
\citet{McQuillan14} for rotational period modulation is 3,023.

To calculate vertical displacements above the main sequence, we use photometry
from 2MASS \citep{Skrutskie06}, parallaxes from Gaia DR2 \citep{Gaia18},
\Teff{} from \citet{Pinsonneault12}, and extinction values estimated from the 
\Kepler{} Stellar Parameter Catalog (KSPC) DR25 \citep{Huber14,Mathur17}. 
For the spectroscopic sample, \Teff{} and [Fe/H] are taken from APOGEE DR14 
\citep{Abolfathi18}. More detail on these catalogs is given below.

\subsection{Default Stellar Parameters}

Stellar parameters for the full \Kepler{} sample are available as part of the
\Kepler{} Stellar Parameter Catalog (KSPC) DR25 \citep{Mathur17}. The KSPC
compiles stellar parameters from the literature and simultaneously fits them
using DSEP \citep{Dotter08} stellar evolutionary tracks via the methodology of
\citet{Huber14}. Revised stellar parameters, such as \Teff, as well as inferred
quantities, such as distance and extinction, are output along with asymmetric
1-\(\sigma\) uncertainties. 

The sophisticated machinery of the KSPC imprints artifacts in the \Teff{}
distribution of cool dwarfs from attempting to stich together the temperature
scales of \citet{Pinsonneault12} and \citet{Dressing13}. Instead of using
\Teff{} from the KSPC, we therefore opted to use temperatures from 
\citet{Pinsonneault12} to ensure uniformly analyzed temperatures; this data is
close to the KSPC scale globally \citep{Huber17}.

For this paper, we also make use of extinctions from the KSPC\@. The extinction 
are inferred from the predicted absolute magnitude given KSPC parameters, and
a 3D extinction map \citep{Amores05}. The extinction is given as \(A_V\), which we
convert to other bands using the \citep{Cardelli89} extinction law. Distances
from the KSPC are used purely for illustrative purposes in
\cref{fig:mcquillan_selection,fig:apogee_selection}.

\subsection{Astrometric Data}

The ability to distinguish between different evolutionary states of stars is
enabled by parallaxes derived from the \Gaia{} mission's Data Release 2
\citep{Gaia18}. The \Gaia{} DR2 performed a fully consistent single-star
5-parameter (\(\alpha\), \(\delta\), \(\mu_\alpha\), \(\mu_\delta\), \(\pi\))
solution to 1.3 billion sources over 22 months of observations
\citep{Lindegren18}. For targets which could not be adequately fit with a
single-star 5-parameter solution, a 2-parameter (\(\alpha\), \(\delta\))
solution is performed instead \citep{Michalik15}. The criteria for a
2-parameter fall-back solution include targets with Gaia \(G > 21.0\), having
fewer than 6 visibility periods of observations, and an error ellipse larger
than a magnitude-dependent limit in its largest dimension \citep{Lindegren18}.

% APOGEE Targeted sample APOKASC (APOKASC + Cool Dwarf - faint dwarfs) - 7642
% Not bad APOGEE targeted sample - 7395
% Good APOGEE with good K detection - 7373
% Good APOGEE in Gaia - 6961

We use  cross-matched database of \citet{Berger18b} to match \Kepler{} targets
against \Gaia{} DR2 detections.
\citet{Berger18b} cross-matched the KSPC DR25 \citep{Mathur17}, with \Gaia{} 
DR2 using both position and G-band flux. Of the successfully cross-matched targets,
\citet{Berger18b} excluded objects with fractional parallax errors greater than
0.2, \(\Teff < 3000\) K, \(\logg < 0.1\), and with low quality 2MASS photometry. 
We converted parallax to distance modulus by using an
exponentially-decreasing volume density prior with a 1.35 kpc length scale
\citep{BailerJones15,Astraatmadja16}, and correcting for the zero-point offset
of 0.5 mas \citep{Zinn18}. \gvs{Should we include a table of this?} 

\subsection{Photometric Data}

In order to characterize our stellar sample, we chose to use absolute K-band
magnitude as a proxy for luminosity. K-band apparent magnitudes were measured
by the 2MASS survey \citep{Skrutskie06}, which has profile-fit photometry with 
uncertainties available for nearly the full sample. A small fraction of 
targets (1.5\%) of the \citet{McQuillan14} sample have photometry flagged due
to blending. To avoid contaminating the binary sequence with unrelated blends,
we exclude these targets.

We chose to use K-band magnitudes because the measurement uncertainties
associated with it are low. The typical K uncertainty for the
\citet{McQuillan14} sample is 0.03 mag. The typical extinction from the KSPC 
and its uncertainty for cool dwarfs is \(A_V = 0.33\), and \(\sigma_{A_V}=0.025\).
Applying the \citet{Cardelli89} relation that \(A_K/A_V = 0.114\), the effect
of extinction itself shrinks to the level of photometric errors. While we
include the extinctions to avoid biasing our sample, the use of K-band makes us
confident that our results are not significantly impacted by extinction.

Because the signature of binarity is substantially larger than the
uncertainties of hte measurements, we eschew detailed modeling of the
uncertainties \citep{Luri18} in favor of the simpler approach of adding the 
uncertainties in quadrature to obtain an absolute magnitude. Most of the 
sample is either strongly dominated by photometric errors or distance errors, 
and this treatment will do an appropriate job of approximating the 
uncertainties in the measured absolute magnitude.

\subsection{Rotation Data}

The rotation periods for our sample are derived from \citet{McQuillan14}, 
which is to-date the largest homogeneous analysis of rotational variability in
\Kepler{} stars. \citet{McQuillan14} selected their sample to have photometric
\(\Teff < 6500\) K \citep{Brown11,Dressing13}, and implemented the color-color
and \logg{} cuts advocated in \citet{Ciardi11}, ranging from \(\logg = 
3.5\) at 6000 K to \(\logg = 4.0\) at 4250 K. 
In addition to the temperature and gravity cuts, \citet{McQuillan14} also 
excluded known eclipsing binaries and plantary transit candidates. 

Rotational periods in \citet{McQuillan14} were measured by calculating the 
autocorrelation function and fitting the location of multiple peaks. In lieu 
of visual verification of the autocorrelation function, \citet{McQuillan14}
performs two automated tests to distinguish physical periodicity from
instrumental artifacts or other sources of variability. First, the 
periodicity must be consistent in different segments of the light curve.
Secondly, the height of the first peak must be larger than a temperature- and
period-dependent threshold. In order to reduce contamination from pulsators, 
\citet{McQuillan14} only considered periods between 0.2--70 days.

\begin{figure}[htb]
    \centering
    \plotone{mcquillan_selection}
    \caption{\emph{Left:} \Teff-\(M_K\) density plot of the sample of
        \citet{McQuillan14} period detections. A binary sequence is clearly 
        visible on the lower main sequence. \emph{Right:} The variation in 
        the \citet{McQuillan14} period detection fraction across the 
        \Teff-\(M_K\) diagram.}\label{fig:mcquillan_selection}
\end{figure}

The distribution of \citet{McQuillan14} targets is shown in a \Teff-\(M_K\)
diagram in \cref{fig:mcquillan_selection}, using pre- and post-\Gaia{}
distances. One of the new results enabled by \Gaia{} is that a binary sequence,
located above the main sequence, is clearly seen. Additionally, many of the
period detections on the giant branch have gone away, which is to be expected
since red giants are expected to be slowly rotating. There are also a
collection of objects several magnitudes brighter than the main sequence, but
cooler than the red giant branch. We do not postulate on the nature of these
objects, but note they may be worth following-up on.

The period detection fraction also shows encouraging trends in
\cref{fig:mcquillan_selection}. Cooler stars tend
to be more spotted, indicating an increase in spottiness on the lower main
sequence. An interesting feature we note is that the detection fraction shows a
bifurcation between the main sequence and binary sequence hotter than about
4500 K. While we do not delve in the feature, we point out its existence, and
that it is consistent with evolution of activity on the single-star sequence, 
with more evolved stars having fewer spots detections.

\subsection{Spectroscopic Parameters}

We draw our spectroscopic sample from the Data Release 14 \citep{Abolfathi18}
of the APOGEE survey \citep{Majewski17}. The workhorse behind the survey is the
APOGEE spectrograph, a high-resolution (\(R \sim 22,000\)) multi-fiber
near-infrared spectrograph \citep{Wilson10}, mounted on the SDSS 2.5-meter
telescope at Apache Point Observatory \citep{Gunn06}.

Reduction of the APOGEE data takes place in three main stages. First,
individual visit spectra are reduced, including detector calibration, bad pixel
masking, wavelength calibration, sky substraction, and determination of
individual radial velocities through cross-correlation of template spectra.
Second, the individual spectra are combined by correcting for radial velocity
differences between the exposures, either by cross-correlating with each other,
or with template spectra, whichever leads to a smaller scatter
\citep{Holtzman18}. The full process is detailed in \citet{Nidever15}.

The final step is the extraction of stellar parameters and chemical abundances
by the APOGEE Stellar Parameter and Chemical Abundances Pipeline (ASPCAP)
\citep{GarciaPerez16}. ASPCAP measures stellar parameters by performing a
chi-squared minimization \citep{AllendePrieto06} over a 6-dimensional space of
\Teff, \logg, [M/H], [\(\alpha\)/M], \vsini, and microturbulent velocity.

After determination of the stellar parameters, they are calibrated to the
photometric system of \citet{GonzalezHernandez09} using a metallicity-dependent
offset. After calibration, the scatter between the calibrated and photometric
temperatures  were found to be a strong function of \Teff{}, ranging from 130 
K at 5500 K down to 85 K at 4000 K \citep{Holtzman18} for solar
metallicity objects observed at a signal-to-noise ratio of 100. 

We inspected the fits to a representative set of spectra for targets flagged 
by the ASPCAP pipeline as having potential quality problems. A small fraction
(2.5\%) of targets have the \STARBAD{} quality flag enabled. Visual inspection
of a representative sample of spectra indicated that the flag was largely the
result of poor subtraction or normalization of the spectrum in the pipeline. Or
due to poor model fits on the cool (\(\Teff < 4250 K\)) end. We did not find
that the presence of a secondary triggered this flag, so we exclude these
targets from further analysis without worrying about biasing our sample. A 
substantially larger fraction of targets (14\%) have the \STARWARN{} quality 
flag enabled. Visual inspection of the spectra indicated that the fit 
reasonably resembled the target spectra. Because we did not find any obvious 
problems with stellar parameter determinations, we included these objects in 
our sample.

\begin{figure}[htb]
    \centering
    \plotone{apogee_selection}
    \caption{\emph{Left:} \Teff-\(M_K\) diagram for the APOGEE observations of
        \Kepler{} targets. Asteroseismic targets are shown as black dots. The
        cool dwarf sample is shown as red dots. The light blue dots indicate 
        eclipsing binary targets and purple dots are Kepler Objects of 
        Interest. A binary sequence is clearly visible on the lower main
        sequence. A representative error bar for the cool dwarf sample is also
        shown. \emph{Right:} A density plot of the full APOGEE sample. To
        preserve the dynamic range of the dwarf sequence, the red clump was 
        allowed to saturate. The size of the bins approximates the size of the
    error bars.}\label{fig:apogee_selection}
\end{figure}

\Cref{fig:apogee_selection} shows all \Kepler{} targets with observed
spectroscopic temperatures in the APOGEE DR14, color-coded by observing
program. One of the driving efforts for \Kepler{} field observations has been
the spectroscopic characterization of asteroseismic targets, shown as small
green, blue and purple dots \citep{Zasowski17,Pinsonneault18}. This sample is 
substantially biased toward giants and subgiants. A complementary program 
observed the full \Kepler{} sample cooler than 5500 K and brighter than 
\(H < 11\) \gvs{Exact number} 1,200 targets, shown as small red dots. Another \(\sim 1400\) observations have been
made to characterize \Kepler{} planet hosts with \(H < 14\), shown as purple
diamonds. Lastly around 200 eclipsing binaries \citep{Prsa11,Slawson11} were
observed, and are shown as light blue circles.

\section{Data Analysis}
\label{sec:analysis}

The crucial quantity we measure for these samples is the vertical displacement
above the main sequence. Vertical displacement has been calculated many times
in clusters \citep{Mermilliod92}, where the main sequence is described by a
single-age stellar population with a uniform metallicity. Measuring the
vertical displacement for a field population is more challenging, because the
field consists of a heterogeneous sample of ages and metallicities. In order to
successfully measure a single-star sequence we must either control for or
characterize these additional factors. 

The vertical displacement consists of two parts: the measured K-band
luminosity of a star, and the inferred luminosity of a single star with the
same temperature and metallicity as the original one at a reference age. The 
former is well-constrained and easily calculated quantity given the 2MASS 
photometry and the \Gaia{} parallax. The latter requires a single-star model.

We use the MIST \citep{Dotter16,Choi16} isochrones to model the 
single-star lower main sequence. For a full description of the MIST 
isochrones, we refer the reader to the source papers, as well as the MESA 
instrument papers \citep{Paxton11, Paxton13, Paxton15}, which is the stellar 
model on which the isochrones are based. The MIST isochrones at solar
metallicity are shown in \cref{fig:mcquillan_selection} and behave as expected for a field
star population.

In order to characterize the vertical displacement, we select 1 Gyr isochrones 
because they span a wide range of temperatures without imposing
structure from the turnoff. This is necessary for us to draw a threshold where 
age effects become unimportant compared to binarity. Because we will 
ultimately be working on the lower main sequence, where age evolution is 
negligible, the choice of isochrone should not affect our result. 

\begin{figure}[htb]
    \centering
    \plotone{sample_dk}
    \caption{Temperature and K luminosity excess above a 1 Gyr MIST isochrone 
        matched to the APOGEE metallicity for the APOGEE-\Kepler{} sample. 
        In order to exclude evolved stars while preserving binaries and
        triples, we define the dwarf sample as targets less than 1.3 magnitudes
        more luminous than the metallicity-adjusted MIST isochrone. The dwarf 
        sample is denoted as red points. A representative error bar for the 
        full APOGEE dwarf sample is shown in the lower left 
    corner.}\label{fig:sample_dk}
\end{figure}

\begin{figure}[htb]
    \centering
    \plotone{ages}
    \caption{Age-imposed behavior for the APOGEE dwarf sample. The purple
        line denotes the predicted luminosity excess of a 9 Gyr stellar 
        population. The dashed line represents the limit for photometric
        binaries. The 25th percentile of the dwarfs in 15 temperature bins is
        shown in aquamarine. We draw a temperature threshold for the unevolved
        lower main sequence at 5250 K. A representative error bar
derived assuming Gaussian uncertainties is shown on the right.}
    \label{fig:ages}
\end{figure}

The vertical displacement of the full APOGEE sample is shown in 
\cref{fig:sample_dk}. For every star, the displacement is calculated by
subtracting the luminosity of the 1 Gyr isochrone at that star's metallicity.
The lower main sequence has now been flattened and goes through the x-axis. The
upper main sequence lies above the isochrone, which is to be expected because
K-band luminosity is substantially more sensitive to age for more massive
stars. In order to separate the cool dwarfs from the base of the red giant 
branch, we draw a limit of 1.3 magnitudes more luminous. This limit
cleanly excludes red giant branch stars while providing generous allowance for
binaries and potential triple systems.

For hot stars, the increase in luminosity due to age is degenerate with
increased luminosity due to a binary companion. In order to avoid age effects,
we choose to restrict ourselves to the unevolved lower main sequence, where age
effects are small. In order to characterize the effects of age, we compare the
behavior of the luminosity excesses with predictions from MIST stellar models.
This comparison is shown in \cref{fig:ages}. 

In order to characterize the behavior of the main sequence, we choose to use
the 25th percentile of the luminosity excess in each bin. Each temperature bin
contains a population of both binaries and single stars, with binaries having
substantially higher luminosity excesses. The 25th percentile statistic
characterizes the main sequence in a way that is robust to the population of
binaries. This method has been used in clusters \citep{Mermilliod92} \gvs{Is
this reference correct?} to define a main sequence, and does so exactly in a
population which is 50\% binaries and 50\% single stars. Even large departures
in the binary fraction should generally only affect the zero-point, but not 
the shape of the main sequence.

The main sequence is generally flat at temperatures cooler than 5000 K with
an increase in luminosity of less than 0.1 mag cooler than 5200 K. A flat main
sequence is needed to characterize residual trends in \Teff{}. However, when it
comes to classifying binaries, slight age evolution will not severely impact
the results, and given that the number of stars increases dramatically between
5000 K and 5500 K, the benefit of increased numbers outweighs the slight
preference for hotter stars.  The trends seen in the data roughly match those 
predicted by the MIST isochrones. The main sequence is roughly flat cooler 
than 5000 K, and begins to increase substantially by 5200 K.

\subsection{\Kepler{} Dwarf Metallicity}

For the full \citet{McQuillan14} sample, where individual metallicities are not
available, we want to select an isochrone that is representative of the full 
sample. We do this by assuming that metallicity distribution of the 
spectroscopic targets is representative of the full \Kepler{} field. 

\begin{figure}[htb]
    \centering
    \plotone{metallicity}
    \caption{Metallicity distribution of the \Kepler{} cool dwarfs, selected as
        having \(4000 \textrm{ K } \le \Teff \le 5000 \textrm{ K }\) and having
        a luminosity less than 1.3 mag above the single-star sequence. The 
        median metallicity of the cool dwarf sample is [Fe/H]=0.08.}
    \label{fig:metallicity}
\end{figure}

Because we will mostly be concerned with the cool dwarfs, we check the
metallicity distribution of dwarfs cooler than 5500 K, based on 
\cref{fig:sample_dk}. The metallicity distribution is shown in
\cref{fig:metallicity}. The median metallicity is 0.08 with a 1-\(\sigma\)
confidence interval ranging from -0.14--0.23. A slightly super-solar
metallicity distribution is consistent with that observed for planet hosts by 
the California \Kepler{} survey \citep{Petigura17}.  We therefore treat the \([Fe/H] =
0.08\) isochrone as the representative one where metallicity is unknown, and
note that the majority of the sample is around solar metallicity.

\subsection{Corrections for Spectroscopic Sample}

Because our analysis only depends on the vertical displacement above the main
sequence, and not absolute position, we correct separately for trends in the
main sequence left over after isochrone subtraction. The NIR behavior of MIST 
isochrones have not been comprehensively tested at a wide range of 
metallicities \citep{Choi16}.

The first trend we want to understand is at which temperature threshold do age
effects start to become important. Because the dwarf sample consists of a
substantial fraction of photometric binaries, any treatment of the main
sequence would have to be robust to outliers. In order to characterize the main
sequence, we choose to use the 75th percentile, as was done in
\citep{Mermilliod92}, where they assumed a population which was 50\% binary and
50\% single. 

The trend in the 75th percentile is shown in \cref{fig:sample_dk}. Based on 
this figure, the main sequence is flat cooler than \(\Teff < 5000 K\); at
hotter temperatures, it trends up due to the effects of age. As a result, we 
use stars cooler than 5000 K to understand the behavior of the main sequence 
when age effects are negligible.

\begin{figure}[htb]
    \centering
    \plotone{metallicity_correction}
    \caption{\emph{Top:} Residual trends in the MIST metallicity-corrected luminosity 
    excess over metallicity. A quadratic trend remains over metallicity. The
    trend is characterized by splitting the sample into 5 bins with equal
    numbers of points.  The 25th percentile of points denote the y-value of the
    bin while the mean temperature red points mark the bins used to fit the trend. \emph{Bottom:} Residuals after
correcting for the quadratic trend. The median absolute deviation of the top 50
percentiles is 0.047 mag.}
    \label{fig:met_trend}
\end{figure}

We find and correct for a quadratic trend in the vertical displacement of the
main sequence over metallicity. The trend is shown in \cref{fig:met_trend}.
The trend is set by 5 bins which each have equal numbers of stars. The
temperature of the bin is the mean temperature of all stars in that bin; the
luminosity excess is the 75th percentile of all stars in that bin. We then fit
a quadratic polynomial to those five bins. We quantify the goodness-of-fit as
the median absolute deviation of the top 50th percentile of the luminosity
excesses. Five bins minimized the median absolute deviation, but changing the
number of bins had a very small effect on the quality of the fit for the bulk
of the sample. Only the sparsely populated low-metallicity regime (\([Fe/H] <
-0.5\)) was sensitive to the binning. Because this population is rare, and also
is expected to both be older and have different abundances, we just treat them
as outliers in our analysis.

\begin{figure}[htb]
    \centering
    \plotone{spec_teff_correction}
    \caption{\emph{Top}: Vertical displacement above a 1 Gyr,
        metallicity-adjusted MIST isochrone for the cool APOGEE sample. A 
        slight linear trend in \Teff{} was measured using five bins. 
        \emph{Bottom:} The residual sample after subtracting the linear 
    trend.}
    \label{fig:apogee_teff_trend}
\end{figure}

After correcting the trend in metallicity, we return to correct the trend in
\Teff{}, shown in \cref{fig:apogee_teff_trend}. This trend can be characterized
using a linear fit, done using the same bins as in metallicity space. The 
resulting points should now be reliably flat. The fact that a binary sequence 
can still be seen in the corrected dataset is encouraging that our 
corrections are meaningful.

\subsection{Corrections for Photometric Sample}

The photometric sample in \citep{McQuillan14} differs from the spectroscopic
sample in two main ways: the lack of metallicity diagnostics, and the necessity
of using photometric \Teff{} instead of a spectroscopic \Teff{}. Both of these
differences will increase the uncertainty of the inferred single-star K-band
luminosity, but not to the point where it affects our results.

\subsubsection{Metallicity Ignorance}

To understand the impact of lacking metallicities, we simulate metallicity
ignorance in our spectroscopic sample by subtracting a single [Fe/H]=0.08
isochrone from all targets. 

\subsubsection{Photometric \Teff{}}

\begin{figure}[htb]
    \centering
    \plotone{Teff_relation}
    \caption{Comparison between the APOGEE spectroscopic \Teff{} and the
        \citet{Pinsonneault12} \Teff{} for objects with \(4000 \textrm{ K } \le
    \textrm{ APOGEE \Teff{} } \le 5000 \textrm{ K }\).}\label{fig:teffdiff}
\end{figure}

We now evaluate the difference between using photometric and spectroscopic
\Teff{}. A direct comparison between the two temperature scales is shown in 
\cref{fig:teffdiff}. Overall, the two temperature scales are reasonably
correlated. There is a temperature scatter of 183 K between the two scales.
Taking into account the 100 K uncertainty in the APOGEE \Teff{} scale, this
implies a 150K uncertainty in the \citep{Pinsonneault12} temperatures. 

\begin{figure}[htb]
    \centering
    \plotone{phot_teff_correction}
    \caption{Same as \cref{fig:apogee_teff_trend}, except subtracting off a
    [Fe/H]=0.08 isochrone for all objects instead of a metallicity-adjusted
isochrone.}\label{fig:photuncor}
\end{figure}

Combining the two changes, subtracting only a single-metallicity isochrone and
using the photometric temperatures yields the points in \cref{fig:photuncor}.
This sample still has a trend with temperature, that we remove using a linear
fit.

\begin{figure}[htb]
    \centering
    \plotone{excess_hist}
    \caption{Histogram showing the distribution of vertical displacements for
        the cool APOGEE sample, both when using spectroscopic \Teff{}s and
    controlling for metallicity (blue), and using photometric \Teff{}s at 
fixed metallicity (red). Two-component Gaussians are fit to the samples. The
primary peak of the spectroscopic measurements has a width of 0.093 mag, while
the primary peak of the photometric measurements has a width of 0.114 mag.}
    \label{fig:histcompare}
\end{figure}

Now that we have corrected vertical displacements over the 4000--5000 K
temperature range, we can now compare the main-sequence scatter for the two
datasets. We do this by generating histograms for the spectroscopic and
photometric samples over vertical displacement, shown in
\cref{fig:histcompare}. We try to characterize the width of the two samples by
fitting a two-component Gaussian to each histogram, and taking the width of the
main-sequence Gaussian as the uncertainty in the vertical displacement in the
vertical displacement. Our fits indicate that the scatter for the spectroscopic
sample is 0.09 mag while the scatter for the photometric sample is 0.11 mag.
The small increase in uncertainty is likely due to the fact that the
metallicity distribution is heavily clustered around solar, with relatively few
outliers.

\section{Results}
\label{sec:results}


After successfully calculating the vertical displacement of the full sample, we
now attempt to delineate between single and binary stars. As noted before, the
1-\(\sigma\) uncertainties of the main sequence were 0.09 mag with known
metallicity, and 0.11 mag when metallicity was unknown. As a result, we set a
threshold between binaries and single stars of 0.2 mag. In order to avoid
age effects from biasing the results, we also only performed statistics in the
4000--5000 K range.

\begin{figure}[htb]
    \centering
    \plotone{apogee_rapid_excess}
    \caption{\emph{Top to Bottom:} Vertical displacement of cool APOGEE targets
        with \citet{McQuillan14} periods \(> 10\) days, between \(5--10\) days, 
        and between \(1--5\) days. Pink stars denote eclipsing binaries with
    orbital periods within the same ranges.}\label{fig:apogee_rapid_excess}
\end{figure}

The well-characterized APOGEE sample illustrates well the
correlation of rapid rotators and binaries, shown in
\cref{fig:apogee_rapid_excess}. In the period regime between 1--5 days, all of the
spectroscopic targets are rapid rotators. The fraction of single stars
increases as the period increases.

To quantify the significance the binary fraction in each bin, we divide the
sample into two bins: binaries and single stars, and use a binomial test to
calculate the chance of getting the observed binary fraction given the
background rate. For the background rate, we the binary fraction of all APOGEE 
targets with temperatures between 4000--5000 K, which is 0.41\%. 

In order to accurately characterize the tidally-synchronized systems, we
include the \Kepler{} eclipsing binaries. \citep{McQuillan14} excluded the
eclipsing binaries from their analysis to avoid contamination from the
eclipses. However, the eclipse probability for these short-period binaries can
be as high as 40\% \citep{Kirk16}. Ignoring this population would substantially
bias the underlying sample. Because we are only looking at systems with orbital
periods greater than 1 day, the eclipsing binary sample should only consist of
detached binaries, and not contact or ellipsoidal systems, which become
prevalent at periods less than a day. We assume that the eclipsing binaries are
synchronized, and that the rotation period is identical to the orbital period.
While this is certainly true at the low-period end, it will not be true at the
high-period end, where the components should rotate like single stars; 
however the number of eclipsing binaries is small at long periods, so we will
neglect this effect.

We find that the binary fraction with periods between 1--5 day is 0.93, which
is significantly different from the binary fraction of the full sample by a
p-value of less than \(10^4\). The p-value for the binary fraction for stars
with periods between 5--9 days is a marginal 0.01. For long periods b
9--11 days, the p-value indicates no preference for binaries over singles.

\begin{figure}[htb]
    \centering
    \plotone{full_mcquillan_rr_excess}
    \caption{\emph{Top to Bottom:} Vertical displacement of all cool
        \citet{McQuillan14} targets in period bins \(> 10\) days, \(5--10\)
        days, and \(1--5\) days. Pink stars denote eclipsing binaries with
    orbital periods within the same ranges.}\label{fig:mcq_rapid_excess}
\end{figure}

Unfortunately, the spectroscopic sample is too small to tell us much else about
the rapid rotators. To achieve this, we use the photometric sample, which is
substantially larger. The number of rapid rotators in each period bin is shown
in \cref{fig:mcq_rapid_excess}. With the substantially larger number of objects
in each bin, there is a distinct single-star sequence at
periods less than 9 days, which disappears at shorter periods. This
disappearance is because angular momentum loss leads to slow rotation in single
stars, and so there should not be any single-stars rotating at periods faster
than 5 days.

As a result, the remaining targets at short periods are likely to be
tidally-synchronized binaries of varying luminosity ratios. Binarity at various
luminosity excesses is supported by the presence of confirmed eclipsing
binaries low rotational periods. We confirm this by checking that the ratio of
eclipsing binaries to rapid rotators at low luminosity ratios is the same as
the ratio of eclipsing binaries to rapid rotators at high luminosity ratio for
periods between 1--5 days.

\begin{figure}[htb]
    \centering
    \plotone{eclipseprob}
    \caption{Comparison of rapid rotator period distribution and eclipsing
    binary. The period distribution of the \citep{McQuillan14} sample is shown
as a blue solid histogram, while the period distribution of the eclipsing binaries
is shown as the red histogram. The predicted EB distribution assuming the
\citep{McQuillan14} sample consists of binaries inclined enough to show
starspot modulation, but not inclined enough to eclipse. Error bars represent
1-\(\sigma\) Poisson confidence intervals.}\label{fig:eclipseprob}
\end{figure}

To check that the rapid rotators and eclipsing binaries are the same
population, we model the period distribution of eclipsing binaries using the 
period distribution of rapid rotators corrected by a geometric factor
reflecting the probability of eclipses \citep{Kirk16} and the probability of
viewing starspot modulation \citep{Jackson10}. The comparison between the two
is shown in \cref{fig:eclipseprob}. The two distribution match to within
1-\(\sigma\) at periods between 1--7 days, before slowly-rotating single-stars 
dominate the population.

\gvs{Maybe add some discussion about this\dots}
With the McQuillan sample, we can crudely characterize the tidally-sychronized
distribution by treating the short-period end as a combination of two
populations

\begin{figure}[htb]
    \centering
    \plotone{vsini_check}
    \caption{Luminosity excess and \vsini{} for the APOGEE spectroscopic
    sample. Slow rotators with \(P > 5\) day are the black points. Rapid
rotators with \(1 \textrm{ day } , P \le 5 \textrm{ day }\) are large purple
circles, and one very rapid rotator with \(P \le 1\) day is shown in blue. }
    \label{fig:vsini_check}
\end{figure}

In addition to the photometric periods, we checked the APOGEE \vsini{} values 
for the rapid rotators. A plot showing \vsini{} and luminosity excess for the
\citet{McQuillan14} rapid rotators is shown in \cref{fig:vsini_check}. We find 
that the rapid rotators overwhelmingly have \vsini{} larger than the rest of 
the sample. There are many objects with large \vsini{} that don't have 
correspondingly rapid rotation. It's likely those are spectral blends from 
SB2s which have not been removed from the sample. They would also show large 
photometric excesses due to the bright companion. \gvs{Maybe mark SB2s
separately. I don't want to go into the weeds with this.} The fact that all of 
the short-period systems show large \vsini{} supports the idea that the rapid 
rotation is intrinsic to the spectroscopic targets themselves and not a
failure mode of the photometric periods.

\section{Discussion}
\label{sec:discussion}

If the rapidly-rotating regime is dominated by tidally-synchronized binaries,
this will have significant consequences for interpreting gyrochronology of the
\Kepler{} field for young stars. As a result, questions of youth in the
\Kepler{} field will have to use different methodologies that don't rely on
rotation or activity, or the tidally-synchronized background will need to be
modeled. 

Aside from the question of age, the rapid rotators in the \Kepler field provide
a sizeable sample of tidally-synchronized binaries. A total of 168 systems for
potential follow-up and characterization. These systems are currently believed
to be driven to circular orbits by Kozai Oscillations and Tidal Friction
\citep{Fabrycky07}. The population is dynamic and can change drastically with
environment. A thorough study of field tidally-synchronized binaries will also
help understand the merger population in the field.

\begin{figure}[htb]
    \centering
    \plotone{ElBadry_Excess}
    \caption{The sample of APOGEE targets analyzed by \citet{ElBadry18b} for
    spectroscopic signs of a companion. Black points indicate the DR14 ASPCAP
values for the targets. Red points indicate the revised temperatures, and the
K-band luminosity excess derived from the revised temperature and metallicity
in \citet{ElBadry18b}.}
    \label{fig:elbadry_excess}
\end{figure}

One of the surprising results we noted was the abundance of systems with
luminosity excesses greater twice the single-star luminosity, even up to
greater than three times the single-star luminosity. One potential
explanation for these systems is that secondary causes the observed temperature
to be cooler than the actual primary temperature \citep{ElBadry18a}, causing the single-star
luminosity to be underestimated. As a test, we use temperatures from
two-component fits that were performed for a subset of the stars in our sample
with APOGEE DR13 spectra \citep{ElBadry18b}. The revised APOGEE temperatures are
shown in \cref{fig:elbadry_excess}.

As seen in the bulk binary sample, the effect of binarity is substantial. The
inferred primary temperature may be off by several hundred Kelvin at
intermediate values of the excess, leading to the K-band luminosity excess to
be overestimated by up to 0.3 mag. The difference in temperature is less severe
for targets with high luminosity excesses, which is to be expected because the
two stellar components have similar spectra. However, the effect is still
non-negligible.

\section{Conclusion}
\label{sec:conclusions}

We calculated K-band luminosity excesses above the main sequence for the full
\citet{McQuillan14} sample. As seen in both a spectroscopically-characterized
subset as well as in the general sample, rapid rotators with periods between
1--5 days show substantially larger luminosity excesses than the slower
rotators. When comparing with the population of EBs, the period distribution of
rapid rotators matches that predicted for non-eclipsing binary systems.

The high luminosity excess as well as the concordance with the eclipsing
binaries implies that the rapid rotators primarily consist of short
orbital-period, tidally-synchronized binaries, and not young
single-stars. Due to the substantial population of tidally-synchronized
binaries in the short-period regime, care needs to be taken in order to
interpret short rotational periods as being signatures of youth in the
\Kepler{} field.

\bibliographystyle{aasjournal}
\bibliography{references}

\end{document}




