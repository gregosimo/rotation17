% aastex6.1 will be released as part of TeXLive 2017 around June 1. 
\documentclass[manuscript]{aastex6}

% Make the underscore behave correctly.
\usepackage[T1]{fontenc}
% Import images
\usepackage{graphicx}
% Used for consistent handling of Figures/Sections/Tables
\usepackage[capitalize]{cleveref}
% Format URLs correctly
\usepackage{url}
\usepackage{amsmath}
\usepackage[T1]{fontenc}
\usepackage{aecompl}
\usepackage{color}

% Want to have a separate storage folder for my figures.
\graphicspath{{./fig/}}

% Settings for cleveref to make Figures and Tables more appropriately
% defined.
% I want figure to be abbreviated.
\crefname{figure}{Fig.}{Figs.}

\newcommand{\vsini}{\ensuremath{v \sin i}}
\newcommand{\Kepler}{\mbox{\textit{Kepler}}}
\newcommand{\Gaia}{\mbox{\textit{Gaia}}}
\newcommand{\McQuillan}{\citep{McQuillan14}}
\newcommand{\Teff}{\ensuremath{T_{eff}}}
\newcommand{\logg}{\ensuremath{\log(g)}}
\newcommand{\kms}{\textrm{~km~s}\ensuremath{^{-1}}}

% Targeting flags
\newcommand{\STARBAD}{\texttt{STAR\_BAD}}
\newcommand{\STARWARN}{\texttt{STAR\_WARN}}
\newcommand{\VSINIWARN}{\texttt{VSINI\_WARN}}
\newcommand{\APOGEESEISMO}{\texttt{APOGEE2\_SEISMO\_DWARF}}

\newcommand{\gvs}{\authorcomment1}

\shorttitle{Comparison of Rotation Measurements in \Kepler{}}
\shortauthors{Simonian et al.}
\begin{document}

\title{Rapid Rotation in the \Kepler{} Field: Not a Single Star
Phenomenon}
\author{Gregory V. Simonian; Marc H. Pinsonneault; Donald M. Terndrup}
\affil{Department of Astronomy, The Ohio State University}
\affil{140 West 18th Avenue, Columbus, OH 43210}
\email{simonian@astronomy.ohio-state.edu}

\begin{abstract}
    The \Kepler{} fields have yielded a large data set of photometric rotation
    periods, which when combined with \Gaia{} parallaxes and APOGEE
    high-resolution spectroscopic data for over 4,000 dwarfs and subgiants, 
    can reveal significant insights into the evolution of rotation in stars, 
    We report on spectroscopic measurements of stellar rotation from APOGEE in an
    easily-characterized sample of over 600 cool dwarfs and subgiants in the
    \Kepler{} field. Our data are in good agreement with literature values for
    asteroseismic and Pleiades targets, with a detection threshold between 7 
    and 10 \kms. We find a rapid rotation fraction among the photometric
    binaries of \(\sim 25\%\). When comparing spectroscopic to photometric 
    rotation rates, we find that the photometric rapid rotator 
    fraction is lower than the spectroscopic fraction by a factor of 4--5, yet
    still greatly enhanced in the photometric binaries. A point-by-point 
    analysis reveals that almost all targets have rotation periods too long 
    for their measured \vsini{}. We find these results suggest that either 
    significant contamination of field \vsini{} measurements by photometric 
    binaries with blended spectral features, or current large-scale 
    photometric surveys may mis-characterize or exclude 
    cool rapid rotators in the field. 
\end{abstract}
\keywords{stars: rotation}

\section{Introduction}

One of the significant results from the \Kepler{} satellite has been the
determination of rotational periods for a large sample of old, field stars
\citep{Basri11,Affer12,Nielsen13,Reinhold13,McQuillan14,Garcia14}.
These rotation rates are expected to be extremely useful to deriving a field
gyrochronology, which can better characterize the ages of
\Kepler{} stars, and hence better understand discovered planets.


Current attempts to synthesize rotation information with known ages of
asteroseismic targets in the \Kepler{} field have consistently found a dearth
of long-period (\(P_{rot} \sim 60\) day) rotators \citep{Angus15,vanSaders16}. 
Forward modeling of the \Kepler{} field using existing rotation-activity
relations and estimates of the \Kepler{} selection function further confirm 
that models significantly under-predict the number of (\(P_{rot} > 30\) day) 
rotators in the \Kepler{} field \citep{vanSaders18}. Proposed solutions for 
relieving this tension have been modifying the activity relation in models to 
drop precipitously at a particular rotation threshold, or modifyling angular 
momentum loss to shut off at \(P_{rot} \sim 30\) days \citep{vanSaders18}.

Rotation has proven to be puzzling on the rapid end as well. The \Kepler{} 
field was chosen to point outside of the plane of the galactic
disk, biasing the field toward older stars.  Despite the old age of the
field, there is still a population of rapid rotators at the level of 10\% 
which have rotation rates significantly higher than expected from
existing gyrochronological relations \citep{McQuillan14}. Possible explanation
for these rapid rotators are a truly young stellar population, a background of
tidally-synchronized binaries, contamination of background pulsators in the
\Kepler{} PSF, or confusion with rapidly-rotating subgiants \citep{vanSaders13}. 

A substantial population of young stars would be a significant change in our
view of the \Kepler{} field. Already, more detailed models of the
\Kepler{} field are indicating that there may be more young stars than
previously thought \citep{vanSaders18}. In addition, a widespread presence of
Lithium may indicate that planet-hosting stars may be more youthful than
expected \citep{Berger18a}. Rotation will provde an independent view into youth
in the \Kepler{} field.

In addition to youth, the discovery of a significant population of
tidally-synchronized binaries would also be an exciting find.
Tidally-synchronized binaries do not form through the typical fragmentation
process, but rather from dynamical interactions on the pre-main sequence
\citep{Bate02}. However, tidally-synchronized binaries have been difficult to
characterize because they are rare \citep{Patience02,Raghavan10}. A substantial
number of new tidally-synchronized systems would be extremely useful for
characterizing the population.

Previous studies have succesfully combined rotational periods and \vsini{} 
to study systematics in the radii of pre-main sequence stars
\citep{Rhode01,Jeffries07}, the inclination distribution in clusters 
\citep{Jackson10}, and the inference of radius inflation
\citep{Jackson09,Jackson16,Jackson18}. All of these studies were done for
clusters, where large numbers of spectra could be observed and analyzed under 
the assumption of uniform distances and composition. The availability of
rotation periods and \vsini{} for field stars will be instrumental in 
illuminating other phenomena such as gyrochronology 
\citep{Barnes07,Mamajek08,Angus15}, radius inflation \citep{Jackson18}, and 
rotation on the subgiant branch \citep{vanSaders13}.

In this paper, we study the rotation distribution of \Kepler{} targets as
seen from APOGEE \vsini{}s and compare to published periods. In 
Section~\ref{sec:data} we describe the various catalogs we synthesize to
successfully determine binarity and rapid rotation.
In Section~\ref{sec:analysis}, we show how we obtain a well-characterized
sample of cool dwarfs with rotation information. In
Section~\ref{sec:vsini_check} we validate the quality of the APOGEE \vsini{}
measurements as well as demonstrate that spectroscopic and photometric rotation
measures agree in the 
APOKASC asteroseismic sample. In Section~\ref{sec:results}, we demonstrate that
rapid rotation is primarily seen in photometric binaries, and that there is
disagreement between photometric and spectroscopic rotation measures in the
cool dwarfs. In Section~\ref{sec:discussion}, we compare the two measures of 
rotation on a point-by-point basis and postulate reasons for disagreement. 
Lastly, we summarize our findings and discuss future efforts in 
Section~\ref{sec:conclusions}.

\section{Catalog Data}
\label{sec:data}

\begin{figure}[htb]
    \plottwo{apogee_selection}{mcquillan_selection}
    \caption{\emph{Left: } The characteristics of the base APOGEE samples 
        according to the Kepler Stellar Parameter Catalog DR25. The
    black points denote the photometric \Teff{} and \logg{} of the full
    sample observed by \Kepler{}. Red points denote the cool dwarf sample 
    observed in APOGEE-1. The blue and green points denote dwarfs and giants 
 observed in APOGEE-2. \emph{Right: } The characteristics of the
 \citet{McQuillan14} nondetection and detection samples. Light pink points show
 the location of period nondetections while the purple points show targets with 
 period detections.}\label{fig:selection}
\end{figure}

This study makes use of two rotation samples in the \Kepler{} field: one
smaller but well-characterized sample with spectroscopic \vsini{} observed by
APOGEE, and a significantly larger sample with starspot variability periods
from \citet{McQuillan14} without spectroscopic characterization.

In order to calculate absolute magnitudes for both samples, we use photometry
from 2MASS \citep{Skrutskie06}, parallaxes from Gaia DR2 \citep{Gaia18}, and
extinction values estimated from the \Kepler{} Stellar Parameter Catalog DR25
\citep{Huber14,Mathur17}.

Lastly, APOGEE \vsini{} have been a relatively new addition to the catalog and
have not been extensively validated. We perform our own validation in two ways:
direct comparison to literature \vsini{} for overlapping samples, and inference
of \vsini{} from period and radius for an asteroseismically-characterized 
sample. Details are all of the input catalog as given below.

\subsection{Spectroscopic Parameters}

The spectroscopic sample for this study was drawn from two observing programs by the APOGEE
survey, both designed are magnitude and color-selected samples of
\Kepler{} targets observed over the two phases of the APOGEE program, APOGEE-1
and APOGEE-2. The APOGEE-1 survey started operations as a bright-time
survey as part of the third generation of the Sloan Digital Sky Survey
\citep{Eisenstein11}. The workhorse behind the survey is the APOGEE instrument,
a high-resolution (R \(\sim\) 22,000) multi-fiber near-infrared spectrograph
\citep{Wilson10}, mounted on the SDSS 2.5-meter telescope at Apache Point
Observatory \citep{Gunn06}. APOGEE-1 was then extended into the fourth stage of
the Sloan Digital Sky Survey \citep{Blanton17} as APOGEE-2. \gvs{Fill in science
requirements of APOGEE-2 vs APOGEE-1}

In the first phase of APOGEE (APOGEE-1) \citep{Majewski17}, around 1000 dwarfs
with \(7 \le H \le 11\), photometric temperatures cooler than 5500 K, and
\logg{} greater than 4.0 \citep{Brown11,Pinsonneault12} were observed as part
of an ancillary science program. A broader effort in APOGEE-2 to completely observe
\Kepler{} targets with \(7 \le H \le 11\) with photometric temperatures between
5000--6500 K and photometric \(\logg > 3.5\), as well as targets with
photometric \(\logg < 3.5\) and photometric temperature less than 5500 K  
\citep{Pinsonneault12,Huber14} has added around 3,500 dwarfs and 3,500 giants 
as of APOGEE Data Release 14 \citep{Zasowski17}. Observations for additional 
objects such as Kepler Objects of Interest and eclipsing binaries were also 
obtained, but with more complex selection functions. In the interest of keeping
a sample with a well-defined selection function, we exclude these other
targeting programs.




Reduction of the APOGEE data takes place in three main stages. First,
individual visit spectra are reduced, including detector calibration, bad pixel
masking, wavelength calibration, sky substraction, and determination of
individual radial velocities through cross-correlation of template spectra.
Second, the individual spectra are combined by correcting for radial velocity
differences between the exposures, either by cross-correlating with each other,
or with template spectra, whichever leads to a smaller scatter (Holtzman et
al, in prep). The full process is detailed in \citet{Nidever15}.

The final step is the extraction of stellar parameters and chemical abundances
by the APOGEE Stellar Parameter and Chemical Abundances Pipeline (ASPCAP)
\citep{GarciaPerez16}. ASPCAP measures stellar parameters by performing a
chi-squared minimization \citep{AllendePrieto06} over a 6-dimensional space of
\Teff, \logg, [M/H], [\(\alpha\)/M], \vsini, and microturbulent velocity.

After determination of the stellar parameters, they are calibrated to the
photometric system of \citet{GonzalezHernandez09} using a metallicity-dependent
offset. After calibration, the scatter between the calibrated and photometric
temperatures  were found to be a strong function of \Teff{}, ranging from 130 
K at 5500 K down to 85 K at 4000 K (Holtzman et al.\ in prep) for solar
metallicity objects observed at a signal-to-noise ratio of 100. 



\subsection{Time-Series Data}

Data from the \Kepler{} satellite \citep{Borucki10, Koch10}, which took data
over the same region of the sky for over four years, has revolutionized 
the field of stellar rotation, and forms the
basis of our photometric rotation sample. It has only been
through the exquisite precision of \Kepler{} just below 1000 part per million for
the general \Kepler{} sample and regular sampling that starspot variability 
could be measured for tens of thousands of stars.

We use photometric rotational periods from \citet{McQuillan14}, which is
to-date the largest homogeneous analysis of rotational variability in
\Kepler{} stars. \citet{McQuillan14} selected their sample to have photometric
\(\Teff < 6500\) K \citep{Brown11,Dressing13}, and implemented a color-color
and temperature-dependent \logg{} cut \citep{Brown11} advocated in
\citet{Ciardi11}, which ranged from 3.5 at 6000 K to 4.0 at 4250 K. 
In addition to the temperature and gravity cuts, \citet{McQuillan14} also 
excluded known eclipsing binaries and plantary
transit candidates. 

Rotational periods in \citet{McQuillan14} were measured by calculating the 
autocorrelation function and fitting the location of multiple peaks. In lieu of
visual verification of the autocorrelation function, \citet{McQuillan14}
performs two automated tests to distinguish physical periodicity from
instrumental artifacts or other sources of variability. First, the 
periodicity must be consistent in different segments of the light curve.
Secondly, the height of the first peak must be larger than a temperature- and
period-dependent threshold. In order to reduce contamination from pulsators, 
only periods between between 0.2-70 days are considered.


\subsection{Astrometric Data}

The ability to distinguish between different evolutionary states of stars is
enabled by parallaxes derived from the \Gaia{} mission's Data Release 2
\citep{Gaia18}. The Gaia DR2 performed a fully consistent single-star
5-parameter (\(\alpha\), \(\delta\), \(\mu_\alpha\), \(\mu_\delta\), \(\pi\))
solution to 1.3 billion sources over 22 months of observations
\citep{Lindegren18}. For targets which could not be adequately fit with a
single-star 5-parameter solution, a 2-parameter (\(\alpha\), \(\delta\))
solution is performed instead \citep{Michalik15}. The criteria for a
2-parameter fall-back solution include targets with Gaia \(G > 21.0\), having
fewer than 6 visibility periods of observations, and an error ellipse larger
than a magnitude-dependent limit in its largest dimension \citep{Lindegren18}.

% APOGEE Targeted sample APOKASC (APOKASC + Cool Dwarf - faint dwarfs) - 7642
% Not bad APOGEE targeted sample - 7395
% Good APOGEE with good K detection - 7373
% Good APOGEE in Gaia - 6961

We use distances derived from \Gaia{} parallaxes included in the cross-matched
database of asterometric radii of \Kepler{} targets in \citet{Berger18b}.
\citet{Berger18b} based their sample on the Kepler Stellar Parameter Pipeline
DR25 \citep{Mathur17}, and cross-matched those sources against \Gaia{} DR2. Of
the cross-matched targets, objects with fractional parallax errors greater than
0.2, \(\Teff < 3000\) K, \(\logg < 0.1\), or missing JHK photometry were
excluded. \gvs{Do I need a full accounting of why certain objects were not
included?} Parallaxes were converted to distance by using an
exponentially-decreasing volume density prior with a 1.35 kpc length scale
\citep{BailerJones15,Astraatmadja16}, and correcting for the zero-point offset
of 0.3 mas \citep{Lindegren18} \gvs{Since I'm converting to distance modulus
    anyway, would it be more appropriate to go directly from parallax to
    distance modulus correctly, as opposed to going from parallax to distance
correctly, and then distance to distance modulus incorrectly?}. In addition to the most probable solutions, the
1-\(\sigma\) asymmetric confidence intervals for distances were also reported
by \citet{Berger18b}.

The analysis by \citep{Berger18b} also provided classifications of objects
based on evolutionary state and binarity. Before settling on specialized
evolutionary state classifications for our analysis, we use objects in their
``Main Sequence'' classification to roughly characterize the dwarf sample. For
example, the median fractional uncertainty in the distance is \gvs{NUM}.

\subsection{Photometric Data}

In order to characterize our stellar sample, we chose to use absolute K-band
magnitude as a proxy for luminosity. K-band apparent magnitudes were collected 
from the 2MASS survey 
\citep{Skrutskie06}, which has profile-fit photometry with uncertainties 
available for nearly the full sample. A small fraction of targets have only 
upper limits reported because of blending contamination. \gvs{Typical K error} 
Because the distance error and photometric
errors are both expected to have small contributions to the absolute K-band
magnitude, on the order of 0.01 mag, we chose to use K-band magnitudes to
minimize the impact of extinction.


Since the full probability density functions for the distances are
computationally intensive to compute, and unnecessary for our purposes, we
approximate the confidence interval of \(M_K\) by adding in quadrature the 
K-band 1-\(\sigma\) uncertainty to the appropriately-scaled asymmetric upper 
and lower 1-\(\sigma\) confidence intervals for the distance separately. 
Because most of our sample is either strongly dominated by photometric errors 
or distance errors, and our results do not depend sensitively on the behavior 
of the K-band absolute magnitude. This treatment does an appropriate job of
approximating the uncertainties.

\subsection{Extinction}

While extinction is low in the K-band, we still attempt to correct for it given
the current best estimates. We use extinction values from the Kepler Stellar
Parameter Catalog DR25 \citep{Mathur17}. These extinctions were derived by
performing fits to broad- and intermediate-band photometry using the
methodology described in \citet{Huber14}. For our main-sequence targets,
typical values of \(A_V\) are \gvs{NUM} with a median uncertainty of \gvs{NUM}.

We converted the estimated \(A_V\) from the catalog to \(A_K\) using the
\(\frac{A_K}{E(B-V)}\) value in \citet{Cardelli89}, and assuming the
canonical Milky-Way \(\frac{A_V}{E(B-V)} = 3.1\). The uncertainties reported in
the Keper Stellar Parameter Catalog were propagated to the unextincted absolute
K-band magnitude uncertainty by adding the upper and lower confidence intervals
to the absolute magnitude in quadrature.

\subsection{Validation Datasets}

We use several previously studied datasets to validate the APOGEE
\vsini{} values, and to check the uncertainties inherent to APOGEE data.

\subsubsection{Direct \vsini{} comparison}

An independent spectroscopic analysis which has significant overlap with APOGEE
is that by \citet{Bruntt12} for a sample of asteroseismic targets.
\citet{Bruntt12} analyzed spectra of 93 \Kepler{} stars showing solar-type
oscillations using two \(R \sim 80,000\) spectrographs 
on the 3.6-meter Canada--France--Hawaii Telescope and 2-m Bernard Lyot
Telescope. While uncertainties are not quoted in \citet{Bruntt12},
previous analyses have determined \vsini{} uncertainties of about 0.6
\kms{} \citep{Bruntt10a,Bruntt10b}. \gvs{Check this number more carefully.} The
number of APOGEE targets overlapping with \citet{Bruntt12} is \gvs{NUM}.

Because the \citet{Bruntt12} asteroseismic targets, which are largely subgiants with
\(\Teff \sim 6000 K\) lie in a different temperature regime than the cool
targets, we also compare APOGEE \vsini{}s to measurements of cool 
dwarfs in the Pleiades \citep{Stauffer87}. Because Pleiades dwarfs are 
significantly younger than those in the field,
they rotate much more rapidly. This rapid rotation is advantageous
because allows for validation over a wide range of \vsini{}. However, 
the \citet{Stauffer87} survey 
reported a detection limit of 10 \kms, which is significantly below that in
\citet{Bruntt12}. As a result, the \citet{Stauffer87} sample only
verifies that APOGEE \vsini{} measurements are not severely different
for cool dwarfs.

\subsubsection{Comparison to asteroseismic periods}

In addition to direct comparisons to literature \vsini{}, we validate the 
APOGEE \vsini{} for a larger sample by comparing \vsini{} to velocities derived
from rotation periods and asteroseismic radii. This validation can be performed
on the significantly larger APOKASC catalog
\citep{Pinsonneault14,Pinsonneault18} to further test the APOGEE \vsini. The 
APOKASC asteroseismic dwarf and subgiant 
sample consists of 426 targets which presented solar-type oscillations in 
\Kepler{} short-cadence data \citep{Chaplin11}, and were observed in APOGEE DR13
\citep{Majewski17}. Using asteroseismic scaling relations, the radii
for these objects can be determined to an accuracy of around 3\%
\citep{Serenelli17}. 

Rotation periods for a subset of 310 asteroseismic dwarfs and subgiants have been reported 
by \citet{Garcia14}. Contrary to \citet{McQuillan14}, the \citet{Garcia14}
periods were derived using a wavelet analysis, and periods between 1--100 days
were considered. The overlap of these targets with APOGEE is \gvs{NUM}.

\section{Stellar Models}

Our fiducial single-star model is the Dartmouth Stellar Evolution Program
\citep[DSEP]{Dotter08}. The DSEP models incorporate gravitational settling of
Helium and metals, standard convection including core overshooting, and PHOENIX
model atmospheres to set the surface boundary condition \citep{Hauschildt99a,
Hauschildt99b}. Stars containing a Helium flash are not continuously modeled,
but are re-generated starting from the zero-age horizontal branch; however, as
we are only interested in the main sequence, this should not be a
concern\citep{Dotter08}.

The DSEP models have proven successful at producing the optical color-magnitude
diagram and luminosity functions of globular clusters at a wide range of
metallicities \citep{Dotter07}. It has also been successful at predicting
near-infrared isochrones in M67 \citep{Sarajedini09}. As such, we expect it to
do well at predicting K-band magnitudes for \Kepler{} cool dwarfs.

In order to derive K-band magnitudes for the spectroscopic sample, we 
interpolate over \Teff{} and [Fe/H]
assuming solar alpha abundance and a fiducial age of 5.5 Gyr, representative
of the typical age of the thick disk. For the sample drawn from
\citet{McQuillan14}, without spectroscopic [Fe/H], we assume the mean
metallicity of the overlap sample with APOGEE\@.


\subsection{Comparison to other isochrones}

To probe the systematic uncertainties of the DSEP isochrone, we compare them to
the MIST isochrones \citep{Dotter16, Choi16}. The MIST isochrones are based on 
the outputs of the MESA stellar evolutionary models \citep{Paxton11, Paxton13, 
Paxton15}. The MIST isochrones adopt different physical prescriptions for their
models, including atomic diffusion prescriptions, reaction rates, boundary
conditions, mixing length, and overshooting prescriptions. We refer the reader
to the source papers for further details on the internal physics of both
models.

We use version 1.1 of the non-rotating MIST isochrones with solar alpha
abundance. The isochrones are interpolated over \Teff{} and [Fe/H] in an
identical manner. 

\begin{figure}[htb]
    \centering
    \plotone{MIST_DSEP_age}
    \caption{Difference in predicted K-band magnitude between solar-metallicity 
    MIST and DSEP isochrones for three ages: 1 Gyr (solid), 5.5 Gyr (dashed), 
and 10 Gyr (dotted).}\label{fig:mist_dsep_age}
\end{figure}

The first test of the isochrones is of the age and \Teff{} dependence of the
solar metallicity isochrones. At three given ages, we generate the
solar-metallicity Teff-Ks relation, and subtract the two relations from each
other. The residuals are shown in \cref{fig:mist_dsep_age}. For the 5.5 Gyr
isochrone, the uncertainties are less than 0.05 mag between \(4500 K \le \Teff
\le 5900 K\). At higher temperatures, the larger differences are driven by
different behaviors near the turn-off. Because we are interested in measuring
luminosity excesses above the main sequence that are not driven by age
evolution, stars hot enough to have evolved significantly in their lifetime
will be excluded from the analysis. The disagreement below 4500 K is more
puzzling. Additional analysis will have to be done to ensure that the
isochrones are not doing terribly here. \gvs{I note that in
    \citet{Sarajedini09}, the region where the DSEP and MIST isochrones
    disagree is the same region in M67 where J-K color becomes insensitive to
    \Teff{}. So more work will have to be done to show that the DSEP isochrones
aren't going crazy in this regime.}

\begin{figure}[htb]
    \centering
    \plotone{MIST_DSEP_metallicity}
    \caption{Difference in predicted K-band magnitude between 5.5 Gyr 
        MIST and DSEP isochrones for six temperatures: 6000 K (purple),
        5500 K (turquoise), 5000 K (blue), 4500 K (aqua), 4000 K (green), and
    3500 K (yellow).}\label{fig:mist_dsep_metallicity}
\end{figure}

After establishing the behavior of the isochrones at solar metallicity, we also
trace their behavior at different metallicities.
\cref{fig:mist_dsep_metallicity} shows the predicted K-band magnitudes between
DSEP and MIST at given \Teff{}s over a wide range of metallicities. The
isochrones generally behave well in the regime warmed than 4500 K, below which
the isochrone behaviors diverge substantially. Particularly near solar, the
uncertainties are on the order of 0.05 for \(\Teff > 4500 K\).

Based on this comparison between isochrones, we conclude that the predicted 
DSEP K-band magnitudes have a systematic uncertainty of 0.05 for when
\(Teff > 4500 K\) and 0.15 when \(\Teff < 4500 K\).

\subsection{Comparison to Data}

\begin{figure}[htb]
    \centering
    \plotone{metallicity_bins}
    \caption{\Teff-K diagram of four metallicity bins of stars classified by 
        \citet{Berger18b} as being on the main sequence. The red dashed lines
    indicate DSEP isochrones with metallicity given by the bin 
edges.}\label{fig:metallicity_bins}
\end{figure}

In addition to measuring the systematic uncertainty of the isochrone with
respect to other isochrones, we also attempt to measure the error of the
isochrone with respect to the data. To accomplish this, we split the data into
four metallicity bins, and attempt to measure the scatter of the points with
respect to the isochrones. The bins are shown in \cref{fig:metallicity_bins} 
\gvs{I realize that this isn't actually a good
strategy to quantify the uncertainty of the isochrone. Will need to discuss
with Marc.} 

\subsubsection{Low metallicity}

Of all the bins, it's clear that the lowest metallicity bin does not match the
theoretical isochrones at all. While it's possible that the isochrones predict
the K-band magnitudes of metal-poor stars poorly, it's more plausible that
metal-poor main-sequence stars are more likely to be excluded due to the
magnitude-limited sample. Aside from being intrinsically less luminous at a
given temperature, metal-poor stars are also rare in the thick disk, and would
only be found at greater distances. Due to both of these effects, it's likely
that only the brighest metal-poor stars, namely those in multiple star systems 
or otherwise peculiar would be included in APOGEE's selection criteria.

To test this, we take the distances of the observed metal-poor objects, predict
a main-sequence H-band absolute magnitude based on the temperature, and check 
to see if it would be fall under APOGEE's inclusion criteria. The distribution
of predicted apparent H-band magnitudes are shown in \cref{fig:metal_poor}.
\gvs{NOTE: Actually do this exercise.}

\section{Data Validation}
\label{sec:analysis}

\subsection{Spectroscopic parameters}

The first step to obtaining a usable spectroscopic sample is to perform a 
quality cut for all targets with APOGEE observations.
Depending on the reduced chi-square of the atmospheric model fit, or
disagreement with photometric temperature relations, the ASPCAP pipeline
may trigger a flag indicating that the ouputted parameters may not be
reliable. We categorized our sample into those with \STARBAD, \STARWARN,
\VSINIWARN{}, and no flag enabled to check the quality of the fits. We
visually inspected the spectra for a representative sample of these
quality classes. 
Visual inspection of spectra for targets with the \STARBAD{} flag showed that they are 
largely the result of poor subtraction or normalization of the spectrum in the
pipeline, or are due to poor model fits on the cool (\(\Teff < 4250\) K) end. 
\gvs{Calculate percentage} in the full sample have the \STARBAD{} quality 
flag enabled, which have been excluded from the rest of the analysis. 
Targets labeled as \STARWARN{} are a significantly larger fraction of the 
sample. We found that the model fits reasonably resembled the target spectra
based on a visual analysis. 
For this reason, we retained targets with the \STARWARN{} flag enabled. 
Targets with the \VSINIWARN{} flags also had visually reasonable spectra.

APOGEE \Teff{} and [Fe/H] uncertainties have largely been characterized.
Holtzmann et al (in prep) calculated an expression for the temperature scatter
for cool dwarfs, which ranges from 135 K at 5500 K to 85 K at 4000 K at solar
metallicity. This is consistent with the scatter between photometric and
spectroscopic \Teff{} measured for the APOKASC dwarfs \citep{Serenelli17}.
Uncertainties in [Fe/H] have also been found to be on the order of 0.1 mag
\citep{Serenelli17}. \gvs{Can also try cross-checking with CKS.}

\subsection{\vsini{} validation}
\label{sec:vsini_check}

\begin{figure*}
  \gridline{\fig{Bruntt_comp}{0.3\textwidth}{(a)}
  \fig{Pleiades_comp}{0.3\textwidth}{(b)} 
  \fig{astero_rot}{0.3\textwidth}{(c)}}
    \caption{\emph{Left:} \vsini{} comparison between the \citet{Bruntt12}
        overlap sample with APOGEE\@. A discontinuity in the scatter occurs
        around \(\vsini = 7 \kms\), indicated by the dotted line. The dashed
    line shows the best-fit relation between the two. Not shown are targets 
    run through the APOGEE giant grid. \emph{Middle:} \vsini{} comparison for 
    the Pleiades cool dwarfs \citep{Stauffer87} overlap sample with APOGEE\@. 
    A discontinuity in the scatter occurs around \(\vsini = 12 \kms\), 
    indicated by the dotted line. 2MASS J03475973+2443528 is not shown
    because \citet{Stauffer87} flagged it as a possible SB2. Red points are 
    upper limits in \citet{Stauffer87}.\emph{Right:} Comparison between
    \vsini{} and equatorial \(v_{eq} = \frac{2\pi R}{P}\) for the 
    asteroseismic sample. Dark blue points correspond to confirmed
    \vsini{} detections while light blue points correspond to marginal
    \vsini{} detections. The lines corresponding to \(\sin i = 1\) and
    \(\sin i = 0.5\) are denoted as solid and dashed lines. The hatch
    marks denote the forbidden region where \(\sin i > 1\).\label{fig:comps}}
\end{figure*}


The availability of published \vsini{} is relatively new to the ASPCAP pipeline, only
being included since DR13. As a result, catalog \vsini{}s have not been 
extensively vetted and verified. APOGEE does not provide a reasonable
detection limit, or an estimate on the uncertainty of \vsini{}. 
\citet{Tayar15} performed an analysis comparing APOGEE \vsini{} to literature
\vsini{} for targets studied with high-resolution optical spectroscopy to 
determine the detection limit and
uncertainty for \vsini{} for APOGEE DR12 attempting to detect rotation in red
giants and found a detection threshold of about 7 \kms. We repeat their 
analysis for DR14 spectra in the cool dwarf regime to ensure their results
continue to hold in this new regime. 

A comparison between the overlapping APOGEE and \citet{Bruntt12} targets is shown in
\cref{fig:comps}. Six objects without \vsini{}, classified by the ASPCAP 
pipeline as giants, are not shown. The \citet{Bruntt12} \vsini{} is less than 5 \kms{} for all of 
these objects, making them consistent with APOGEE
nondetections.  In order to characterize the APOGEE \vsini{}s, we model the 
two datasets as being linearly related above a specific detection threshold 
with constant fractional uncertainty. The sample shows a 
discontinuity in the scatter at APOGEE \(\vsini=7\kms\), which we 
interpret as the detection limit of APOGEE, above which \vsini{} yields a 
valid stellar rotation measurement. Since the \citet{Bruntt12} errors are
significantly smaller than the expected APOGEE errors, we performed a simple 
linear regression for all targets with APOGEE \vsini{} above the detection 
limit to obtain the relationship. The two datasets have slopes
consistent with one, but with a significant zero-point offset of 9\%,
the APOGEE \vsini{} being lower than the \citet{Bruntt12}
\vsini{}. We estimated the measurement error by calculating the mean squared 
error of the residuals, yielding a 13\% uncertainty which we adopted as the 
uncertainty in APOGEE \vsini{}.

To make sure that \vsini{} continues to be a reliable measure of rotation for
cool dwarfs, we plot the APOGEE \vsini{} agains the \vsini{} measured from
\citet{Stauffer87} in \cref{fig:comps}. At high \vsini{}, the two datasets 
agree well, even
without the offset seen in the \citet{Bruntt12} sample. Near the the 10
\kms{} \citet{Stauffer87} detection limit, they find a few targets with
much higher \vsini{} than APOGEE detects. While this may be
underestimated noise near the detection limit for \citet{Stauffer87}, we
note that this could also indicate that genuine rapid rotators may be
missed by APOGEE\@.

\gvs{Does this actually support the point? Or is it a distraction now?} 
As an additional check on the quality of \vsini{}, we used the asteroseismic 
sample to compared the measured APOGEE \vsini{} to the equatorial velocity 
derived from the rotational periods \citep{Garcia14} and the asteroseismic 
radii \citep{Serenelli17}. The resulting 
comparison is shown in the right panel of \cref{fig:comps}. We find that
twelve out of \gvs{NUM} asteroseismic targets have
non-physical solutions with \(\sin i > 1\). \gvs{The overlap with Garcia is
much larger, so I need to double-check these}

Seven of the targets with unphysical \vsini{} have \vsini{} values close to 
the detection threshold. Because the errors of 
marginal \vsini{} detections are complicated, it may be possible that these 
are outliers. 

Two of the remaining objects with unphysical \vsini{}, are only
slightly above the predicted equatorial velocity. \gvs{Check the UKIRT images
for these, just in case.} It
is a 1.7-\(\sigma\) difference based on the \vsini{} uncertainty, which
may be consistent as a statistical fluctuation. 

Two other targets can be explained by the measured rotation period actually 
being a half-period detection alias, implying a twice as large equatorial 
velocity. 

The final outlier, KIC 9908400, is so significantly off of the relation
that it can not be explained by either uncertainties or aliasing. However,
it may be adequately explained by contamination of the \Kepler{} PSF\@. 
High-resolution J-band imaging of the \Kepler{} field, provided by the United
Kingdom InfraRed Telescope (UKIRT) reveals a bright, nearby blended companion.
Additionally, KIC 9908400 has a high \Teff{} placing it above the Kraft
break, where convective envelopes are thin;
making spots unlikely. Photometric variation originating from the fainter 
star would explain the mismatch in \vsini{} and rotation period.

To obtain a reliable sample of \vsini{} in our full sample, we also inspected 
the spectra for every target with \(\vsini > 7 \kms\) for the presence of 
double lines. Most targets showed obvious cases of ASPCAP 
fitting a single broad line over two separate narrow features, other cases 
only revealed the second component in individual visit spectra. The targets 
showing double-lined spectra were excluded from the analysis because of
the unreliability of the catalog \vsini{}. More sophisticated future
analyses may be able to disentangle \vsini{} measurements from each component 
\citep{ElBadry18}. There are
also cases where the observed spectra show high RV variability
without a visible second component. We include these cases in our sample
because the measured \vsini{} shouldn't be impacted.

For typical APOGEE 
dwarfs in our temperature range, the claimed precision is about 0.01
dex (Holtzmann et al.\ in prep). \gvs{What did Serenelli et al (2017) say about
metallicity uncertainties?}

\section{Binary Selection}

Characterizing the binary population in the rapid rotators requires separating
photometric binaries from other classes of objects which lie above the main
sequence, such as subgiants, metal-rich stars, or high-noise outliers. We first
attempt to exclude subgiants from the sample, then we select a \Teff{} regime
where the dwarf and subgiant branches are sufficiently separated that
photometric binaries can be easily distinguished. Lastly, we classify stars as
being part of photometric binaries if they lie sufficiently off the main
sequence.

\subsection{Subgiant Exclusion}

\begin{figure}[htb]
    \centering
    \plotone{turnoff}
    \caption{10 Gyr DSEP models of [Fe/H] = -0.5 (blue), 0.0 (black) and 0.5
    (red) overlayed on the spectroscopic sample. The absolute K-band magnitude
    of the turnoff at all metallicities is shown as the dashed line at \(M_K =
2.7\). Points above the line are a mixture of photometric binaries, main
sequence stars, and subgiants. Points below should be free of 
subgiants.}\label{fig:turnoff}
\end{figure}

We select a regime which should be devoid of subgiants by modeling the turnoff
for a 10 Gyr population, representative of the oldest thick-disk stars.
Zoom-ins of the 10 Gyr turnoff are overlayed on our sample for three different
metallicities in \cref{fig:turnoff}. One useful property of the DSEP
predictions is that the turnoff K-band magnitude is independent of metallicity.
As a result, we restrict ourselves to the sample less luminous than \(M_K =
2.7\) to ensure a clean sample of dwarfs.

\subsection{Unbiased Binary Regime}

\begin{figure}[htb]
    \centering
    \plotone{binarycut}
    \caption{Magnitude difference between each point in the spectroscopic
    sample and the 5.5 Gyr DSEP isochrone with the corresponding [Fe/H]. Green
points are those which meet the magnitude cut in the previous section, cleaned
of subgiants. The horizontal dotted line represents a factor of two increase in
luminosity above the predicted main sequence, characteristic of binaries. The
vertical dashed line indicates the temperature where the photometric binaries
are separated from the subgiant branch.}
    \label{fig:binarycut}
\end{figure}

While the sample has been selected to exclude subgiants, binary statistics on
the hot end will be biased because photometric binaries would fall above the
magnitude cut. This bias is seen in \cref{fig:binarycut}, which shows the
magnitude excess of the dwarf sample selected in the previous section compared
to the more luminous sample. The main sequence isochrones are adjusted for
metallicity, which should reduce the scatter about the zero-point. Above 
5500 K, the sample containing subgiants lies less than 0.75 mag above the 
main sequence, indicating that photometric binaries would have removed in the
magnitude cut. Below 5500 K, the subgiant branch lies above 0.75 mag, and the
population of photometric binaries should be complete.

\subsection{Flagging Binaries}

Binaries are flagged as targets which lie at least 3\(\sigma\) above the
DSEP-predicted single-star isochrone. This uncertainty is
defined\ldots{} \gvs{I'm still working on how to get to the uncertainties.}

\section{Results}
\label{sec:results}

\begin{figure}[htb]
  \centering
  \includegraphics[width=0.8\textwidth]{binarity}
  \caption{The absolute K-band magnitude of rapid rotators compared to a 
  fiducial 3 Gyr isochrone. Rapid rotators are shown as dark blue points
while marginal rotators are shown as light blue points. Objects showing
double-lined spectra are denoted with salmon stars. Objects with
ambiguous stellar classification are shown as pentagons. The range of
possible K-band magnitudes achievable on the main sequence are given as
solid lines. The red, black, and blue lines correspond to [Fe/H]=0.5,
0.0, -0.5, respectively.}
  \label{fig:binarity}
\end{figure}

\begin{figure}[htb]
  \centering
  \includegraphics[width=0.8\textwidth]{mcq_binarity}
  \caption{Cumulative histograms of the K-band magnitude excess with respect to
      a 3 Gyr solar metallicity isochrone for all cool dwarfs with 
      \citet{McQuillan14} periods. The black line is the magnitude excess
      distribution for the full sample while the blue light is just the rapid
  (\(P_{rot} < 3\) day) rotator sample.\label{fig:mcqbinarity}}
\end{figure}


When plotting rapid rotators on an HR diagram, it is clear that the
overwhelming majority of them lie above the main sequence. To make this
result clearer, we plot temperature against the difference between the
observed magnitude, and the magnitude predicted by the corresponding
metallicity-calibrated DSEP model at a fiducial age of 3 Gyr. The
calculated luminosity excess is shown in \cref{fig:binarity}. The range
of luminosity differences achievable by single-star models of varying
ages only reaches a 0.5 mag excess at the hottest part of the range, and
drops significantly with temperature. Most rapid rotators are beyond the
range achievable by single-star models, and must be in multiple-star
systems.

For the subset of our sample that are photometric rapid rotators, they
also show a clear preference for being multiple-star systems systems.
This is further confirmation that rapid rotation is not common in
isolated single stars.

The tendency for rapid rotators to be in multiple star systems is also
seen in the \citet{McQuillan14} cool dwarfs. To select this sample, we
use the same absolute K-band magnitude and temperature cuts as with the
APOGEE sample. Instead of spectroscopic temperatures, we use photometric 
temperature calculated in \citet{Mathur17}. Because the full \citet{McQuillan14} 
sample does not have metallicity information, we cannot glean a sharp
photometric binary sequence as was seen with the APOGEE sample. Nonetheless,
we can note general trends just by removing the overall \Teff dependence of
\(M_K\) by subtracting off  a single 3 Gyr solar-metallicity
isochrone. Plotted in \cref{fig:mcqbinarity} is the cumulative distribution
function of \citet{McQuillan14} targets with \(P < 3\) day compared to the bulk
sample. It is clear from the figure that rapid rotators tend to be found at
larger magnitude excesses than the bulk sample.

\begin{figure}[htb]
    \plotone{detection_fraction}
    \caption{Cumulative distribution function of \vsini{} for the combined
        dwarf sample for spectroscopic and photometric rotation data. The 
        rapid rotator fraction at a given \vsini{} threshold
    would be one minus the value of the cumulative distribution function
    at that \vsini{} value. Solid lines denote the cumulative
    distribution function of APOGEE \vsini{} for the subsample with rotation
    periods. Dashed lines denote  
    the cumulative distribution function of \(v_{eq}\) derived from the
    rotation periods and stellar radii. Dotted lines denote the
    cumulative distribution function of \vsini{} for the subsample without
    rotation periods. Error bars represent 
    1-\(\sigma\) binomial confidence levels.\label{fig:detection_fraction}}
\end{figure}

Given that two rotation measures are available for our sample, we also compare 
the spectroscopic and photometric rapid rotator fractions. In order to
ensure that our comparison results are not driven by the particular
\vsini{} detection limit, we perform our comparison at a variety of
\vsini{} limits to ensure that they are robust.

\gvs{Also need a uniform method of dealing with inclination biases.}
In order to obtain a photometric rapid rotator fraction corresponding to
a \vsini{} detection limit, we transformed the \vsini{} threshold into period 
space. The relationship between rotational period and equatorial velocity is 
given by the relation 
\begin{displaymath}
    v_{eq} = \frac{2 \pi R}{P} 
\end{displaymath}
where \(R\) is the equatorial radius and \(P\) is the rotation period. As we
noted in Section~\ref{sec:data}, there is a relatively clean and narrow
relationship between radius and \Teff{} on the main sequence. We used the 
DSEP-derived radius for each individual object to transform  between
velocity and period coordinates. The fraction of objects with period
short enough to produce detectable \vsini{} is shown as the dashed 
lines in \cref{fig:detection_fraction}. The photometric rapid rotator fraction
ranges from 5-8\% for photometric binaries at 0.5-1\% for single stars.

Due to the \(\sin i\) ambiguity, there isn't a direct correlation
between a \vsini{} cut and the corresponding period cut given a stellar radius.
Many rapid rotators in period-space scatter into being non-detections in
velocity space after convolution with the inclination distribution. However,
this distribution can be statistically characterized for randomly
aligned systems showing photometric variability. Studies of spots in young open
clusters indicate that spots only produce variability when \(\sin i < 1/2\). 
Since our underlying sample was chosen based on
\Teff{} and \logg{}, we only expect a strong bias in inclination caused by spot
geometry.
Therefore, we perform a statistical correction of \(\sqrt{3}/8+\pi/6=0.740\)
to account for photometric rapid rotators scattering to low
\vsini{}. Applying this correction exacerbates the disagreement between
spectroscopic and photometric rapid rotator rates because the inferred range of
rapid rotators increases to 35-67\% for binaries.

The cumulative \vsini{} distribution of our sample is shown 
in \cref{fig:detection_fraction}, with each bin corresponding to the
fraction of targets with APOGEE \vsini{} less than the given
\vsini{}. Error bars denote  binomial 1-\(\sigma\) confidence 
intervals based on selecting the number of rapid rotators from the full 
sample.  As such, the rapid rotator fraction is the complement of that
value. 

As a result, we find that the fraction of
spectroscopic rapid rotators is significantly larger than the fraction
of spectroscopic rapid rotators by a factor of 4-13. 
Accounting for
inclination effects makes the discrepancy even worse to a 4-\(\sigma\)
level. This discrepenacy is highly robust to the choice of
\vsini{} detection limit. Unlike  
the case with the asteroseismic targets, the fraction of spectroscopic rapid
rotators is greater than the fraction of photometric rapid
rotators at a highly-significant level, which is exacerbated afer taking the
inclination correction into account.

We also checked the spectroscopic rapid rotator fraction of the 
\citet{McQuillan14} nondetections for the single and binary sequences. We 
found that the spectroscopic rapid rotator fraction is significantly
lower in the nondetections than the overlap sample for the binaries, but are 
inconsistent with zero for both the binary and single stars. 
A non-zero rapid rotator fraction is surprising because the nondetection
sample is expected to be comprised of slow-rotating inactive stars and 
high-inclination spotted stars, both of which are incompatible with low 
\vsini{} values. 

\begin{figure*}
  \plotone{cool_rot}
  \caption{\emph{Left:} APOGEE \vsini{} plotted against equatorial
  velocity computed from the rotation period and radius for targets with
  detected rapid rotation. Targets which are photometric binaries are plotted
  as large circles while targets which are on the single photometric sequence
  are plotted as small stars. The confirmed \vsini{} detections are shown
  in dark blue while the marginal \vsini{} detections are shown in light
  blue. The solid and dashed lines correspond to values where \(\sin i =
  1, 0.5\), respectively. The hatched area represents the forbidden
  region where \(\sin i > 1\). \emph{Middle:} Symbols are similar to left
  side, except points are projected such that the DSEP-derived radius is
  plotted against the radius inferred from \vsini{} and rotation
  period. \emph{Right:} Symbols are similar to left side, except points are
  projected such that the \citet{McQuillan14} period is plotted against the
  period inferred from the \vsini{} and radius.\label{fig:rot}}
\end{figure*}

To further investigate the discrepancy in rapid rotation rates, we
performed a point-by-point comparison of rotation measures for these 
targets is shown in the bottom panels of
\cref{fig:rot}. The comparison between \vsini{} and equatorial velocity shows
that the discrepancy between \vsini{} and photometric period is not due to slight
but coherent model errors, or a few discrepant points. Unlike with the
asteroseismic sample, all but one of the targets
lie in the forbidden \(\sin i > 1\) region with both the confident and
marginal detections affected. This is indicative of a systematic
bias where periods are consistently (and sometimes drastically) too slow to be
consistent with \vsini{}.

\section{Discussion}
\label{sec:discussion}

Our analysis by combining APOGEE and \Gaia{} data shows that rapid
rotation is not a single-star phenomenon. This is an indication that
consideration of multiplicity is important before applying gyrochronological
relations to the \Kepler{} field. Confirmation of photometric excesses for 
both photometric and spectroscopic rapid rotators indicates that, at least
for some systems, their rapid rotation is physical.

There are two main mechanisms to explain why objects on the photometric
sequence should produce rapid rotation. One is that the system contains
a binary where the two components undergo tidal interactions. These
systems ought to have orbital periods less than 7 day
\citep{Raghavan10}. Since these are short-period systems, they're predicted 
to be strongly biased toward having equal-mass components
\citep{Bate02}. The high rate of photometric excess among the rapid
rotators supports that view. Another explanation for rapid rotation
among objects with photometric excesses is that those systems are
interaction products, and were spun up by mass-transfer. This
explanation is much more likely for the objects above the photometric
binary sequence, which may be thermally unstable.

Our result that rapid rotators tend to be binaries supports the
predictions of efforts attempting to detect stellar multiplicity from
spectra \citep{ElBadry18}. \citet{ElBadry18} analyzed all APOGEE spectra
from DR13 before \Gaia{} parallaxes were available, including a
significant fraction of this sample. They
reported that 18/19 rapid rotators on the photometric sequence and 2/2 on 
the main sequence show signs of composite spectra based on their
data-driven classification. While \citet{ElBadry18} successfully inferred the
high binary fraction of our sample, the fact that they classified both objects
with low photometric excesses as having composite spectra suggests that 
rapid rotation may be degenerately classified with a composite spectrum
in their machine-learning method.

Our other major result is that spectroscopic measures of rapid rotation 
disagree significantly with photometric measures. In particular,
photometric rapid rotation is generally confirmed by spectroscopic rapid
rotation, but not vice versa. This suggests two
possibilities: the \citet{McQuillan14} method misclassifies rapid
rotators as slow, or that the APOGEE \vsini{} have significant false
positives.

The observation that this discrepancy exists for cool dwarfs and not
asteroseismic stars is a significant clue to its origin. It indicates 
that the discrepancy is only significant for certain parts of the HR diagram. 
The main differences between the two are temperatures, with asteroseismic 
targets being significantly warmer than our cool sample, and evolutionary 
state, where asteroseismic targets are subgiants and the cool sample consists 
of dwarfs.

The discrepancy between spectroscopic and 
photometric rapid rotation fractions can be seen in other published work as well. 
\citet{Nielsen13} compared their measurements of \Kepler{} rotation periods to
an amalgamation of \vsini{} from various clusters. They found that the
\vsini{} distribution was consistent with the period distribution for spectral
types F--K, and then the periods dropped below the \vsini{} measurements for
K--M (top panel of their Fig. 2). They ascribe the discrepancy in the cool
stars to a shift in the \Kepler{} age distribution between the hot and cool
stars; however, we consider that the photometric periods may degrade for
binaries.  Additional confirmation occured through the APOGEE M-dwarf rotation
program \citep{Gilhool18}, which surveyed a color- and proper-motion-selected 
sample of nearby M-dwarfs. Unlike in this work, the \vsini{} were measured 
directly through the spectra as opposed to through ASPCAP\@. For the \Teff{} range that
overlaps with \citet{McQuillan14} \(4000 \textrm{ K } > \Teff > 3200\) K, they 
find that 5.8\% of their targets have \(\vsini > 8 \kms\). This is 
significantly greater than our photometric rapid rotator fraction at 8 \kms{} 
of 2.5\%. 

Confusion in the \Kepler{} pixels is one potential explanation for the
discrepancy. Because the \Kepler{} pixels are relatively large (4\arcsec), a 
bright, hot primary star sharing a pixel with a fainter, spottier companion 
contributing the photometric modulation in the
\Kepler{} light curve would explain the observed contradiction between
\vsini{} and rotation period. Despite this setup being unlikely, because for 
stars cooler than the Kraft break, rotation and activity are highly 
correlated, we check for bright companions
within the \Kepler{} pixel using the \Kepler{} field high-resolution UKIRT
J-band images introduced in Section~\ref{sec:vsini_check}. We did not find a 
correlation between that the presence of a bright companion and the magnitude 
of the offset, indicating that confusion is likely not the main source of the 
problem.

The presence of high \vsini{} objects in the period nondetections is
further evidence of a systematic discrepancy. Previous studies have found an 
inverse correlation between photometric variability and Rossby number (defined 
as \(R_o = P / \tau_{cz}\), where \(P\) is the rotation period and 
\(\tau_{cz}\) is the color-dependent convective overturn timescale) 
\citep{Messina01,Hartman09}. Since activity is also inversely
proportional to the Rossby number, this predicts that the stars without
spots should also be rotating slowly, which is not consistent with high
\vsini{}. As mentioned in Section~\ref{sec:vsini_check}, inclination may
also lead to period nondetections, but the low-inclination systems which
don't show starspot modulation would also be expected to show low
\vsini{}. Therefore, we conclude that the existence of 
stars with low Rossby number and low photometric variability may either 
indicate that additional factors determine photometric variability 
aside from the Rossby number, or that \citet{McQuillan14} is not detecting 
rapid rotation in genuine, spotted rapid rotators.

One other potential explanation for the mismatch between the photometric
rotation periods and \vsini{} is the presence of SB2s with a special geometry
where the two components are offset enough that their spectral features
overlap, yet not offset enough for the two components to be visually
distinguished. Binarity would also explain the difference in behaviors
between the cool dwarfs and asteroseismic targets; stars with close, but
not identical masses would have similar luminosities on the main
sequence, but significantly different luminosities when the primary
becomes a subgiant, reducing the impact of the secondary on
\vsini{}. However, the main drawback of this explanation is that
binaries in this configuration should be rarer than observed rate of
rapid rotators. These contaminants will be dominated by systems whose
orbital velocity is in a narrow range of near half the instrumental
resolution, which should be less than 1\% of the sample. However, we are
not able to discount this explanation for the large number of
spectroscopic rapid rotators.

Lastly, the explanation we propose for the behavior of rapid rotatiors is that 
the large-scale period-finding algorithms are yielding
spurious results on short-period tidally-coupled active systems. The
reliability of these algorithms have not been tested for systems that
have a superposition of spot-modulation signals at equal or near-equal
periods with a phase offset. 

While we have two potential explanations for the discrepancy between the
photometric and spectroscopic rapid rotator rate, our currently
available data is unable to distinguish between the two. The most
effective way to break the degeneracy will be to follow-up the high
\vsini{} objects with spectroscopy at higher resolution than APOGEE\@. If
the line broadening is truly due to blending, the the two components
should easily separate out. If not, then we can conclude that the
spectroscopic rapid rotation is real, and the line is intrisically
broadened.

\section{Conclusion}
\label{sec:conclusions}

We analyzed the APOGEE sample of 368 cool dwarfs in the \Kepler{} field with 
rotation periods in \citet{McQuillan14} using DSEP-derived stellar radii. One
of the first results we found was that 86\% of the rapid rotators are
photometric binaries, indicating that rapid rotation is largely the product of
binary evolution. For the binaries in particular, we
found a substantial discrepancy between the two measures of rotation
among the high \vsini{} targets, with the spectroscopic rapid
rotator rate being 4--13 times the photometric rapid rotator rate. 

A point-by-point analysis of the 49 rapid rotators in our sample revealed that
rotational periods for the entire sample are universally too long to match the 
\vsini{}. This finding implies severe systematic uncertainties in either the
measurement of \vsini{} or rotation period. We suggest several possible ways 
to investigate the underlying causes.

The best way to determine how much of the discrepancy is due to blending of
spectral line is to follow-up with high-resolution spectroscopy beyond the
resolution of APOGEE\@. Whether or not the broadening is found to be real, or to
be composed of multiple components will help determine the true rapid
rotator rate in the \Kepler{} field.

Systematic uncertainties in period determinations will be significantly more
difficult to investigate. One start may be to perform a targeted period search
at short periods for the rapid rotators where the blind search returned a long
period. The presence of a short-period signal would confirm that the
spectroscopic rapid rotators are physical. Another direction would be to use an
algorithm which can successfully identify multiple periods, as has been used
for the K2 cluster campaign \citep{Rebull16,Rebull17}. If multiple signals are
present in a light curve, these methods may be able to pick both as opposed to
favoring one over the other.

Regardless of the source of the inconsistency, we caution readers about 
interpreting large datasets of rotation in the \Kepler{} field without 
independent confirmation until the source of this inconsistency is found.


\bibliographystyle{aasjournal}
\bibliography{references}

\end{document}




